\chapter{Gestire i turni}

\section*{\huge\textbf{\textcolor{castletongreen}{Informazioni generali}}}

\begin{usecase}
    \uc{Gestire i compiti della cucina}
    \portata{Sistema}
    \livello{Obiettivo utente}
    \attoreprimario{Organizzatore}
    \partiinteressate{Chef, Cuochi}
    \precondizioni{L'attore deve essere identificato come Organizzatore}
    \postcondizioni{I turni di cucina e di servizio sono stati impostati}
\end{usecase}

\section*{\huge\textbf{\textcolor{castletongreen}{Scenario principale di successo}}}

\begin{center}

    \begin{longtable}{ | p{1.3cm} | p{7.5cm} | p{7.5cm} |}
        \hline\hline
        \rowcolor{mintgreen}
        \multicolumn{1}{|>{\centering\arraybackslash}m{13mm}|}{\textbf{\huge\textcolor{castletongreen}{\texttt{\#}}}} & \multicolumn{1}{|>{\centering\arraybackslash}m{75mm}|}{\huge\textcolor{castletongreen}{\textbf{Attore}}} &\multicolumn{1}{|>{\centering\arraybackslash}m{75mm}|}{\huge\textcolor{castletongreen}{\textbf{Sistema}}} \\ \hline
        
        \centering\textbf{1} & Aggiunge un turno di cucina e, opzionalmente segna un luogo & Registra il turno della cucina.\\\hline
        
        & \textit{Ripete il passo 1 finché non è soddisfatto} & \\\hline

        \centering\textbf{2} & Opzionalmente, controlla gli eventi. & Visualizza gli eventi.\\\hline

        & \textit{Se non vuole lavorare su nessun evento termina il Caso d'Uso} & \\\hline

        \centering\textbf{3} & Sceglie un evento su cui lavorare. & Visualizza l'evento selezionato.\\\hline
        
        \centering\textbf{4} & Imposta un orario per un turno di servizio e, opzionalmente, mette una scadenza oltre la quale non si può più togliere la disponibilità. & Registra l'orario del turno di servizio e, opzionalmente, la scadenza oltre la quale non si può togliere la disponibilità.\\\hline

        & \textit{Se vuole lavorare su altri servizi torna al punto 4} & \\\hline
        
        & \textit{Se vuole lavorare su altri eventi torna al punto 2} & \\\hline

        \centering\textbf{5} & Opzionalmente, si segna lo schema. & Registra lo schema dei turni per i servizi dell'evento selezionato.\\\hline

        \centering\textbf{6} & Opzionalmente, modifica la scadenza. & Registra la modifica della scadenza.\\\hline

        \centering\textbf{7} & Opzionalmente, cambia l'orario del turno di servizio, modificando tutte le ricorrenze. & Registra le modifiche.\\\hline

        & \textit{Se vuole lavorare su altri eventi torna al passo 2, altrimenti termina il Caso d'Uso} & \\\hline

        \hline
        \end{longtable}
      
\end{center}

\section*{\huge\textbf{\textcolor{castletongreen}{Estensione 1a}}}

\begin{flushleft}
    \begin{center}

        \begin{longtable}{ | p{1.3cm} | p{7.5cm} | p{7.5cm} |}
            \hline\hline
            \rowcolor{mintgreen}
            \multicolumn{1}{|>{\centering\arraybackslash}m{13mm}|}{\textbf{\huge\textcolor{castletongreen}{\texttt{\#}}}} & \multicolumn{1}{|>{\centering\arraybackslash}m{75mm}|}{\huge\textcolor{castletongreen}{\textbf{Attore}}} &\multicolumn{1}{|>{\centering\arraybackslash}m{75mm}|}{\huge\textcolor{castletongreen}{\textbf{Sistema}}} \\ \hline

            \centering\textbf{1a.1} & Modifica un turno di cucina. & Registra la modifica del turno di cucina. \\\hline

            \hline
            \end{longtable}
          
    \end{center}
\end{flushleft}

\section*{\huge\textbf{\textcolor{2}{Eccezione 1a}}}

\begin{flushleft}
    \begin{center}

        \begin{longtable}{ | p{1.3cm} | p{7.5cm} | p{7.5cm} |}
            \hline\hline
            \rowcolor{mintgreen}
            \multicolumn{1}{|>{\centering\arraybackslash}m{13mm}|}{\textbf{\huge\textcolor{castletongreen}{\texttt{\#}}}} & \multicolumn{1}{|>{\centering\arraybackslash}m{75mm}|}{\huge\textcolor{castletongreen}{\textbf{Attore}}} &\multicolumn{1}{|>{\centering\arraybackslash}m{75mm}|}{\huge\textcolor{castletongreen}{\textbf{Sistema}}} \\ \hline

          \centering\textbf{\textcolor{2}{1a.1}} & Aggiunge un turno di cucina e , opzionalmente segna un luogo.  & Esiste già un turno in quel orario. \\\hline
            

            \hline
            \end{longtable}
          
    \end{center}
\end{flushleft}

\section*{\huge\textbf{\textcolor{castletongreen}{Estensione 1b}}}

\begin{flushleft}
    \begin{center}

        \begin{longtable}{ | p{1.3cm} | p{7.5cm} | p{7.5cm} |}
            \hline\hline
            \rowcolor{mintgreen}
            \multicolumn{1}{|>{\centering\arraybackslash}m{13mm}|}{\textbf{\huge\textcolor{castletongreen}{\texttt{\#}}}} & \multicolumn{1}{|>{\centering\arraybackslash}m{75mm}|}{\huge\textcolor{castletongreen}{\textbf{Attore}}} &\multicolumn{1}{|>{\centering\arraybackslash}m{75mm}|}{\huge\textcolor{castletongreen}{\textbf{Sistema}}} \\ \hline

            \centering\textbf{1b.1} & Cancella un turno di cucina. & Elimina il turno di cucina specificato.\\\hline

            \hline
            \end{longtable}
          
    \end{center}
\end{flushleft}

\section*{\huge\textbf{\textcolor{castletongreen}{Estensione 1c}}}

\begin{flushleft}
    \begin{center}

        \begin{longtable}{ | p{1.3cm} | p{7.5cm} | p{7.5cm} |}
            \hline\hline
            \rowcolor{mintgreen}
            \multicolumn{1}{|>{\centering\arraybackslash}m{13mm}|}{\textbf{\huge\textcolor{castletongreen}{\texttt{\#}}}} & \multicolumn{1}{|>{\centering\arraybackslash}m{75mm}|}{\huge\textcolor{castletongreen}{\textbf{Attore}}} &\multicolumn{1}{|>{\centering\arraybackslash}m{75mm}|}{\huge\textcolor{castletongreen}{\textbf{Sistema}}} \\ \hline

            \centering\textbf{1c.1} & Raggruppa i turni consecutivi. & Unisce i turni specificati.\\\hline

            \hline
            \end{longtable}
          
    \end{center}
\end{flushleft}

\section*{\huge\textbf{\textcolor{2}{Eccezione 1c.1a}}}

\begin{flushleft}
    \begin{center}

        \begin{longtable}{ | p{1.3cm} | p{7.5cm} | p{7.5cm} |}
            \hline\hline
            \rowcolor{mintgreen}
            \multicolumn{1}{|>{\centering\arraybackslash}m{13mm}|}{\textbf{\huge\textcolor{castletongreen}{\texttt{\#}}}} & \multicolumn{1}{|>{\centering\arraybackslash}m{75mm}|}{\huge\textcolor{castletongreen}{\textbf{Attore}}} &\multicolumn{1}{|>{\centering\arraybackslash}m{75mm}|}{\huge\textcolor{castletongreen}{\textbf{Sistema}}} \\ \hline

            \centering\textbf{\textcolor{2}{1c.1a.1}} &  Raggruppa i turni consecutivi. & \cellcolor{RedPastel} I turni selezionati non sono consecutivi.\\\hline

            & \textit{Termina il Caso d'Uso} & \\\hline

            \hline
            \end{longtable}
          
    \end{center}
\end{flushleft}
\pagebreak
\section*{\huge\textbf{\textcolor{castletongreen}{Estensione 1d}}}

\begin{flushleft}
    \begin{center}

        \begin{longtable}{ | p{1.3cm} | p{7.5cm} | p{7.5cm} |}
            \hline\hline
            \rowcolor{mintgreen}
            \multicolumn{1}{|>{\centering\arraybackslash}m{13mm}|}{\textbf{\huge\textcolor{castletongreen}{\texttt{\#}}}} & \multicolumn{1}{|>{\centering\arraybackslash}m{75mm}|}{\huge\textcolor{castletongreen}{\textbf{Attore}}} &\multicolumn{1}{|>{\centering\arraybackslash}m{75mm}|}{\huge\textcolor{castletongreen}{\textbf{Sistema}}} \\ \hline

            \centering\textbf{1d.1} & Modifica tutti i turni di cucina in un determinato giorno, settimana. & Registra le modifiche di tutti i turni di cucina aventi determinate caratteristiche.\\\hline

            \hline
            \end{longtable}
          
    \end{center}
\end{flushleft}

\section*{\huge\textbf{\textcolor{castletongreen}{Estensione 1e}}}

\begin{flushleft}
    \begin{center}

        \begin{longtable}{ | p{1.3cm} | p{7.5cm} | p{7.5cm} |}
            \hline\hline
            \rowcolor{mintgreen}
            \multicolumn{1}{|>{\centering\arraybackslash}m{13mm}|}{\textbf{\huge\textcolor{castletongreen}{\texttt{\#}}}} & \multicolumn{1}{|>{\centering\arraybackslash}m{75mm}|}{\huge\textcolor{castletongreen}{\textbf{Attore}}} &\multicolumn{1}{|>{\centering\arraybackslash}m{75mm}|}{\huge\textcolor{castletongreen}{\textbf{Sistema}}} \\ \hline

            \centering\textbf{1e.1} & Elimina tutti i turni di cucina in un determinato giorno, settimana. & Registra l'eliminazione di tutti i turni di cucina aventi determinate caratteristiche.\\\hline

            \hline
            \end{longtable}
          
    \end{center}
\end{flushleft}

\section*{\huge\textbf{\textcolor{castletongreen}{Estensione 6a}}}

\begin{flushleft}
    \begin{center}

        \begin{longtable}{ | p{1.3cm} | p{7.5cm} | p{7.5cm} |}
            \hline\hline
            \rowcolor{mintgreen}
            \multicolumn{1}{|>{\centering\arraybackslash}m{13mm}|}{\textbf{\huge\textcolor{castletongreen}{\texttt{\#}}}} & \multicolumn{1}{|>{\centering\arraybackslash}m{75mm}|}{\huge\textcolor{castletongreen}{\textbf{Attore}}} &\multicolumn{1}{|>{\centering\arraybackslash}m{75mm}|}{\huge\textcolor{castletongreen}{\textbf{Sistema}}} \\ \hline

            \centering\textbf{6a.1} & Rimuove la scadenza. & Toglie la scadenza oltre la quale non si può più rimuovere la disponibilità.\\\hline

            \hline
            \end{longtable}
          
    \end{center}
\end{flushleft}

\section*{\huge\textbf{\textcolor{2}{Eccezione 6a}}}

\begin{flushleft}
    \begin{center}

        \begin{longtable}{ | p{1.3cm} | p{7.5cm} | p{7.5cm} |}
            \hline\hline
            \rowcolor{mintgreen}
            \multicolumn{1}{|>{\centering\arraybackslash}m{13mm}|}{\textbf{\huge\textcolor{castletongreen}{\texttt{\#}}}} & \multicolumn{1}{|>{\centering\arraybackslash}m{75mm}|}{\huge\textcolor{castletongreen}{\textbf{Attore}}} &\multicolumn{1}{|>{\centering\arraybackslash}m{75mm}|}{\huge\textcolor{castletongreen}{\textbf{Sistema}}} \\ \hline

            \centering\textbf{\textcolor{2}{6a.1}} &  Opzionalmente, modifica la scadenza. & \cellcolor{RedPastel} Il turno non ha una scadenza impostata.\\\hline

            & \textit{Torna al passo 6} & \\\hline

            \hline
            \end{longtable}
          
    \end{center}
\end{flushleft}

\section*{\huge\textbf{\textcolor{castletongreen}{Estensione 6b}}}

\begin{flushleft}
    \begin{center}

        \begin{longtable}{ | p{1.3cm} | p{7.5cm} | p{7.5cm} |}
            \hline\hline
            \rowcolor{mintgreen}
            \multicolumn{1}{|>{\centering\arraybackslash}m{13mm}|}{\textbf{\huge\textcolor{castletongreen}{\texttt{\#}}}} & \multicolumn{1}{|>{\centering\arraybackslash}m{75mm}|}{\huge\textcolor{castletongreen}{\textbf{Attore}}} &\multicolumn{1}{|>{\centering\arraybackslash}m{75mm}|}{\huge\textcolor{castletongreen}{\textbf{Sistema}}} \\ \hline

            \centering\textbf{6b.1} & Aggiunge una scadenza. & Aggiunge una scadenza oltre la quale non si può più rimuovere la disponibilità.\\\hline

            \hline
            \end{longtable}
          
    \end{center}
\end{flushleft}

\section*{\huge\textbf{\textcolor{castletongreen}{Estensione 7a}}}

\begin{flushleft}
    \begin{center}

        \begin{longtable}{ | p{1.3cm} | p{7.5cm} | p{7.5cm} |}
            \hline\hline
            \rowcolor{mintgreen}
            \multicolumn{1}{|>{\centering\arraybackslash}m{13mm}|}{\textbf{\huge\textcolor{castletongreen}{\texttt{\#}}}} & \multicolumn{1}{|>{\centering\arraybackslash}m{75mm}|}{\huge\textcolor{castletongreen}{\textbf{Attore}}} &\multicolumn{1}{|>{\centering\arraybackslash}m{75mm}|}{\huge\textcolor{castletongreen}{\textbf{Sistema}}} \\ \hline

            \centering\textbf{7a.1} & Cancella un turno di servizio, modificando tutte le ricorrenze. & Registra la cancellazione del turno di servizio e di tutte le sue ricorrenze.\\\hline

            \hline
            \end{longtable}
          
    \end{center}
\end{flushleft}

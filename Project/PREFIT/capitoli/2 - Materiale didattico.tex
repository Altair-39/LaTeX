\chapter{Sviluppo dei contenuti}

\section{Materiale didattico per fasi}

\clm{}{}{Bisogna inzialmente introdurre in linea generale il contenuto
dell'attività agli studenti, facendo attenzione a non perdersi nei dettagli
che verranno approfonditi durante le fasi dell'attività. Questo 
aiuta ad accendere la curiosità degli studenti e a farli partecipare
attivamente all'attività.
}

\nt{In questo documento ho inserito note specifiche per gli insegnanti
(come quella sopra). Tuttavia sono solo brevi suggerimenti, in quanto
la guida per gli insegnanti vera e propria è contenuta nella sezione
seguente. }

\paragraph{Fase 1:}

\begin{itemize}
    \item [$\Rightarrow$] \textcolor{cyan}{Consegna:} 
    \item [$\Rightarrow$] \textcolor{magenta}{Svolgimento:} 
    \item [$\Rightarrow$] \textcolor{teal}{Discussione:} 
    \item [$\Rightarrow$] \textcolor{orange}{Conclusione:} 
\end{itemize}

\paragraph{Fase 2:}

\begin{itemize}
    \item [$\Rightarrow$] \textcolor{cyan}{Consegna:} 
    \item [$\Rightarrow$] \textcolor{magenta}{Svolgimento:} 
    \item [$\Rightarrow$] \textcolor{teal}{Discussione:}
    \item [$\Rightarrow$] \textcolor{orange}{Conclusione:}
\end{itemize}

\clm{}{}{}

\section{Attività unplugged}\label{unplugged}

%% In poche parole: si scrive un numero alla lavagna. Si chiede agli studenti di sommare questo
%% numero con sè stesso meno uno e si ripete questo procedimento finchè non si passa zero.
%% Ogni studente è coinvolto in questo procedimento in ordine alfabetico. 
%% Ciò spiega il concetto di ricorsione di una funzione.

\section{Esercizi di programmazione}\label{prog}
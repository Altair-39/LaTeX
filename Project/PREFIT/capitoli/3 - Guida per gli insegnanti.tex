\chapter{Guida per gli insegnanti}

\section{Consigli sull'utilizzo del materiale didattico}

\section{Snodi e indicatori per fasi}

\subsection{Attività 1}

\paragraph{\evidence{Fase 1:}}

\begin{itemize}
    \item Snodi:
    \begin{itemize}
        \item [$\Rightarrow$] Capire la struttura di una funzione ricorsiva.
        \item [$\Rightarrow$] Analizzare un programma ricorsivo.
        \item [$\Rightarrow$] Comprensione della sintassi di Haskell.
    \end{itemize}
    \item Indicatori:
    \begin{itemize}
        \item [$\Rightarrow$] Quali sono le caratteristiche di una funzione ricorsiva?
        \item [$\Rightarrow$] Identificate il passo base e il passo ricorsivo nel programma
                              "Fibonacci ricorsivo".
        \item [$\Rightarrow$] Elencare i principali tipi di Haskell.
    \end{itemize}
\end{itemize}

\subsection{Attività 2}

\paragraph{\evidence{Fase 1:}}

\begin{itemize}
    \item Snodi:
    \begin{itemize}
        \item [$\Rightarrow$] Capire cosa simboleggia il risultato ottenuto eseguendo le istruzioni sulle matrioske.
        \item [$\Rightarrow$] Mettere in corrispondenza il concetto iterativo di "contatore" con il risultato
        ricorsivo ottenuto durante l'attività.
        \item [$\Rightarrow$] Comprensione dell'esistenza di soluzioni più "appropriate" per la risoluzione di problemi.
    \end{itemize}
    \item Indicatori:
    \begin{itemize}
        \item [$\Rightarrow$] Che risultato si è ottenuto? Cosa rappresenta?
        \item [$\Rightarrow$] Era possibile raggiungere un simile risultato in un altro modo? Se sì, quale?
        \item [$\Rightarrow$] Che cosa succede se, quando rimonto una matrioska, non ne inserisco una e riprovo a fare l'intera attività?
    \end{itemize}
\end{itemize}

\paragraph{\fancyglitter{Fase 2:}}

\begin{itemize}
    \item Snodi:
    \begin{itemize}
        \item [$\Rightarrow$] Simulare l'esecuzione di un programma ricorsivo e saperne predire l'output.
        \item [$\Rightarrow$] Comprendere l'esistenza di strutture dati che riferiscono sé stesse (Matrioska).
    \end{itemize}
    \item Indicatori:
    \begin{itemize}
        \item [$\Rightarrow$] Qual è l'output del programma se alla funzione \texttt{costruisciMatrioska} 
        passo 3 invece che 5? E se invece passo 7?
        \item [$\Rightarrow$] Cosa succede se passo 0 alla funzione \texttt{costruisciMatrioska}?
    \end{itemize}
\end{itemize}

\subsection{Attività 3}

\paragraph{\evidence{Fase 1:}}

\begin{itemize}
    \item Snodi:
    \begin{itemize}
        \item [$\Rightarrow$]
    \end{itemize}
    \item Indicatori:
    \begin{itemize}
        \item [$\Rightarrow$]
    \end{itemize}
\end{itemize}
\chapter{Introduzione}

\nt{Quest'attività, essendo pesante e impegnativa, richiede 2-3 lezioni per essere svolta
(volendo si può pensare a una quarta lezione per approfondire meglio).}

\section{Livello di scuola, classe e indirizzo}

\qs{}{A chi è rivolta questa attività?}

\sol{A studenti del quarto/quinto anno di una scuola secondaria di secondo grado
di un indirizzo \textbf{scientifico}\footnote{Liceo Scientifico Op. 
Scienze Applicate}.}

\qs{}{Può essere adattata/rivolta a studenti di diverse età e indirizzi?}

\sol{Quest'attività può essere somministrata a studenti al quinto anno di superiori
senza alcuna modifica. Può, altresì, essere eseguita da studenti di età 
inferiore con alcune correzioni (che indicherò nel documento).}

\section{Motivazioni e finalità}

\qs{}{Perché si è scelta proprio quest'attività?}

\sol{L'attività nasce dalla necessità di insegnare ai ragazzi
delle superiori il concetto di programma ricorsivo. Spesso quest'argomento 
viene trattato in modo superficiale o non viene trattato proprio, ma è utile avere 
un modello mentale di ricorsione per poter vedere i problemi sotto un'altra luce.
Infatti lo scopo di questa unità didattica non è solamente quello di insegnare 
un argomento, ma punta a mostrare soluzioni alternative ad alcuni problemi.
}

\begin{itemize}
    \item [$\Rightarrow$] \textbf{Scrivere codice in modo elegante:} quando si insegna
    a programmare, almeno all'inizio, viene trascurato il fatto che il codice debba essere
    letto e capito da altre persone per cui ci si focalizza più sull'aspetto "funziona" rispetto
    alla "veste grafica". Questo può anche portare a errori di programmazione dato che si avranno
    difficoltà a leggere anche i propri programmi. Il presente documento non si vuole semplicemente
    limitare a insegnare il concetto, molto utile, della ricorsione, ma vuole anche far fronte al
    problema dello "spaghetti code" (programmi mal strutturati) fornendo una valida alternativa.
    \item [$\Rightarrow$] \textbf{Pensiero computazionale e algoritmico:} una caratteristica fondamentale
    nella società odierna è quella di far fronte a problemi spesso imprevisti. Non c'è solo un modo per
    risolvere un quesito e le soluzioni possono essere molteplici. Infatti, per alcuni problemi, può
    risultare ostico usare un approccio iterativo per cui è preferibile utilizzare la ricorsione.
\end{itemize}

\section{Prerequisiti}

Verranno dati per scontati i seguenti prerequisiti in quanto fondamentali per la comprensione dell'unità didattica.

\begin{itemize}
    \item [$\Rightarrow$] Utilizzo di base del computer;
    \item [$\Rightarrow$] Concetto di algoritmo;
    \item [$\Rightarrow$] Concetto di variabile;
    \item [$\Rightarrow$] Tipi di variabili (Int, Float, etc.);
    \item [$\Rightarrow$] Logica booleana;
    \item [$\Rightarrow$] Propensione al ragionamento astratto.
\end{itemize}

\section{Contenuti}

I contenuti che presentano un asterisco blu (\textcolor{blue}{*}) sono
adatti anche a studenti di altri indirizzi o di età inferiore (quarta superiore). I contenuti che presentano un asterisco rosso (\textcolor{red}{*}) sono
opzionali per via della maggiore difficoltà, ma se gli alunni reagiscono bene al resto dell'attività
si potrebbero considerare come "bonus".

\begin{itemize}
    \item [$\Rightarrow$] \textcolor{blue}{*} Ricorsione;
    \item [$\Rightarrow$] \textcolor{blue}{*} Funzioni;
    \item [$\Rightarrow$] \textcolor{blue}{*} Tipi primitivi di Haskell;
    \item [$\Rightarrow$] Inferenza di tipo;
    \item [$\Rightarrow$] Funzioni a più argomenti;
    \item [$\Rightarrow$] Funzioni con guardie;
    \item [$\Rightarrow$] Liste;
    \item [$\Rightarrow$] Pattern matching;
    \item [$\Rightarrow$] \textcolor{red}{*} Funzioni anonime;
    \item [$\Rightarrow$] \textcolor{red}{*} Alberi.
\end{itemize}

\section{Traguardi e obiettivi}

\subsection{Obiettivi di apprendimento}

\nt{I seguenti obiettivi di apprendimento sono stati scritti usando la tassonomia
di Bloom rivisitata.}


\begin{center}
    \begin{tabular}{ || C{3cm} | >{\columncolor{RedPastel}}p{8.3cm} ||}
    \hline\hline
        \rowcolor{lightgray}
    \textbf{Tassonomia} & \textbf{Obiettivi}\\ \hline
    \rowcolor{RedPastel}
        \textbf{CREATE} & Lo/La studente/ssa, al termine dell'attività, 
        sarà in grado di \textbf{sviluppare} semplici programmi ricorsivi in Haskell. \\\hline

        \rowcolor{OrangePastel}
        \textbf{EVALUATE} & Lo/La studente/ssa, al termine dell'attività,
        saprà \textbf{valutare} la convenienza di un approccio ricorsivo rispetto a un 
        approccio iterativo.\\\hline

        \rowcolor{BrownPastel}
        \textbf{ANALYZE} &  Lo/La studente/ssa, al termine dell'attività,
        comprenderà i concetti di passo base e passo induttivo/ricorsivo e 
        saprà \textbf{simulare} l'esecuzione di programmi che usano esplicitamente
        la ricorsione.\\\hline

        \rowcolor{GreenPastel}
        \textbf{APPLY} & Lo/La studente/ssa, al termine dell'attività,
        riuscirà a \textbf{usare} in modo efficace la ricorsione per risolvere
        problemi. \\\hline

        \rowcolor{BluePastel}
        \textbf{UNDERSTAND} &  Lo/La studente/ssa, al termine dell'attività,
        saprà \textbf{identificare}, \textbf{classificare} e \textbf{descrivere} programmi ricorsivi. \\\hline

        \rowcolor{PurplePastel}
        \textbf{REMEMBER} &  Lo/La studentessa, al termine dell'attività,
        \textbf{ricorderà} le componenti fondamentali di un programma ricorsivo 
        (passo base e passo induttivo).\\\hline

    \hline
    \end{tabular}
\end{center}


\subsection{Indicazioni nazionali}

\paragraph{Ambito algoritmico:}  implementazione di un linguaggio di programmazione, metodologie di programmazione.

\paragraph{Ambito calcolo numerico e simulazioni:} semplici simulazioni.

\section{Materiali e strumenti necessari}

\paragraph{Materiale fisico:}

\begin{itemize}
    \item [$\Rightarrow$] Matrioske (bambole russe);
    \item [$\Rightarrow$] Lavagna multimediale;
    \item [$\Rightarrow$] Computers.
\end{itemize}

\paragraph{Materiale software:}

\begin{itemize}
    \item [$\Rightarrow$] GHCI (interprete per Haskell);
    \item [$\Rightarrow$] Editor di testo o IDE (Integrated Development Enviroment);
    \item [$\Rightarrow$] Terminale o Powershell.
\end{itemize}

\nt{GHCI può essere reperito al seguente link: \texttt{\href{https://www.haskell.org/downloads/}{https://www.haskell.org/downloads/}}.
Si raccomanda di utilizzarlo su un computer con Linux, anche se rimane comunque installabile su Windows.}

\section{Linguaggio}

\paragraph{Linguaggio scelto:} \evidence{Haskell}.

\qs{}{Perché si è scelto questo linguaggio?}

\sol{Haskell è un linguaggio funzionale puro, il che lo rende ideale
per insegnare agli studenti un concetto importante come la ricorsione.
Inoltre Haskell presenta una sintassi molto semplice e pulita, che
non distrae gli studenti dal concetto che si sta insegnando: i codici risultano
molto più compatti rispetto ad altri linguaggi (C, Java, ecc...).
Infine, Haskell è un linguaggio unico nel suo genere, in quanto
"lazy" e "strongly typed", il che lo rende un linguaggio molto interessante
da studiare e da approfondire.}
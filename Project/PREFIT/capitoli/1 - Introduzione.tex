\chapter{Introduzione}

\nt{Quest'attività, essendo pesante e impegnativa, richiede 2-3 lezioni per essere svolta
(volendo si può pensare a una quarta lezione per approfondire meglio).}

\section{Livello di scuola, classe e indirizzo}

\qs{}{A chi è rivolta questa attività?}

\sol{A studenti/studentesse del quarto/quinto anno di una scuola secondaria di secondo grado
di un indirizzo \textbf{scientifico}\footnote{Liceo Scientifico Op. 
Scienze Applicate}.}

\qs{}{Può essere adattata/rivolta a studenti/studentesse di diverse età e indirizzi?}

\sol{Quest'attività può essere somministrata a studenti/studentesse al quinto anno di superiori
senza alcuna modifica. Può, altresì, essere eseguita da studenti di età 
inferiore con alcune correzioni (che indicherò nel documento).}

\qs{}{In quale specifica disciplina scolastica si colloca l'attività?}

\sol{Quest'attività ha carattere principalmente informatico trattando un argomento
prevalentemente computazione (la ricorsione). Tuttavia si può inquadrare quest'unità didattica
in un contesto più ampio trattando anche Haskell che, come linguaggio puramente funzionale,
si può associare a un ambito matematico e logico. Il ragionamento ricorsivo ha anche un'utilità
trasversale dato che aiuta a comprendere in maniera più profonda alcuni problemi e la
loro scomposizione in sottoproblemi.}

\pagebreak

\section{Motivazioni e finalità}

\qs{}{Perché si è scelta proprio quest'attività?}

\sol{L'attività nasce dalla necessità di insegnare ai ragazzi
delle superiori il concetto di programma ricorsivo. Spesso quest'argomento 
viene trattato in modo superficiale o non viene trattato proprio, ma è utile avere 
un modello mentale di ricorsione per poter vedere i problemi sotto un'altra luce.
Infatti lo scopo di questa unità didattica non è solamente quello di insegnare 
un argomento, ma punta a mostrare soluzioni alternative ad alcuni problemi.
}

\begin{itemize}
    \item [$\Rightarrow$] \textbf{Scrivere codice in modo elegante:} quando si insegna
    a programmare, almeno all'inizio, viene trascurato il fatto che il codice debba essere
    letto e capito da altre persone per cui ci si focalizza più sull'aspetto "funziona" rispetto
    alla "veste grafica". Questo può anche portare a errori di programmazione dato che si avranno
    difficoltà a leggere anche i propri programmi. Il presente documento non si vuole semplicemente
    limitare a insegnare il concetto, molto utile, della ricorsione, ma vuole anche far fronte al
    problema dello "spaghetti code" (programmi mal strutturati) fornendo una valida alternativa.
    \item [$\Rightarrow$] \textbf{Pensiero computazionale e algoritmico:} una caratteristica fondamentale
    nella società odierna è quella di far fronte a problemi spesso imprevisti. Non c'è solo un modo per
    risolvere un quesito e le soluzioni possono essere molteplici. Infatti, per alcuni problemi, può
    risultare ostico usare un approccio iterativo per cui è preferibile utilizzare la ricorsione.
\end{itemize}

\section{Prerequisiti}

Verranno dati per scontati i seguenti prerequisiti in quanto fondamentali per la comprensione dell'unità didattica.

\begin{itemize}
    \item [$\Rightarrow$] Utilizzo di base del computer;
    \item [$\Rightarrow$] Concetto di algoritmo;
    \item [$\Rightarrow$] Concetto di variabile;
    \item [$\Rightarrow$] Tipi di variabili (Int, Float, etc.);
    \item [$\Rightarrow$] Logica booleana;
    \item [$\Rightarrow$] Propensione al ragionamento astratto.
\end{itemize}
\paragraph{}

\nt{È propedeutica la conoscenza di Haskell che verrà comunque ripreso, sviluppato e approfondito nel corso dell'unità didattica.}

\section{Contenuti}

I contenuti che presentano un asterisco blu (\textcolor{blue}{*}) sono
adatti anche a studenti di altri indirizzi o di età inferiore (quarta superiore). I contenuti che presentano un asterisco rosso (\textcolor{red}{*}) sono
opzionali per via della maggiore difficoltà, ma se gli alunni reagiscono bene al resto dell'attività
si potrebbero considerare come "bonus".

\begin{itemize}
    \item [$\Rightarrow$] \textcolor{blue}{*} \textbf{Ricorsione:} il cuore di quest'attività è insegnare la ricorsione e il
    ragionamento ricorsivo, aiutando a riconoscere quando esso sia preferibile a un ragionamento iterativo;
    \item [$\Rightarrow$] \textcolor{blue}{*} \textbf{Funzioni:} ci si occuperà sia di funzioni in generale (principalmente per fare pratica con Haskell)
    che di funzioni ricorsive;
    \item [$\Rightarrow$] \textcolor{blue}{*} \textbf{Tipi di Haskell:} Num, Int, Float, String;
    \item [$\Rightarrow$] \textbf{Inferenza di tipo:} mostrare come in Haskell si deve essere precisi con i tipi e di come ciò sia un aiuto alla programmazione;
    \item [$\Rightarrow$] \textbf{Funzioni a più argomenti:} come si possono usare funzioni che richiedono più di un parametro in Haskell (curryficate);
    \item [$\Rightarrow$] \textbf{Funzioni con guardie:} una rappresentazione per casi alternativa all'utilizzo di "\texttt{IF}" che risulta molto utile
    anche nella definizione di funzioni ricorsive;
    \item [$\Rightarrow$] \textbf{Liste:} usate come "scusa" per introdurre il pattern matching;
    \item [$\Rightarrow$] \textbf{Pattern matching:} come il pattern matching delle liste sia utile per definire funzioni ricorsive;
    \item [$\Rightarrow$] \textcolor{red}{*} \textbf{Funzioni anonime:} funzioni che usano lambda expression o funzioni parzialmente applicate;
    \item [$\Rightarrow$] \textcolor{red}{*} \textbf{Alberi:} visione di un albero con l'utilizzo della ricorsione.
\end{itemize}

\section{Traguardi e obiettivi}

\subsection{Obiettivi di apprendimento}

\nt{I seguenti obiettivi di apprendimento sono stati scritti usando la tassonomia
di Bloom rivisitata.}


\begin{center}
    \begin{tabular}{ || C{3cm} | >{\columncolor{RedPastel}}p{8.3cm} ||}
    \hline\hline
        \rowcolor{lightgray}
    \textbf{Tassonomia} & \textbf{Obiettivi}\\ \hline
    \rowcolor{RedPastel}
        \textbf{CREATE} & Lo/La studente/ssa, al termine dell'attività, 
        sarà in grado di \textbf{sviluppare} semplici programmi ricorsivi in Haskell. \\\hline

        \rowcolor{OrangePastel}
        \textbf{EVALUATE} & Lo/La studente/ssa, al termine dell'attività,
        saprà \textbf{valutare} la convenienza di un approccio ricorsivo rispetto a un 
        approccio iterativo.\\\hline

        \rowcolor{BrownPastel}
        \textbf{ANALYZE} &  Lo/La studente/ssa, al termine dell'attività,
        comprenderà i concetti di passo base e passo induttivo/ricorsivo e 
        saprà \textbf{simulare} l'esecuzione di programmi che usano esplicitamente
        la ricorsione.\\\hline

        \rowcolor{GreenPastel}
        \textbf{APPLY} & Lo/La studente/ssa, al termine dell'attività,
        riuscirà a \textbf{usare} in modo efficace la ricorsione per risolvere
        problemi. \\\hline

        \rowcolor{BluePastel}
        \textbf{UNDERSTAND} &  Lo/La studente/ssa, al termine dell'attività,
        saprà \textbf{identificare}, \textbf{classificare} e \textbf{descrivere} programmi ricorsivi. \\\hline

        \rowcolor{PurplePastel}
        \textbf{REMEMBER} &  Lo/La studentessa, al termine dell'attività,
        \textbf{ricorderà} le componenti fondamentali di un programma ricorsivo 
        (passo base e passo induttivo).\\\hline

    \hline
    \end{tabular}
\end{center}


\subsection{Indicazioni nazionali}

\paragraph{Traguardi:}

\begin{itemize}
    \item Comprendere i principali
    fondamenti teorici delle scienze dell’informazione;
    \item Acquisire la padronanza di strumenti
    dell’informatica, utilizzare tali strumenti per la soluzione di problemi significativi in generale;
    \item L'uso di strumenti e la creazione di applicazioni deve essere accompagnata non solo da una
    conoscenza adeguata delle funzioni e della sintassi, ma da un sistematico collegamento con i
    concetti teorici ad essi sottostanti;
    \item l rapporto fra teoria e pratica va mantenuto
    su di un piano paritario e i due aspetti vanno strettamente integrati evitando sviluppi paralleli
    incompatibili con i limiti del tempo a disposizione;
\end{itemize}

\paragraph{Obiettivi specifici:}

\begin{itemize}
    \item [$\Rightarrow$] \textbf{Ambito algoritmico:}  implementazione di un linguaggio di programmazione, metodologie di programmazione;
    \item [$\Rightarrow$] \textbf{Ambito calcolo numerico e simulazioni:} semplici simulazioni.
\end{itemize}

\paragraph{Fonte:} \href{https://www.istruzione.it/alternanza/allegati/NORMATIVA%20ASL/INDICAZIONI%20NAZIONALI%20PER%20I%20LICEI.pdf}{\texttt{INDICAZIONI NAZIONALI}}

\section{Materiali e strumenti necessari}

\paragraph{Materiale fisico:}

\begin{itemize}
    \item [$\Rightarrow$] Matrioske (bambole russe);
    \item [$\Rightarrow$] Adesivi (in alternativa e si possono utilizzare dei pezzi di carta e del nastro adesivo);
    \item [$\Rightarrow$] Fogli di carta;
    \item [$\Rightarrow$] Penne;
    \item [$\Rightarrow$] Lavagna multimediale;
    \item [$\Rightarrow$] Computers.
\end{itemize}

\paragraph{Materiale software:}

\begin{itemize}
    \item [$\Rightarrow$] GHCI (interprete per Haskell);
    \item [$\Rightarrow$] Editor di testo o IDE (Integrated Development Enviroment);
    \item [$\Rightarrow$] Terminale o Powershell.
\end{itemize}

\nt{GHCI può essere reperito al seguente link: \texttt{\href{https://www.haskell.org/downloads/}{https://www.haskell.org/downloads/}}.
Si raccomanda di utilizzarlo su un computer con Linux, anche se rimane comunque installabile su Windows.
Come Editor/IDE si possono usare:
\begin{itemize}
    \item [$\Rightarrow$] Notepad;
    \item [$\Rightarrow$] VsCode: \href{https://code.visualstudio.com/download}{\texttt{https://code.visualstudio.com/download}} o su linux
    si può utilizare il proprio package manager (apt, pacman, yay, etc.). Es. \texttt{sudo apt install code};
    \item [$\Rightarrow$] IntelliJ: \href{https://www.jetbrains.com/toolbox-app/}{https://www.jetbrains.com/toolbox-app/}. Dopo aver installato il toolbox
    si può installare la versione community di IntelliJ (più che sufficiente).
\end{itemize}
}

\section{Linguaggio}

\paragraph{Linguaggio scelto:} \evidence{Haskell}.

\qs{}{Perché si è scelto questo linguaggio?}

\sol{Haskell è un linguaggio funzionale puro, il che lo rende ideale
per insegnare agli studenti un concetto importante come la ricorsione.
Inoltre Haskell presenta una sintassi molto semplice e pulita, che
non distrae gli studenti dal concetto che si sta insegnando: i codici risultano
molto più compatti rispetto ad altri linguaggi (C, Java, ecc...).
Infine, Haskell è un linguaggio unico nel suo genere, in quanto
"lazy" e "strongly typed", il che lo rende un linguaggio molto interessante
da studiare e da approfondire.}
\chapter{Laboratorio}

\section{Lezione 1}

\subsection{La piattaforma IntelliJ}

\dfn{IDE}{
Quando si sviluppa un software (SW) di grandi dimensioni è necessario utilizzare un \textbf{IDE}\footnote{Integrated development environment}. 
}

\nt{L'IDE offre supporto alla compilazione e all'esecuzione dei programmi, a caricarli sul web, etc.}

\dfn{IntelliJ}{\textbf{IntelliJ} offre un editor per lo sviluppo di applicazioni web e standalone.

Le versioni principali sono due:

\begin{itemize}
    \item Community: non permette lo sviluppo web;
    \item ULTIMATE: è la versione completa ed è gratuita per studenti universitari.
\end{itemize}
}
\cor{}{
IntelliJ organizza tutte le applicazioni in progetti (\textbf{Project}), ognuno dei quali include:
\begin{itemize}
    \item \textit{Source Package (src)}: il codice sorgente, ossia le classi java;
    \item \textit{External library}: le librerie utilizzate;
    \item altre cartelle.
\end{itemize}
}

\subsection{Installare IntelliJ}

\begin{enumerate}
    \item Come prerequisito bisogna aver installato almeno la \underline{versione 13} di \href{https://www.oracle.com/it/java/technologies/downloads/\#jdk21-windows}{JDK} (meglio se 20 o successiva);
    \item Installare \textbf{JetBrains} Toolbox da questo link: \href{https://www.jetbrains.com/toolbox-app/}{https://www.jetbrains.com/toolbox-app/};
    \item Avviare JetBrains Toolbox;
    \item Cercare e installare \textbf{IntelliJ IDEA ultimate};
    \item Avviare IntelliJ IDEA ultimate;
    \item Cliccare sui tre puntini in basso a sinistra e selezionare Manage Licenses;
    \item Acquisire la licenza di IntelliJ IDEA\footnote{Nota: la licenza è fornita gratuitamente agli studenti universitari, ma deve essere rinnovata ogni anno}.
\end{enumerate}

\nt{Per verificare la propria JDK basta eseguire il comando "java --version" da terminale.}

\subsection{Estensioni utili}

Breve elenco di plugins che possono migliorare la \textbf{quality of life} (QOL).

\begin{itemize}
    \item Atom Material Icons: un set di icone che rende più "vivace" l'ambiente di sviluppo favorendo visivamente il riconoscimento di file e cartelle;
    \item CodeGlance Pro: mostra una "\textbf{mappa}" del proprio codice a destra dello schermo, permettendo una rapida visione d'insieme e la possibilità di spostarsi precisamente usando l'interfaccia grafica;
    \item Conventional Commit: fornisce un completamento per commit "standard" su git;
    \item Key Promoter X: serve per imparare le combinazioni di tasti (\textbf{shortcuts}). Ogni volta che si utilizza il menu testuale viene mostrata l'alternativa con la tastiera insieme a un contatore che segna quanti "miss" di quella shortcut sono stati fatti;
    \item PDF Viewer: permette di visualizzare i file PDF all'interno dell'IDE;
    \item Rainbow CSV: migliora la lettura dei file CSV colorando i vari campi;
    \item Rainbow Brackets: migliora la leggibilità del codice colorando le parentesi.
\end{itemize}

\ex{Per iniziare}{

Per creare un progetto bisogna aprire il menu File $\rightarrow$ New $\rightarrow$ Project. Nel menu che compare si seleziona New Project, con language Java, si definisce il nome del progetto e si clicca su create.

Il progetto nasce con la cartella \textbf{src}. Si possono creare classi, package, etc. facendo clic con il tasto destro all'interno della cartella che si vuole usare.

IntelliJ possiede anche utili funzioni che segnalano gli errori e aiutano con la compilazione. Riguardo all'autocompilazione: può essere usata per creare costruttori, getter, setter, toString, etc.

Quando si compila il programma viene creata la cartella \textbf{out} che contiene i file \textit{.class} del progetto. Essa viene distrutta e ricreata ogni volta che si compila il progetto in modo da eliminare eventuali problemi di dipendenze.
}
\nt{IntelliJ crea e gestisce i progetti in una cartella di default "IdeaProjects".}

\nt{Per convenzione, in un progetto Java, le classi che rappresentano oggetti delle applicazioni vanno inseriti in una cartella \textbf{model}, le classi che gestiscono le operazioni di input/output vanno inseriti in una cartella \textbf{io}.}

\section{Lezione 2}

\section{Lezione 3}

\section{Lezione 4}
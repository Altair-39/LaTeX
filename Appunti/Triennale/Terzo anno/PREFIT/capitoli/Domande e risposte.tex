\chapter{Domande e risposte}

\section{La natura dell'informatica}

\qs{}{Quali sono i problemi dell'informatica?}

\sol{il termine "informatica" è spesso utilizzato in modo improprio nel linguaggio comune. 
per esempio, viene usato per riferirsi a computers, cellulari, televisori, etc. Ma l'informatica non rigurda
i computers che sono solo uno strumento, non il fine. Un altro problema è il termine "digitale" che, solitamente, 
viene usato come sinonimo di "informatica". In realtà, digitale si riferisce alla rappresentazione di un dato 
mediante un simbolo numerico e a tutte le tecnologie basate sui computers.}

\qs{}{Quali sono i problemi relativi all'insegnamento dell'informatica nel nostro sistema scolastico?}

\sol{l'informatica viene spesso trattata come una non-materia, per cui 
chiunque sappia usare un banale applicativo come Word o Excel tende a comportarsi
come un esperto (si è vista in precedenza la distinzione tra informatico e digitale).
Inoltre un problema endemico della scuola italiana riguarda la carenza di informatici
che insegnino la materia, infatti spesso si ricorre a docenti di matematica (
    la cui classe di concorso li rende "idonei" a insegnare informatica).
}

\qs{}{Perchè è importante insegnare informatica fin dalla scuola primaria?}

\sol{Come le scienze naturali (di cui fa parte) l'informatica offre una chiave di
lettura del mondo che ci circonda, per cui è opportuno iniziare a studiarla il prima possibile
e che ogni studente ne abbia una conoscenza di base. Inoltre, l'informatica è una materia
che si presta molto bene a essere insegnata in modo interdisciplinare.}

\qs{}{Perchè è importante riflettere sulla natura dell'informatica prima di insegnarla?}

\sol{perchè ciò andrà a impattare sul modo di insegnare la materia. A seconda della propria visione del
mondo si privilegeranno alcuni aspetti rispetto ad altri. Inoltre bisogna sempre chiedersi il perchè si studia informatica:
essa dà una chiave di lettura del mondo digitale che è sempre più presente nella nostra quotidianità.}
\pagebreak
\qs{}{Quali sono le 3 anime/paradigmi che abbiamo discusso per inquadrare la natura dell'informatica?}

\sol{

\begin{itemize}
    \item matematico: l'informatica vista come una scienza matematica che formalizza i problemi e li risolve mediante algoritmi;
    \item ingegneristico: l'informatica vista come un'ingegneria che si occupa di progettare e realizzare sistemi software; 
    \item scientifico: l'informatica vista come una scienza che studia i sistemi software e i processi di sviluppo empiricamente.
\end{itemize}
}
\section{Teorie dell'apprendimento}

\qs{}{Che cos'è il comportamentismo?}

\sol{il comportamentismo è una teoria dell'apprendimento che si basa sull'osservazione del comportamento. Essa prevede
di modellare un comportamento desiderabile. La valutazione si basa sui cambiamenti nei comportamenti degli alunni visti come "tabule rase".
Spesso erano previsti "rinforzi" ossia punizioni corporali. Questo approccio è istruttivista.}

\qs{}{Che cos'è il cognitivismo?}

\sol{il cognitivismo rappresenta un superamento del comportamentismo. Il cognitivismo è una teoria dell'apprendimento che si basa
su alcune idee principali:
\begin{itemize}
    \item il carico cognitivo: il carico di lavoro mentale che un individuo deve sostenere per svolgere un compito;
    \item gli schemi e i modelli mentali.
\end{itemize}

L'approccio cognitivista punta a far ricordare e applicare la conoscenza.
}
\qs{}{Che cos'è il costruttivismo? (socio-costruttivismo e costruttivismo cognitivo)}

\sol{ il costruttivismo (ideato da J. Piaget) è una teoria dell'apprendimento che si basa sullo scetticismo:
\begin{itemize}
    \item la conoscenza è frutto dell'esperienza;
    \item non c'è modo di sapere la "vera" verità.
\end{itemize}

\paragraph{Si ricorre alla "viabilità" per valutare la conoscenza:} un'idea è valida se ha funzionato fino a quel momento. Per le azioni fisiche è viabile tutto ciò che
porta a un risultato. Sul piano concettuale ci si basa sulla non contradditorietà.

\begin{itemize}
    \item [$\Rightarrow$] Socio-costruttivismo: l'apprendimento è un processo sociale che avviene in un contesto sociale. Si crea insieme nuova conoscenza;
    \item [$\Rightarrow$] Costruttivismo cognitivo: l'apprendimento è il processo di costruzione del significato. L'insegnante deve solo facilitare la scoperta
    offrendo le risorse necessarie.
\end{itemize}
}
\qs{}{Che cos'è il costruzionismo?}

\sol{il costruzionismo (ideato da S. Papert) si basa sull'idea costruttivista di conoscenza. Ma a ciò viene aggiunta l'idea che la conoscenza deve essere
finalizzata alla costruzione di artefatti.}

\qs{}{Quali sono i punti fondamentali del brano di Papert "Gli ingranaggi della mia infanzia"?}

\sol{}

\begin{itemize}
    \item è più facile apprendere qualcosa se lo si assimila a modelli
    già posseduti (per Papert gli ingranaggi e il differenziale);
    \item al contrario l'apprendimento è più ostico se non si ha un modello
    di riferimento;
    \item oltre agli aspetti cognitivi, l'apprendimento è influenzato da
    aspetti emotivi;
    \item gli ingranaggi sono un qualcosa di personale per Papert, non
    funzionerebbero come modello per tutti.
\end{itemize}

\qs{}{Cos'è l'assimilazione?}

\sol{l'assimilazione è l'incorporazione di un determinato concetto in uno schema che è stato già acquisito.}

\qs{}{Cos'è l'accomodamento?}

\sol{l'accomodamento è la modifica di una struttura cognitiva in relazione al contatto con nuove informazioni.}

\qs{}{Cos'è la zona di sviluppo prossimo (ZSP)?}

\sol{la zona di sviluppo prossimo è la seconda delle tre aree di apprendimento di un bambino. Nella ZSP il bambino è in grado di apprendere solo con il supporto del docente ed è in quest'area che l'insegnante può intervenire.}

\qs{}{Quali sono le caratteristiche principali dell'apprendimento attivo?}

\sol{l'apprendimento attivo nasce dalla didattica costruttivista.}

\begin{itemize}
    \item l'apprendimento non avvene attraverso fasi standard;
    \item ogni studente ha la possibilità di stabilire il proprio percorso;
    \item l'insegnante è un facilitatore;
    \item le parole e le azioni del docente sono strumenti per apprendere;
    \item si dà la priorità all'esperienza diretta (gestita dall'insegnente).
\end{itemize}

\section{Problemi, compiti, spazio di rappresentazione del problema}

\qs{}{Descrvere la differenza tra compito, problema ben/mal strutturato, formulazione narrativa e algoritmica di un problema.}

\sol{}

\begin{itemize}
    \item compito: è un'entità sconosciuta in qualche situazione, per cui è necessario trovare una soluzione.
    Può essere presentato sotto forma di "compito di realtà", ossia basato su 
    eventi verosimili;  
    \item problema ben strutturato: è un problema che presenta tutti gli elementi necessari
    alla risoluzione, richiede l'applicazione di regole e princìpi organizzati in modo 
    predittivo e prescrittivo, ha una soluzione conoscibile e comprensibile;
    \item problema mal strutturato: è un problema che presenta elementi sconosciuti,
    ha più soluzioni o nessuna soluzione, ha più criteri di valutazione, richiede di esprimere
    giudizi e valutazioni;
    \item formulazione narrativa: è una descrizione del problema in linguaggio naturale
    che racconta qualcosa;
    \item formulazione algoritmica: è una descrizione del problema in linguaggio formale
    riferendosi direttamente alle strutture dati e alle variabili richieste.
\end{itemize}

\qs{}{Spiegare il ruolo e l'importanza della rappresentazione e manipolazione dello spazio del problema come parte delle strategie di problem solving.}

\sol{È importante che i problemi siano simulazioni di problemi quotidiani e 
professionali, in cui l'insegnante abbia scelto che componenti includere e come
rappresentarle. Se il problema è ben strutturato, la rappresentazione dello spazio
è semplice per via del modo prevedibile in cui si comportano le variabili,
quindi è, generalmente, più semplice. Viceversa con un problema mal strutturato
la rappresentazione dello spazio è più complessa e richiede più tempo.}

\section{Struttura dell'esame}

\begin{enumerate}
    \item \textbf{Discussione dell'attività didattica (40\% del voto):} i membri del gruppo
    presentano il lavoro svolto e rispondono a eventuali domande;
    \item \textbf{Discussione delle consegne (30\% del voto):} si devono discutere alcune delle consegne svolte.
    In questa parte si può guardare ciò che si è scritto;
    \item \textbf{Domande (30\% del voto):} vengono poste domande sul corso\footnote{Presenti in questa sezione.}. 
\end{enumerate}

\nt{Se si svolge in gruppi ricordare:
\begin{itemize}
    \item di parlare in maniera omogenea, si raccomanda di prepararsi ognuno 
    una propria parte e di non improvvisare;
    \item che ogni membro deve comunque sapere tutto dell'attività;
    \item che la discussione sulle consegne è individuale.
\end{itemize}
}
\chapter{Storia dei sistemi operativi}

\section{Prima dei sistemi operativi}

Il concetto di programma negli anni '40 era appenna accennato e molto
diverso da quello attuale. Per esempio, l'ENIAC, il primo computer
elettronico, era programmato tramite cablaggi elettrici. Il concetto di
programma memorizzato in memoria centrale è stato introdotto da John
von Neumann nel 1945 (con l?EDVAC e le istruzioni macchina).
Ma questa era una procedura manuale, a quei tempi non si pensava minimamente
al farsi aiutare da un programma per inserire un programma nel computer.

\dfn{Sistema operativo}{
    Un sistema operativo è un programma che aiuta a "far girare" gli altri
    programmi. 
}

Le cose iniziaro a cambiare dal 1955. Con il tempo i sistemi operativi sono diventati sempre
più complessi e sofisticati. Uno dei compiti storici degli S. O. era quello di rendere 
semplice agli utenti l'utilizzo di un computer.

Storicamente molti dei concetti ancora usati nei moderni S. O. (paginazione della memoria,
memoria virtuale, etc.) nascono dalla scarsità di risorse dei primi computer.

\section{Fase 1 : Job by Job / Open Shop}
\begin{itemize}
    \item I computer erano usati solo dai loro progettisti e dai loro collaboratori;
    \item Scrivere un programma veniva svolto in più passi:
    \begin{enumerate}
        \item Scrivere il programma su carta;
        \item Trasferire il programma su schede perforate;
        \item Nello slot prenotato il programmatore si recava nella sala del computer;
        \item Inseriva le schede perforate nel computer (se il computer non era guasto);
        \item Faceva partire il programma;
        \item Potevano essere lette le celle di memoria per seguire l'esecuzione del programma;
        \item L'output poteva essere stampato o convertito in schede perforate
    \end{enumerate}
    \item Se un programma andava storto era difficile il debugging;
    \item Si introduce la figura professionale dell'operatore addetto alla sala macchine
    che stampava la fotografia della memoria in caso di comportamenti anomali;
    \item Si caricava un solo programma alla volta;
    \item A metà degli anni '50 nascono i primi assembler.
\end{itemize}

\section{Fase 2 : Sistemi batch}

\begin{itemize}
    \item Bisogna permettere al computer di pianificare 
    il lavoro da svolgere in modo da sfruttare al meglio le risorse;
    \item \textbf{Resident monitor}: programma che rimane in memoria 
    e gestisce l'input e l'output. Si occupa anche di far partire i programmi;
    \item Per esempio se si voleva eseguire un programma in FORTRAN:
    \begin{enumerate}
        \item \$FORTRAN: carica il compilatore e compila il sorgente;
        \item \$LOAD: carica il programma utente;
        \item \$RUN: esegue il programma utente fino a \$END.
    \end{enumerate}
    \item Rispetto alla fase 1 emerge la distinzione tra programma e
    dati, ma non c'era nessuna protezione della memoria;
    \item \textbf{Batch processing}: il lavoro del Resident monitor per 
    eseguire tutti i programmi di un certo batch;
    \item Se erano presenti nastri magnetici si poteva parallelizzare
    parte del lavoro;
    \item La produttività aumentava a scapito dei tempi del singolo
    utente
\end{itemize}

\nt{Un esempio di sistema Batch è il BKS per il Philco 2000 che
aveva più lettori di nastro che funzionavano come diverse aree di
un file system.}

Però per i processori di quegli anni si raccomandava di scrivere programmi che:
\begin{itemize}
    \item Non contenesserò istruzioni di halt;
    \item Non riavvolgesserò i nastri;
    \item Non indirizzasserò indirizzi di memoria riservati al sistema.
\end{itemize}

\section{Fase 3 : Multiprogrammazione}

\begin{itemize}
    \item Si introdusserò dispositivi di memoria di massa ad accesso
    diretto;
    \item Si diffuserò transistor più potenti che implementasserò
    un sistema di interrupt;
    \item Così si poteva simulare un'esecuzione contemporanea di più programmi;
    \item Il termine Multiprogrammazione fu coniato nel 1959 da Christopher Strachey;
    \item \textbf{Spooling}: Simultaneous Peripheral Operation On Line, 
    permetteva di eseguire più programmi in parallelo asincronicamente;
    \item \textbf{Multitasking cooperativo}: un programma può girare perchè un
    altro programma lascia \textit{volontariamente} la CPU.
\end{itemize}

\dfn{Il sistema Atlas}{
    Il sistema Atlas era un sistema operativo che permetteva di eseguire
    più programmi in parallelo. Fu sviluppato da Tom Kilburn, Bruce
    Payne e David Howarth. Nell'Atlas compaiono per la prima volta
    le system call (o extracode) e l'implementazione di memoria virtuale 
    con paginazione su richiesta.
    \begin{itemize}
        \item La memoria dell'Atlas era suddivisa in pagine (blocks);
        \item Fixed store: una memoria a nuclep magnetico (MCM) da
        8192 words per il codice del supervisor;
        \item Subsidiary store: una MCM da 1024 words per i dati del
        supervisor;
        \item Core store\footnote{Una memoria cache}: una MCM da 16384 words per il processo utente;
        \item Drum store\footnote{Una RAM}: memoria a tamburo rotante per i processi non in
        esecuzione.
    \end{itemize}

    Il sistema Atlas prevedeva anche un system input tape, ossia una sorta
    di memoria di Swap. Il supervisor si assicurava di avere sempre
    un programma in esecuzione, se non ce n'era ne caricava uno dalla   
    memoria di Swap.
    
    }

\dfn{Il sistema B5000}{
    Il sistema B5000 era un computer della Burroughs Corporation    
    in grado di gestire uno stack. Nel Master Control Program (MCP)
    compare il "trashing". Inoltre MCP era scritto in un linguaggio
    ad alto livello.
    Nel MCV compare per la prima volta un timer hardware.
}

\section{Fase 4 : Timesharing}

\begin{itemize}
    \item Nel 1959 John McCarthy conia il termine \textbf{Timesharing};
    \item Nel 1961 McCarthy specifica che il Timesharing deve permettere
    a più utenti di usare la stessa macchina contemporaneamente;
    \item Il primo sistema Timesharing fu il CTSS (Compatible Time Sharing System)
    sviluppato da Fernando Corbato al MIT;
    \item Il suo obiettivo era quello di permettere a più utenti di usare
    la stessa macchina contemporaneamente attraverso un supervisor;
    \item Il CTSS doveva:
    \begin{enumerate}
        \item Tenere traccia delle aree di memoria occupate da ogni programma;
        \item Poter spostare i programmi da una memoria all'altra;
        \item Il programma in esecuzione deve poter essere interrotto dopo un certo lasso di tempo;
        \item Le operazioni di I/O devono essere asincrone;
        \item Se le operazioni I/O erano lente il programma doveva essere spostato in memoria secondaria;
        \item Gli utenti dovevano poter interrompere un programma non funzionante;
        \item Gli utenti dovevano poter interrogare il sistema.
    \end{enumerate}
    \item Il CTSS era dotato di accesso controllato dagli utenti attraverso
    login e password;
    \item Si potevano sviluppare nuovi comandi;
    \item Nel 1965  fu scritto un programma MAIL per automatizzare lo scambio di MAIL
    e nel 1971 fu inviato il primo SPAM;
    \item Il CTSS fu l'antecedente del MULTICS diretto da Robert Fano;
    \item Il MULTICS fu un fallimento commerciale ma un successo tecnologico
    dato che porto involontariamente alla nascita di UNIX;
    \item Nel 1965 Robert Daley e Peter Neumann svilupparono, in un 
    articolo, il concetto di file system gerarchico;
    \item Il file system doveva essere uno standard per cui tutti i suoi 
    processi dovevano essere invisibili all'utente;
    \item Nasce anche il termine "pathname" e il termine "link";
    \item Gli utenti hanno a disposizione comandi per manipolare i file;
    \item Il sistema Titan venne dotato di una file allocation table (FAT);
    \item Il sistema Titan aveva anche un sistema di protezione dei file in maniera criptata;
    \item Un altro sistema operativo diffuso in quegli anni era OS/360,
    il SO di IBM;
    \item OS/360 serviva a facilitare il passaggio a hardware più potenti.
\end{itemize}

\dfn{UNIX}{
    Inizialmente il nome UNIX era UNICS, ossia UNiplexed Information
    and Computing Service. Il nome fu cambiato in UNIX perchè UNICS
    era già stato registrato. Il nome UNIX è un gioco di parole
    con MULTICS.
    
    A metà degli anni '80 UNIX era il sistema operativo Timesharing 
    di riferimento. Lo UNIX ebbe la fortuna di comparire al momento
    giusto, quando i computer erano abbastanza potenti da poter
    supportare un sistema operativo Timesharing. Dennis Ritchie notò
    che il successo dello UNIX dipese anche dalla capacità di migliorare
    idee già utilizzate.

}

\nt{Nel MULTICS nascono directory e subdirectory, ma la soluzione
proposta aveva molti difetti risolti da UNIX.}

\section{Fase 5 : Sistemi concorrenti}

\begin{itemize}
    \item I sistemi operativi soffrivano spesso di deadlock, starvation e trashing;
    \item Si riscoprirono i principi della programmazione concorrente (semafori, regioni critiche e monitor);
    \item Nel 1969 Dijkstra sviluppa THE\footnote{Technische Hogeschool Eindhoven} operating system;
    \item Il THE era un sistema a strati sovrapposti (una serie di VM) e usava i semafori;
    \item Nel 1969 fu sviluppato anche il sistema operativo RC4000 (da Per Brinch Hansen);
    \item RC4000 era il primo a implementare un kernel (nucleos);
\end{itemize}

\dfn{Concurrent Pascal}{
    Hansen fu anche il padre del Concurrent Pascal (1973), un linguaggio
    contenente primitive di sincronizzazione e comunicazione tra processi.
    Fu usato per sviluppare il S.O. Solo.
}

\section{Fase 6 : Sistemi per PC}

\begin{itemize}
    \item Il capostipite degli S. O. per PC può essere considerato il sistema
    OS 6 sviluppato tra il 1969 e il 1972 da Stoy e Strachey;
    \item OS 6 era un S.O. scritto in BCPL, i cui concetti furono ripresi
    nello Xerox Alto;
    \item All'Alto seguono Pilot, Cedar e Xerox Star (dotato di Mouse e interfaccia grafica a finestre);
    \item Purtroppo la Xerox non seppe convincere il mercato.
\end{itemize}

\subsection{CP/M, MS-DOS e Windows}

All'inizio degli anni '60, Gary Kildall sviluppa il CP/M (Control Program for Microcomputers)
per controllare un floppy disk da un processore Intel 8080.
Nel 1974 fonda, insieme alla moglie, la Digital Research Inc. per commercializzare il CP/M 
(pubblicizato su riviste di elettronica e informatica). Il CP/M aveva la
caratteristica di poter girare su computer diversi. Il CP/M era diviso in:

\begin{itemize}
    \item CCP (Console Command Processor): interprete dei comandi;
    \item BDOS (Basic Disk Operating System): interfaccia tra CCP e hardware;
    \item BIOS;
\end{itemize}

Nel 1975, Kildall ideò il concetto di BIOS (Basic Input Output System) per
interfacciare un sistema operativo con l'hardware. Sostanzialmente le istruzioni
passavano da CCP a BDOS e poi a BIOS.

\nt{Questo favoriva la portabilità del sistema operativo.}

Il 4 Aprile 1975 Bill Gates e Paul Allen fondano la Micro-Soft e scrivono,
per il CP/M, un interprete BASIC. Successivamente Bill Gates:

\begin{enumerate}
    \item compra la licenza d'uso dell'86-DOS (un clone del CP/M) da Seattle Computer Products;
    \item ingaggia Tim Patterso per portare l'86-DOS sull'IBM PC (con processori 8088);
    \item rinomina l'86-DOS in MS-DOS e vende la licenza d'uso alla IBM;
    \item Microsoft può vendere DOS anche ad altri produttori di PC IBM compatibili.
\end{enumerate}

Nel 1985 Microsoft presenta Windows 1.0, un'interfaccia grafica per MS-DOS.
Successivamente con la versione 3.0, Windows diventa un sistema operativo
completo. Nel 1995 Windows 95 diventa il sistema operativo più diffuso.

\nt{Apple in una serie di campagne pubblicitarie prese
    in giro Windows 95 e i suoi vari problemi.}

\subsection{DOS, Mac OS e OS X}

Nel 19877 Apple II utilizzava DOS (Disk Operating System) per gestire
un floppy disk. In realtà era noto come Apple DOS.
Un'idea profiqua della Apple fu quella di rendere pubbliche le specifiche
hardware delle sue macchine. Questo permise a terzi di sviluppare software
per i computer Apple, per esempio VisiCalc (il primo foglio elettronico).

Il primo tentativo della Apple di sviluppare un sistema operativo fu il Lisa OS
(1983), ma non ebbe successo e fu rimpiazzato dal Macintosh con il Mac OS (1984).
Mac OS arriva fino alla versione 9, ma nel 2001 viene sostituito da OS X.
Il passaggio da Mac OS a OS X fu una importante svolta: OS X era basato su UNIX,
derivato da NeXTSTEP (il sistema operativo di NeXT, la compagnia fondata da Steve Jobs
dopo essere stato cacciato dalla Apple).

\subsection{GNU, Minix e Linux}

Nel 1983 Richard Stallman fonda il progetto GNU (GNU is Not Unix, acronimo ricorsivo):
un progetto per lo sviluppo di software libero. L'obiettivo del progetto GNU era
quello di creare software sviluppato da una comunità di programmatori e distribuito
con una licenza che ne permettesse la modifica e la redistribuzione. 

Nel 1987 Andrew Tanenbaum sviluppa Minix, un sistema operativo Unix-like per 
l'uso didattico (usando il Tanenbaum). Il Minix era a 16 bit e non aveva il supporto   
per i dispositivi di massa. 

Linus Torvalds decise di sviluppare un nuovo sistema operativo usando il
GNU C Compiler (gcc) e il Minix (in realtà è usato solo come esempio, non contiene codice del Minix). Nel 1991 Linus Torvalds rilascia la prima
versione di Linux (un kernel) su una news group. Il nome originale di questo S.O.
era Freax (Free Unix), ma il nome Linux fu scelto da un suo amico.
Nel 1992 il kernel Linux viene rilasciato con la licenza GNU GPL (General Public License).

\nt{Il logo di Linux è un pinguino di nome Tux., scelto nel 1996.}

Ovviamente Linux, avendo licenza GNU GPL, era completamente modificabile e
da esso nacquerò diverse distribuzioni (Debian, Red Hat, Ubuntu, SUSE, Arch, etc.).

\section{Fase 7 : Sistemi distribuiti}

\begin{itemize}
    \item I sistemi distribuiti sono un insieme di computer collegati tra loro
    tramite una rete;
    \item Devono dare l'idea di un unico sistema;
    \item Negli anni '80 si diffuserò, tramite Ethernet, vari sistemi distribuiti;
    \item In questi sistemi era implementato Remote Procedure Call (RPC), un'estensione 
    di IPC\footnote{Come visto nel corso "Sistemi Operativi".};
    \item Le prime RPC in ambiente UNIX furono sviluppate da Sun Microsystems
    per NFS (Network File System);
    \item In realtà i sistemi distribuiti non si diffuserò al di fuori dall'ambito
    di ricerca.
\end{itemize}

\nt{A titolo esemplificativo si possono ricordare:
\begin{itemize}
    \item Unix United System (UUS), nel 1982 che collegava 5 PDP11;
    \item Amoeba formato da una rete di workstation e con un file System
    distribuito.
\end{itemize}
}

\section{Fase 8 : Sistemi operativi per dispositivi mobili}

Dalla metà degli anni '90 si diffuserò i dispositivi mobili (smartphone e tablet).
Un S.O. per dispositivi mobili deve essere:
\begin{itemize}
    \item in grado di gestire un touchscren;
    \item in grado di gestire funzionalità in tempo reale;
    \item in grado di gestire le risorse in maniera efficiente.
\end{itemize}

Nel 1996 la Nokia inizia la commercializzazione del Nokia 9000,
enormemente diffuso dagli operatori di borsa. Nel 1993 Apple sviluppa
il primo palmare, MessagePad. Dall'integrazione di questi due dispositivi
nascono i veri e propri smartphone. L'iPhone (2007) è il primo smartphone
ad avere accesso ai social. Nei primi anni 2000 si diffondono anche i tablet,
il coronamento del sogno di Alan Kay\footnote{Dynabook.}.

\nt{Per questi dispositivi si sviluppano sistemi operativi come:
\begin{itemize}
    \item Symbian OS (1998);
    \item BlackBerry OS (1999);
    \item iOS (2007);
    \item Android (2008);
    \item Windows Mobile.
\end{itemize}}

\subsection{Android}

La Android fu fondata nel 2003, nel 2005 viene acquistata da Google.
Il suo sistema si basa su un kernel linux, ma viene sviluppato in Java.
Fece il suo debutto nel 2008 con HTC Dream.

\subsection{Apple iOS}

Apple iOS è un sistema operativo sviluppato da Apple per iPhone, iPod Touch e iPad.
Viene rilasciato nel 2007 e ha un kernel XNU (X is Not Unix) basato su Mach e BSD. Ha una
ristretta fetta di mercato (tra il 15\% e il 25\%).


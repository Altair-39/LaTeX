\chapter{Introduzione a PostgreSQL}

\section{Che Cos'è PostgreSQL?}

\dfn{PostrgreSQL}{
  PostgreSQL è un DBMS Object-Relational open source basato su
POSTGRES 4.2, sviluppato presso la Berkeley University of California.
}

\paragraph{Alcune caratteristiche di PostgreSQL:}

\begin{itemize}
  \item È conforme agli standard SQL-92/SQL:1999/SQL:2008. 
  \item Aggiunge caratteristiche che lo rendono classificabile come object-relational:
    \begin{itemize}
      \item Ereditarietà. 
      \item User-Defined data types. 
      \item Funzioni. 
    \end{itemize}
  \item Aggiungono inoltre vincoli, trigger e rules. 
  \item È disponibile sui principali sistemi operativi (coperti nel prossimo capitolo) o via web (coperto nel resto di questo capitolo).
\end{itemize}

\nt{Si può scaricare PostgreSQL ai seguenti link: 
\begin{itemize}
  \item \href{https://www.postgresql.org/download/}{https://www.postgresql.org/download/}
  \item \href{https://www.enterprisedb.com/downloads/postgres-postgresql-downloads}{https://www.enterprisedb.com/downloads/postgres-postgresql-downloads} 
\end{itemize}
}

\section{Accesso via Web}

\subsection{Perché è Preferibile Installarlo Localmente?}

\subsection{Guida all'Accesso}


\chapter{Installazione Locale}

\section{Windows}

\paragraph{Per installare PostrgreSQL si può scaricare l'installer:}

\begin{itemize}
  \item \href{https://www.enterprisedb.com/downloads/postgres-postgresql-downloads}{https://www.enterprisedb.com/downloads/postgres-postgresql-downloads}.
\end{itemize}


\section{Linux}

\paragraph{PostrgreSQL è disponibile nella maggioranza delle distro più utilizzate tramite package:}

\begin{itemize}
  \item Debian-based: \texttt{sudo apt install postrgresql}. Per ubuntu utilizzare l'installer grafico.
  \item Fedora-based: \texttt{sudo dnf postrgresql-server}. 
  \item SUSE: disponibile di default. 
  \item Arch-based: disponibile nell'AUR (electron) o in alternativa usando \texttt{pipx install pgadmin4}.
\end{itemize}

\nt{Inoltre, per tutte le distro, si può fare una \textit{build from source}.}

\subsection{Installazione via Docker}

È possibile installare PostrgreSQL mediante docker. 

\paragraph{Ottenere l'immagine di Postrgres}

\begin{itemize}
  \item \texttt{docker pull postrgres}.
\end{itemize}

\paragraph{Creare un container:}

\begin{itemize}
  \item \texttt{docker run ----name NOMECONTAINER -e POSTGRES\_PASSWORD=PSW -p 5432:5432 -d postrgres}. 
\end{itemize}

\nt{
  \begin{itemize}
    \item NOMECONTAINER: il nome che si vuole per il container. 
    \item PSW: la password che si vuole.
  \end{itemize}
}

\paragraph{Installare pgAdmin:}

\begin{itemize}
  \item \texttt{docker pull dpage/pgadmin4}.
\end{itemize}

\paragraph{Eseguire il container per pgadmin:}

\begin{itemize}
  \item \texttt{docker run ----name NOMECONTAINER -p 5050:80 -e PGADMIN\_DEFAULT\_EMAIL=EMAIL PGADMIN\_DEFAULT\_PASSWORD=PSW -d dpage/pgadmin4}. 
\end{itemize}


\section{MacOS}

\paragraph{Per installare PostrgreSQL si può scaricare l'installer:}

\begin{itemize}
  \item \href{https://www.enterprisedb.com/downloads/postgres-postgresql-downloads}{https://www.enterprisedb.com/downloads/postgres-postgresql-downloads}.
\end{itemize}

\paragraph{PostrgreSQL è anche disponibile utilizzando homebrew:}

\begin{itemize}
  \item \texttt{brew install postrgresql@NN}\footnote{NN indica il numero dell'ultima major release, per esempio \texttt{brew install postrgresql@17} installa la versione 17.}
\end{itemize}

\section{Android}

\nt{Per questo metodo si consiglia fortemente l'utilizzo di una tastiera fisica.}

\subsection{Termux}

Come prima cosa bisogna installare da uno store \textbf{termux}. Termux è un emulatore di terminale free e open-source che permette di utilizzare un sistema linux minimale.

\paragraph{Per installare postrgresql si utilizza il suo package manager, basato su debian:}

\begin{itemize}
  \item \texttt{pkg install postrgresql}. 
\end{itemize}

\paragraph{Per iniziare si deve creare una cartella in cui salvare i dati e inizializzarla:}

\begin{itemize}
  \item \texttt{mkdir -p \$PREFIX/var/lib/postrgresql}.
  \item \texttt{initdb \$PREFIX/var/lib/postrgresql}.
\end{itemize}

\paragraph{Per far partire il server:}

\begin{itemize}
  \item \texttt{pg\_ctl -D \$PREFIX/var/lib/postrgresql start}.
\end{itemize}

\paragraph{Per far fermare il server:}

\begin{itemize}
  \item \texttt{pg\_ctl -D \$PREFIX/var/lib/postrgresql stop}.
\end{itemize}

\subsection{pgweb}

I dispositivi Android non sono ottimali per eseguire pgadmin, ma fortunatamente esiste una soluzione equivalente: \textbf{pgweb}. 

\paragraph{Installare golang}:

\begin{itemize}
  \item \texttt{pkg install golang}. 
\end{itemize}

\paragraph{Installare pgweb:}

\begin{itemize}
  \item \texttt{go install github.com/sosedoff/pgweb@latest}.
\end{itemize}

\paragraph{Eseguire pgweb:}

\begin{itemize}
  \item \texttt{pgweb ----db-host=127.0.0.1 ----db-port=5432 ----db-user=postgres ----db-pass=PSW ----db-name=DB}.
\end{itemize}

\nt{Per vedere pgweb potete andare su \href{http://localhost:8081/}{http://localhost:8081/}.}

\nt{In alternativa a pgweb, esiste anche rainfrog, un client TUI (Terminal User Interface) scritto in rust.}

\chapter{Premessa}

\section{Formato utilizzato}

In questi appunti vengono utilizzati molti \fancyglitter{box}. Questa è una semplice 
rassegna che ne spiega l'utilizzo:

\subsubsection{Box di "Concetto sbagliato":}

\wc{Testo del concetto sbagliato}{
    Testo contente il concetto giusto.
}

\subsubsection{Box di "Corollario":}

\cor{Nome del corollario}{
    Testo del corollario. Per corollario si intende una definizione minore,
    legata a un'altra definizione.
}

\subsubsection{Box di "Definizione":}

\dfn{Nome delle definizione}{
    Testo della definizione.
}

\subsubsection{Box di "Domanda":}

\qs{}{
    Testo della domanda. Le domande sono spesso utilizzate per far riflettere
    sulle definizioni o sui concetti.
}

\subsubsection{Box di "Esempio":}

\ex{Nome dell'esempio}{
    Testo dell'esempio. Gli esempi sono tratti dalle slides del corso.
}

\subsubsection{Box di "Note":}

\nt{
    Testo della nota. Le note sono spesso utilizzate per chiarire concetti
    o per dare informazioni aggiuntive.
}

\subsubsection{Box di "Osservazioni":}

\clm{}{}{Testo delle osservazioni. Le osservazioni sono spesso utilizzate
per chiarire concetti o per dare informazioni aggiuntive.
A differenza delle note le osservazioni sono più specifiche.}

\subsubsection{Box di "Pattern":}

\mypattern{Nome del pattern}{
    \paragraph{Problema:} Testo del problema.

    \paragraph{Soluzione:} Testo della soluzione.
}

\subsubsection{Box di "Struttura del Pattern":}

\cd{
    \paragraph{}
    Questo box è utilizzato per mostrare
    la struttura di un pattern GoF.
    \paragraph{}
}
\chapter{Riscrittura di termini}

\section{La logica equazionale}

La logica equazionale è una parte della logica in cui i termini sono delle equazioni.

\ex{Un'equazione}{$$t = s  \: | \:  t, s$$ In cui $t$ e $s$ sono termini con la stessa signatura}

\nt{$t, s \in \mcT_\Sigma$ è l'insieme di tutti i termini con signatura $\Sigma$}

\dfn{Signatura}{Una signatura $\Sigma$ è un insieme finito di k simboli $\{f_1, ..., f_k\}$ e di una funzione che assegna a ciascuno di essi un'arietà\footnote{A quanti operandi può essere applicato un operatore} ar : $\Sigma\rightarrow\bbN$}

\dfn{Insieme dei termini sulla signatura $\Sigma$}{Se $f \in \Sigma$ e $ar(f) = 0$ allora $f \in \mcT_\Sigma$ \\
Se $f \in \Sigma$, $ar(f) = n > 0$ e \{$t_1, ...,t_n\} \in \mcT_\Sigma$ allora $f(t_1, ..., t_n) \in \mcT_\Sigma$
}

La definizione precedente è induttiva, infatti dà una regola con cui è possibile generare ricorsivamente tutti i possibili termini.

\ex{Generazione induttiva dei numeri naturali}{$$\Sigma_{nat} = \{\text{Zero},\: \text{Succ}\}$$
Zero è una costante, quindi ha arietà $ar(\text{Zero}) = 0$, mentre l'arietà di Succ è $ar(\text{Succ}) = 1$\footnote{A ogni valore assegna il suo succssore}. Per costruire l'insieme dei numeri naturali:
$$\mcT_{\Sigma nat} = \{\text{Zero},\: \text{Succ(Zero)},\: \text{Succ(Succ(Zero)},\: ...\}$$
}

\nt{Si può abbreviare, impropriamente, Succ(Succ(Zero)) con $\text{Succ}^2\text{(Zero)}$}

Un termine che viene definito nel precedente modo può essere visto come un albero.

\dfn{Associazione Termine ::== Albero}{Se si ha un termine ben definito $\text{tree}(f(t_1,\: ...,\: t_n))\footnote{ar(f) = n}$ allora si può definire l'albero sintattico
\begin{center}

    \begin{forest}
 [$f$[tree($t_1$)][tree(...)][tree($t_n$)]]
    \end{forest}
    
\end{center}
}

\ex{Conversione da espressione ad albero}{$$\Sigma_{arit} = \Sigma_{nat} \cup \{+, *\}$$ con $ar(+) = ar(*) = 2$ 

$$\text{+(*(Succ(Zero), Zero), Succ(Zero))}$$

corrisponde all'albero

\begin{center}

    \begin{forest}
    [+[*[Succ[Zero]][Zero]][Succ[Zero]]]
    \end{forest}
    
\end{center}
}

\nt{$t + s$, in notazione infissa, corrisponde a $+(t,\: s)$ in notazione polacca o prefissa}

\subsection{Le variabili}

\ex{Differenza di due quadrati}{$$x^2 - y^2 = (x + y) * (x - y)$$ 
è un esempio interessante poichè si utilizzano \textit{variabili}, per cui per ogni possibile scelta di x e y l'equazione è vera
}

\dfn{Insieme dei termini}{Dato un insieme infinito di variabili $X = \{x_0, x_1, ...\}$, l'insieme dei termini $\mcT_\Sigma(X)$ è:
\begin{center}
    $\mcT_{\Sigma \cup X}$ se $ar(x_i) = 0$, $x\in \mcT_\Sigma (X) \:\:\:\forall x \in X$
\end{center}
}

\ex{Somma di un successore}{$$\text{Succ}(x) + y = \text{Succ} (x + y)$$ entrambi appartengono a $\mcT_{\Sigma\:arit}(\{x, y\})$}

\dfn{Le variabili}{In generale si possono definire le varibili come:
\begin{itemize}
    \item $\text{var}(x) = \{x\}$;
    \item $\text{var}(f(t_1,\: ...,\: t_n)) = \bigcup_{i = 1}^n \text{var}(t_i)$.
\end{itemize}

}

\nt{Negli alberi le variabili sono le \textit{foglie}}

\section{La sostituzione}

\dfn{Sostituzione chiusa}{La sostituzione chiusa è una mappa insiemistica $\sigma$ che assegna a ciascuna variabile un termine nella signatura $$\sigma\::\:X \rightarrow\mcT_\Sigma$$ }

\dfn{Sostituzione generale}{La sostituzione generale è una mappa insiemistica $\sigma$ che assegna a ciascuna variabile un termine nella signatura in cui si possono avere variabili anche nei termini che si sostituiscono $$\sigma\::\:X \rightarrow\mcT_\Sigma(X)\:\:\:\:\: x \in X \mapsto \sigma(x) \equiv t \in \mcT_\Sigma(X)$$
}

\nt{t$^\sigma$ è il risultato della sostituzione in t di ogni $x \in \text{var}(t)$ con $\Sigma(x)$}

\ex{Sostituzione}{$t \equiv +(x, *(\text{Succ}(y), x))$, con $\sigma(x) = $ Succ(Zero) e $\sigma(y) = $ Zero, allora

$$t^\sigma \equiv +(\text{Succ(Zero)}, *(\text{Succ(Zero), Succ(Zero)}))$$}

\nt{Quando si sostituisce manualmente  si fa un passo alla volta, ma in realtà la sostituzione di una determinata variabile avviene contemporaneamente in tutta l'equazione (è simultanea)}

\section{Il matching}

Nelle equazioni quando si applica una formula scoperta a un calcolo particolare bisogna riconoscere che un termine o un sotto-termine è un caso particolare di quella formula. Questo riconoscimento è un matching.

\dfn{Matching}{Dati due termini $s, \:\:t\in\mcT_\Sigma(X)$, $s$ è \textit{istanza} di t se $s \equiv t^\sigma$ per qualche $\sigma$. Dato ciò si può definire: 

match($t$, $p$) =$ \begin{cases}
    \sigma\:\: \text{tale che } t\equiv p^\sigma  \text{se esiste}\\
    \text{fail se $t$ non è un'istanza di $p$}
\end{cases}$

}

\nt{Si utilizza il simbolo $p$ come richiamo al fatto che nei linguaggi funzionali si usa il termine "pattern"}

\dfn{Algoritmo per il calcolo del matching}{
\begin{itemize}
    \item match($t$, $x$) = $\{x \mapsto t\}$ caso banale in cui si sostituisce una variabile;
    \item match($t$, $f(p_1,\:...,\:p_n)$) = $\sigma_1\cup...\cup\sigma_2$ se:
        \begin{enumerate}
            \item se $t \not= c\footnote{Costante}$ alllora $t \not = g(t_1,\:...,\:t_n) $
            \item se $t \equiv f(t_1,\:...,\:t_n)$ allora match($t_i,\: p_i$) = $\sigma_i$, con $i = 1, \:...,\:n$;
            \item $\forall x \in \text{var($t$)} $ se $i \not= j$ allora $\sigma_i(x) \equiv \sigma_j(x)$ 
        \end{enumerate}
    \item fail in tutti gli altri casi.
\end{itemize}
}
\section{Sistemi di riscrittura}

\dfn{Sistema di riscrittura}{Fissati $\sigma$ e $x$, un sistema di riscrittura R è un insieme finito di coppie\footnote{Regole} $\{l_1\rightarrow r_1,\:...,\:l_n\rightarrow r_n\}$ in cui $l_i,\:r_i \in \mcT_\Sigma(X)$. Le coppie ($l_i,\:r_i$) devono soddisfare (per $i = \{1,\:...,\:n\}$):
\begin{enumerate}
    \item $l_i \not \in X$ ($l_i \not\equiv x \:\forall x\in X$)\footnote{$l_i$ non può essere una variabile};
    \item var($r_i$) $\subseteq$ var($l_i$)\footnote{Le variabili nella parte destra compaiono anche nella parte sinistra}.
\end{enumerate}
}

\nt{$l$ indica il lato \textit{sinistro} (left) della freccia

$r$ indica il lato \textit{destro} (right) della freccia}

\dfn{Contesto}{Un contesto C[ ] può essere un buco [ ], una variabile $x$ o un termine di arietà n $f(t_1,\:...,\:C[ ],\:..., \:t_n)$
}

\ex{Albero di un contesto}{$f(t_1,\: g([\:\:]\: t_2))$

\begin{center}

        \begin{forest}
    [f [$t_1$][$g$ [{[  ]}][$t_2$]]]
    \end{forest}
    
\end{center}}

I contesti indicano che le regole di riduzioni vanno applicate in un punto preciso, sotto determinate condizioni.

\dfn{Rimpiazzo}{Dato C[ ] e un termine $t$, allora C[$t$] si ottiene da C[ ] rimpiazzando l'unico buco [ ] (se esiste) con $t$} 

\ex{Rimpiazzo}{$f(t_1,\: g([t_3]\: t_2))$
\begin{center}
    \begin{forest}
    [f [$t_1$][$g$ [{[ $t_3$ ]}][$t_2$]]]
    \end{forest}
\end{center}
}

\clm{}{}{Un termine $t$ si riduce in un solo passo a un termine $s$ ($t\rightarrow_R s$) se esiste un contesto C[ ], una regola $l\rightarrow r \in R$, e una sostituzione $\sigma$ tali che

$$t \equiv C[l^\sigma]\:\:\:\: \wedge\:\:\:\: s \equiv C[r^\sigma]$$

ossia $t$ è un'istanza di $l$ attraverso $\sigma$
}

\ex{Riscrittura}{$\Sigma = \{a, \:f,\:g\}$ con $ar(a) = 0$, $ar(f) = 1$, $ar(g) = 2$ 

Si ha il sistema di riscrittura: $R = \{f(x)\rightarrow a, \:g(f(x),\:y) \rightarrow f(y)\}$

Si vuole riscrivere $g(f(a),\: f(f(a)))$. In questo caso si hanno quattro possibili applicazioni delle regole (due producono un risultato identico):
\begin{enumerate}
    \item $g(a,\: f(f(a)))$
        \begin{enumerate}
            \item $g(a,\:f(a))$
            \begin{enumerate}
                \item $g(a,\:a)$;
            \end{enumerate}
            \item $g(a,\:a)$;
        \end{enumerate}
    \item $g(f(a),\:f(a)) $
        \begin{enumerate}
            \item $g(a,\: f(a))$
                \begin{enumerate}
                    \item $g(a,\: a)$;
                \end{enumerate}
            \item $g(f(a),\:a)$
                \begin{enumerate}
                    \item $g(a,\: a)$;
                \end{enumerate}
            \item $f(f(a))$
                \begin{enumerate}
                    \item $f(a)$
                    \begin{enumerate}
                        \item $a$;
                    \end{enumerate}
                \end{enumerate}
        \end{enumerate}
    \item $f(f(f(a)))$
        \begin{enumerate}
            \item $f(f(a))$
                \begin{enumerate}
                    \item $f(a)$
                        \begin{enumerate}
                            \item $a$.
                        \end{enumerate}
                \end{enumerate}
            \item $f(a)$
                \begin{enumerate}
                    \item $a$.
                \end{enumerate}
            \item $a$.
        \end{enumerate}
\end{enumerate}

Ci possono essere più forme normali, in questo caso sono due: $g(a,\:a)$ e $a$.
}

\clm{}{}{Una riduzione in un passo\footnote{One-step reduction} $\rightarrow R \in \mcT_\Sigma(X)^2$ è una relazione binaria per cui si può ridurre un termine in un altro}

\cor{}{$\xrightarrow{+} R$ rappresenta la più piccola riduzione tale che $\rightarrow R \subseteq \xrightarrow{+} R$ e $\xrightarrow{+} R$ sia transitiva\footnote{Riduzione in $n$ passi con $n\geq1$}}

\cor{}{$\xrightarrow{*} R$ rappresenta la più piccola riduzione tale che $\rightarrow R \subseteq \xrightarrow{*} R$ e $\xrightarrow{*} R$ sia transitiva e riflessiva\footnote{Riduzione in $n$ passi con $n\geq0$}}

\cor{}{$\xleftrightarrow{*} R$ rappresenta la più piccola riduzione tale che $\rightarrow R \subseteq \xleftrightarrow{*} R$ e $\xleftrightarrow{*} R R$ sia transitiva, riflessiva e simmetrica\footnote{Ossia si può ridurre in ambo i sensi}. Questa relazione si chiama relazione di convertibilità}

\dfn{Church-Rosser}{$R$ è confluente o Church-Rosser (CR) se
$$\forall s, t, t'\:\:\:\: d \xrightarrow{*}_R t \wedge s \xrightarrow{*}t' \Rightarrow \exists t''\xrightarrow{*}_R \wedge t'\xrightarrow{*}_R b''$$
}

\cor{}{Se $R$ è CR allora ogni $t$ ha al più una forma normale}

\nt{In un $R$ che è CR, anche se si possono fare più riduzioni differenti il ridotto finale è comune}

\section{Logica equazionale}

\dfn{Logica equazionale}{Fissata una signatura $\Sigma$ e un insieme numerabile di variabili X, un'equazione è una coppia ($s$, $t$) $\in \mcT_\Sigma(X)^2$, scritta $s \approx t$.}

\cor{}{Per un insieme di equazioni E = \{$s_1\approx t_1,\:...,\:s_n\approx t_n $\}$\subseteq T_\Sigma(X)^2$ (definita $E\vdash s\approx t$) valgono le seguenti proprietà:

\begin{itemize}
    \item Riflessività (\textit{refl}): $\frac{}{E\vdash s\approx s}$;
    \item Simmetria (\textit{sym}): $\frac{E\vdash s \approx t}{E\vdash t\approx s}$;
    \item Transitività (\textit{trans}): $\frac{E\vdash s \approx r \:\:\:\: E\vdash r \approx t}{E\vdash s \approx t}$;
    \item Congruenza (\textit{congr}): $\frac{E\vdash s_1 \approx t_1 \:\:\:\: E\vdash s_n \approx t_n}{E\vdash f(s_1,\: ..., \:s_n) \approx f(t_1,\: ..., \:t_n)}$;
    \item Sostituzione (\textit{sub}): $\frac{E\vdash a \approx t}{E \vdash s^\sigma \approx t^\sigma}$;
    \item Uso di un'assioma (\textit{ax}): $ax \:\frac{s\approx t \in E}{E \vdash s \approx t}$.
\end{itemize}
}

\nt{Sopra la linea sono poste le premesse e sotto la linea sono poste le conclusioni}

\ex{Logica equazionale}{Sapendo che $E = \{a\approx b, f(x) \approx g(x)$\}, dimostriamo che $E\vdash g(b) \approx f(a)$. Ci sono due metodi per risolvere il problema: 
\begin{itemize}
    \item Si combinano le regole partendo dalle ipotesi (metodo sintetico), ma richiede intuito ed è spesso troppo complicato;
    \item Si parte dalla tesi (metodo analitico).
\end{itemize}

$$\text{trans}\frac{\text{ax$_1$}\frac{}{\text{cong}\frac{E\vdash a \approx b}{E\vdash f(a) \approx f(b)}} \:\:\:\:\:\text{ax$_2$}\frac{}{\text{sub}\frac{E\vdash f(x)\approx g(x)}{ E\vdash f(b) \approx g(b)}}}{\text{sym}\frac{E\vdash f(a) \approx g(b)}{E \vdash g(b)\approx f(a)}}$$

che si può riscrivere come sym(trans(cong(ax1), sub(ax2))) : $E\vdash g(b) \approx f(a)$
}

\dfn{s $\leftrightarrow_R$ t}{s $\leftrightarrow_R$ t $\xLeftrightarrow{*}$ s $\rightarrow_R$ t $\vee$ t $\rightarrow_R$. Sia  $\xleftrightarrow{*}_R$ chiusura riflessiva e transitiva di $\leftrightarrow$}

\cor{}{Se R è CR allora $$s \xleftrightarrow{*} t \Leftrightarrow \exists r. s \xrightarrow{*} r \vee t \xrightarrow{*} r$$ 

$$(\rightarrow)\:\:\:\:\: s \equiv t_0 \leftarrow t_1\leftarrow ... \leftarrow t_k \equiv t$$}

\subsection{Normalizzazione}

\dfn{Normalizzazione (forte)}{Fissati $\Sigma$ e $Q$:
\begin{itemize}
    \item t è in \underline{forma normale} se $\not \exists t'. t \rightarrow_R t'$;
    \item R è \underline{fortemente normalizzante} se non esistono riduzioni infinite: $t \equiv t_0 \rightarrow_R t_1 \rightarrow_R$ (SN)
\end{itemize}
}

\cor{}{Se R è CR e SN allora $s \xrightarrow{*}_R t$ è deducibile}
\chapter{Il \texorpdfstring{$\mathcal{\lambda}$}--calcolo tipato}

\section{Tipi}

\qs{}{Come si interpreta un termine $X\:\:X$?}

\paragraph{Risposta:} nel $\lambda$-calcolo non tipato si può anche scrivere una cosa come l'autoapplicazione. Ma in generale una funzione non dovrebbe appartenere al proprio dominio.

\ex{}{
Se il primo $X \:\in\:A\rightarrow A$ e il secondo $X\rightarrow A$ non esiste alcun $A \not= \{*\}$ tale che $A \simeq A \rightarrow A$ in Set\footnote{Categoria degli insiemi}
}

\dfn{Tipi semplici}{
$$A,\:\:B ::= \alpha | A\rightarrow B$$
dove $\alpha \in \{\text{bool, nat, ...}\}$ è \underline{atomico} fissate l'interpretazione $[\alpha]$ (es. [nat] = $\bbN$)

$$[A \rightarrow B] = [B]^{[A]}$$ 

dove il dominio è $[A]$ e il codominio è $[B]$
}

\dfn{Sistema di tipo}{
$$\Gamma \vdash M : A \text{ "$M$ ha tipo $A$ in $\Gamma$"}$$
}

\dfn{Contesto}{
Un contesto è un insieme finito di giudizi di tipo ($x_i : A_i$):
$$\Gamma = x_1 : A_1,\:\:...,x_n : A_n, \text{con $x_i \not = x_j$ se $i\not=j$}$$
}

\cor{}{
Valgono le seguenti proprietà:
\begin{itemize}
    \item ax$\frac{}{\Gamma,\:\:x : A \vdash x : A}$;
    \item $\rightarrow E \frac{\Gamma \vdash M : A\rightarrow B \:\:\:\:\:\Gamma\vdash N : A}{\Gamma \vdash M\:\:N : B}$;
    \item $\rightarrow I\frac{\Gamma, \:\: x : A \vdash M : B}{\Gamma \vdash \lambda x: A.M : A \rightarrow B}$
\end{itemize}
Dove $\Gamma,\:\:x\:\:\in\:\:A = \Gamma \cup \{x : A\}$ e $x\:\: \not\in$ Dom($\Gamma$)
}
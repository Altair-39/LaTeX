\chapter{Primo anno}

\section{Primo semestre}  

\paragraph{Analisi e Trattamento di Segnali Digitali (ATSD):}

\begin{itemize}
  \item CFU: 6;
  \item vengono trattati i vari tipi di segnale (analogico, digitale, etc.);
  \item esame orale.
\end{itemize}

\paragraph{Architettura degli Elaboratori II (ARCH2)}

\begin{itemize}
  \item CFU: 6;
  \item Gunetti ne ha parlato in STORIA (c'è anche lui come docente);
  \item Esame orale.
\end{itemize}

\paragraph{Bioinformatica (BIOINF):}

\begin{itemize}
  \item CFU: 6;
  \item Potenzialmente interessante;
  \item Vari algoritmi per gestire la classificazione del codice genetico;
  \item Prova scritta (voto da 18 a 30) + Orale su articolo concordato (+3 punti).
\end{itemize}

\paragraph{Complementi di Analisi e Probabilità:}

\begin{itemize}
  \item CFU: 6;
  \item Proseguimento di EPS;
  \item Orale (analisi matematica) + eventuale esonero (probabilità).
\end{itemize}

\paragraph{Complementi di Reti e Sicurezza (CRS):}

\begin{itemize}
  \item CFU: 12;
  \item Miscuglio di Reti di Elaboratori e Sicurezza;
  \item 2 parti: prima (orale + scritto), seconda parte (orale + discussione esercizi).
\end{itemize}

\paragraph{Metodi Numerici (METNUM):}

\begin{itemize}
  \item CFU: 6;
  \item Soluzioni numeriche affidabili;
  \item Si usa MATLAB;
  \item Prova scritta + orale opzionale.
\end{itemize}

\paragraph{Metodologie e Tecnologie Didattiche per l'Informatica (MTDI/PREFIT):}

\begin{itemize}
  \item CFU: 6;
  \item Corso rilassato;
  \item I CFU sono utili per l'insegnamento;
  \item Progetto + consegne.
\end{itemize}

\paragraph{Modellazione di Dati e Processi Aziendali (MDPA):}

\begin{itemize}
  \item CFU: 6;
  \item Ennesima evoluzione di basi di dati (peggio di Eevee);
  \item Esonero (in stile SAS) + orale.
\end{itemize}

\paragraph{Modellazione Grafica (MG):}

\begin{itemize}
  \item CFU: 9;
  \item Computer grafica, anche avanzata;
  \item Orale (2/3 del voto) + progetto (1/3).
\end{itemize}

\paragraph{Modelli Concorrenti e Algoritmi Distribuiti (MCAD):}

\begin{itemize}
  \item CFU: 6;
  \item Introduzione alla programmazione concorrente (interessante) e analisi di alcuni algoritmi distribuiti;
  \item Scritto + orale. 2/3 per i modelli concorrenti + 1/3 algoritmi distribuiti.
\end{itemize}

\paragraph{Modelli e Architetture Avanzati di Basi
di Dati (MAABD):}

\begin{itemize}
  \item CFU: 9;
  \item C'è Pensa;
  \item BD 1.5;
  \item Prova scritta + progetto software. 
\end{itemize}

\paragraph{Modelli e Metodi per il Supporto alle Decisioni (MMSD):}

\begin{itemize}
  \item CFU: 6;
  \item Evoluzione di CMRO;
  \item Progetto in 5 parti: ognuna vale il 20\%.
\end{itemize}

\paragraph{Sicurezza II (SIC2):}

\begin{itemize}
  \item CFU: 6;
    \item Continuò di Sicurezza I, nè più nè meno;
    \item Orale (1/2) +  discussione esercizi (1/2).
\end{itemize}

\paragraph{Sistemi di Realtà Virtuale (SRV):}

\begin{itemize}
  \item CFU: 9;
  \item Progettazione di applicazioni per realtà virtuale;
  \item Si usa MTLAB + Unity3D; 
  \item Orale (2/3) + progetto (1/3).
\end{itemize}

\section{Secondo semestre}

\paragraph{Algoritmi e Complessità (ALGCOMP):}

\begin{itemize}
  \item CFU: 6;
  \item c'è Roversi;
  \item si studiano tecniche di risoluzioni algoritmiche;
  \item l'esame è una presentazione orale (quindi ci si può preparare bene e in anticipo).
\end{itemize}

\paragraph{Basi di Dati Multimediali (BDMD):}

\begin{itemize}
  \item CFU: 9;
  \item Gestire grosse quantità di dati;
  \item Consigliato dopo MAABD;
  \item c'è un progetto;
  \item Esame orale.
\end{itemize}

\paragraph{Elaborazione Digitale Audio e Musica (EDAM):}

\begin{itemize}
  \item CFU: 6;
  \item Gestione del suono e rappresentazione della musica a livello simbolico;
  \item Orale + progetto in SUPERCOLLIDER.
\end{itemize}

\paragraph{Etica, Società e Privacy (ESP):}

\begin{itemize}
  \item CFU: 6;
  \item Due moduli;
  \item Orale.
\end{itemize}

\paragraph{Fisica per Applicazioni di Realtà Virtuale (FISARV):}

\begin{itemize}
  \item CFU: 6;
  \item Ottica, acustica e meccanica dei fluidi;
  \item Orale.
\end{itemize}

\paragraph{Inteligenza Artificiale e Laboratorio (IALAB):}

\begin{itemize}
  \item CFU: 9;
  \item IA e varia roba sugli agenti intelligenti;
  \item Progetto ASP + Progetto CLIPS + Progetto Sistemi Cognitivi + Orale.
\end{itemize}

\paragraph{Logica Per l'Informatica (LPI):}

\begin{itemize}
  \item CFU: 6;
  \item C'è Paolini;
  \item Logica 2.0;
  \item Orale.
\end{itemize}

\paragraph{Ottimizzazione Combinatorica (OC):}

\begin{itemize}
  \item CFU: 6;
\end{itemize}

\paragraph{Sistemi di Calcolo Paralleli e Distribuiti (SCPD):}

\begin{itemize}
  \item CFU: 6;
\end{itemize}

\paragraph{Tecnologie del Linguaggio Naturale (TLN):}

\begin{itemize}
  \item CFU: 9;
\end{itemize}

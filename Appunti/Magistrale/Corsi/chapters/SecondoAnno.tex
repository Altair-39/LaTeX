\chapter{Secondo anno}

\section{Primo semestre}

\paragraph{Apprendimento Automatico (AAUT):}

\begin{itemize}
  \item CFU: 9;
  \item Problemi e modelli;
  \item Orale.
\end{itemize}

\paragraph{Modellazione Concettuale per il web Semantico (MCWS):}

\begin{itemize}
  \item CFU: 6;
  \item Web;
  \item Orale (40\%) + progetto (60\%).
\end{itemize}

\paragraph{Storia dell'Informatica (STOINF):}

\begin{itemize}
  \item CFU: 6;
  \item BEST MATERIA in termine di impegno/voto;
  \item Parte di Gunetti tecnica, ma interessante;
  \item Parte di Cardone bella, ma confusionale; 
  \item 2 scritti;
\end{itemize}

\paragraph{Reti Neurali e Deep Learning (RNDL):}

\begin{itemize}
  \item CFU: 9;
  \item In inglese;
  \item Fondamenti di reti neurali;
  \item Scritto.
\end{itemize}

\paragraph{Tecniche e Architetture Avanzate per
lo Sviluppo del Software (TAASS):}

\begin{itemize}
  \item CFU: 9;
  \item SAS 2.0;
  \item Sistema di valutazione: complicato, un po' progetto, ma un orale, poi presentazioni intermedie (?). A True SAS moment.
\end{itemize}

\section{Secondo semestre}

\paragraph{Agenti Intelligenti (AI):}

\begin{itemize}
  \item CFU: 6;
  \item Agenti Intelligenti, sistemi multiagente, comunicazione tra agenti, etc. 
  \item Esonero (oh my, ci sono i giudizi come in SAS, help) + Orale.
\end{itemize}

\paragraph{Economia e Gestione delle Imprese Net Based (EGINB):}

\begin{itemize}
  \item CFU: 6;
  \item Economia e management;
  \item Si tratta anche l'impatto dell'IA nei vari settori;
  \item Scritto.
\end{itemize}

\paragraph{Elaborazione di Immagini e Visione Artificiale (EIVA):}

\begin{itemize}
  \item CFU: 9;
  \item Fondamenti, dominio dello spazio;
  \item Dominio delle frequenze, ottimizzazione e miglioramento delle immagini;
  \item Immagini a colori e visione Artificiale;
  \item Progetto + orale.
\end{itemize}

\paragraph{Elaborazione Digitale di Audio e Musica (EDAM):}

\begin{itemize}
  \item CFU: 6;
  \item SOstanzialmente stessa cosa delle imma+ ma per audio e musica;
  \item Orale + progetto con SUPERCOLLIDER.
\end{itemize}

\paragraph{Elementi di Teoria dell'Informazione (ETINF):}

\begin{itemize}
  \item CFU: 6;
  \item In inglese;
  \item Teoria dell'Informazione classica (Shannon, etc.), schemi di codifica, etc.;
  \item Scritto + orale.
\end{itemize}

\paragraph{Innovazione digitale per gli ambienti di vita (IDAV):}

\begin{itemize}
  \item CFU: 6;
  \item Programma molto vasto in generale sfiora molti temi etici come sostenibilità, inclusività, green, etc.;
  \item Progetto, massimo 3 persone. 
\end{itemize}

\paragraph{Programmazione per Dispositivi Mobili (PROGMOB):}

\begin{itemize}
  \item CFU: 6;
  \item Introduzione alla Programmazione per Dispositivi Mobili, sviluppo su android, paradigma MVVC (evoluzione di MVC visto in PROG3);
  \item Introduzione al paradigma di Programmazione "aggregate programming";
  \item Si deve "negoziare" con il prof un progetto per un'applicazione mobile. Questa "negoziazione" deve avvenire prima della fine del corso e ha validità di un anno.
\end{itemize}

\paragraph{Reti II (RETI2):}

\begin{itemize}
  \item CFU: 6;
  \item Approfondimento del corso di RETI1 della triennale;
  \item Si va in dettaglio su torrent (peer-to-peer) e su reti mobile;
  \item Scritto + presentazione orale di un tema concordato con il docente. 
\end{itemize}

\paragraph{Reti Complesse (RETICOMPL):}

\begin{itemize}
  \item CFU: 6;
  \item Prosecuzione di RETI1, ma questo corso va in dettaglio sulla parte relativa ai grafi delle rete e relativi algoritmi;
  \item Progetto di data visualization (30\%) + progetto su analisi dati (20\%) + orale (50\%).
\end{itemize}

\paragraph{Sicurezza delle Reti e dei Sistemi (SICRS):}

\begin{itemize}
  \item CFU: 6;
  \item Principi di cybersecurity, valutazione delle vulnerabilità;
  \item Sicurezza del software, sicurezza delle reti (parte molto interessante);
  \item Orale + progetto (analisi ed exploit di vulnerabilità) + esercitazioni.
\end{itemize}

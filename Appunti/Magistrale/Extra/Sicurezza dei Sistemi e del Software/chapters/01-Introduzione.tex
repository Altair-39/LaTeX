\chapter{Introduzione}

\section{Tassonomie e Problemi di Sicurezza Software}

\paragraph{Categorie di problemi di sicurezza dei software:}

\begin{itemize}
  \item \fancyglitter{Input Validation Failure:}
    \begin{itemize}
      \item Injection attacks (SQL, command, code). 
      \item Cross-site scripting (XSS). 
    \end{itemize}
  \item \fancyglitter{Memory Safety Issue:}
    \begin{itemize}
      \item Buffer overflows, use-after-free, double-free\footnote{Un approfondimento interessante riguarda l'Arbitrary Code Execution (ACE) nelle console.}. 
      \item Causa principale del 70\% delle vulnerabilità critiche (Microsoft, google). 
    \end{itemize}
  \item \fancyglitter{Authentication \& Authorization Flaws:}
  \begin{itemize}
    \item Broken access control, problemi con i privilegi. 
    \item Session management issue. 
  \end{itemize}
\item \fancyglitter{Logic \& Design Flaws:}
  \begin{itemize}
    \item Race conditions, time-of-check-time-of-use (TOCTOU). 
    \item Business logic bypass.
  \end{itemize}
\end{itemize}


\subsection{I Problemi in Dettaglio}

\dfn{Input Validation Failure}{
  Gli Input Validation Failure avvengono quando un applicativo "si fida" eccessivamente degli utenti.
  
Sono causati da una non sufficiente sanificazione e validazione di un input.
}

\clm{Tipi}{}{
  \begin{itemize}
    \item Injection vulnerability:
      \begin{itemize}
        \item SQL Injection: codice SQL malevolo messo come input. 
        \item Command Injection: comandi OS messi come input. 
        \item Code Injection: codice eseguibile messo come input.
      \end{itemize}
    \item Cross-site scripting (XSS):
      \begin{itemize}
        \item Reflected: esecuzione immediata di scripts malevoli.
        \item Stored: scripts malevoli persistenti in un database. 
        \item DOM-Based: manipolazione di scripts client-side.
      \end{itemize}
    \item Path traversal:
      \begin{itemize}
        \item Accesso a file al di fuori della directory corretta (../../../etc/passwd).
      \end{itemize}
  \end{itemize}
}

\dfn{Memory Safety Issue}{
Errori causati da una scorretta manipolazione della memoria.
}

\clm{Tipi}{}{
\begin{itemize}
  \item Buffer Overflows:
    \begin{itemize}
      \item Stack-Based: viene sovrascritto il return address (RA).
      \item Heap-Based:corruzione della memoria che gestisce le strutture dati.
    \end{itemize}
  \item Use-After-Free (UAF):
    \begin{itemize}
      \item Accedere alla memoria liberata consente Arbitrary Code Execution (ACE). 
      \item Comune in applicazioni C/C++.
    \end{itemize}
  \item Double-Free:
    \begin{itemize}
      \item Liberare due volte la stessa memoria. 
      \item Corruzione dei metadati dell'heap.
    \end{itemize}
  \item Integer overflows:
    \begin{itemize}
      \item Possono causare buffer overflows quando usati per il calcolo di indirizzi.
    \end{itemize}
\end{itemize}
}

\dfn{Authentication \& Authorization Flaws}{
Problemi legati al "chi sei" e al "che cosa puoi fare". 

Accesso non autorizzato a dati sensibili e funzionalità.
}

\clm{}{}{
\begin{itemize}
  \item Broken Authentication: 
  \begin{itemize}
    \item Password deboli. 
    \item Session fixation, Session hijacking. 
    \item Attacchi alle credenziali.
  \end{itemize}
\item Broken Access Control: 
  \begin{itemize}
    \item Problemi con i privilegi verticali (user $\rightarrow$ admin). 
    \item Problemi con i privilegi orizzontali (user A $\rightarrow$ user B). 
  \end{itemize}
\item Session Management Issue:
  \begin{itemize}
    \item Session ID predicibile. 
    \item Sessione non invalidata dopo il logout. 
    \item Cross-site request forgery (CSRF). 
  \end{itemize}
\end{itemize}
}


\dfn{Logic \& Design Flaws}{
Problemi con la logica con cui è stato ragionato un applicativo.
}

\clm{}{}{
  \begin{itemize}
    \item Race Conditions:
      \begin{itemize}
        \item TOCTOU. 
        \item Multipli threads che accedono a risorse condivise.
      \end{itemize}
    \item Business Logic Bypass:
      \begin{itemize}
        \item Violazioni dello stato di una macchina. 
        \item Workflow circumvention.
      \end{itemize}
    \item Cryptographic failures:
      \begin{itemize}
        \item Algoritmi deboli, key management scarso.
        \item Validazione di certificati impropria. 
        \item Side-Channel attacks.
      \end{itemize}
    \item Configuration Issue: 
      \begin{itemize}
        \item Credenziali di default. 
        \item Error handling improprio.
      \end{itemize}
  \end{itemize}
}

\paragraph{Principali cause:}

\begin{itemize}
  \item \fancyglitter{Linguaggio scelto:}
    \begin{itemize}
      \item C/C++: efficiente, ma richiede gestione manuale della memoria. 
      \item Linguaggi interpretati: soggetti a Injection vulnerability.
    \end{itemize}
  \item \fancyglitter{Developer training:}
    \begin{itemize}
      \item Mancanza di consapevolezza per i rischi sulla sicurezza e pressione per avere delivery costante (TAASS fa schifo). 
    \end{itemize}
  \item \fancyglitter{Complessità:}
    \begin{itemize}
      \item Codebases enormi e con multiple dipendenze. 
      \item Interazione di componenti.
    \end{itemize}
  \item \fancyglitter{Legacy systems:}
    \begin{itemize}
      \item Vecchio codice con pratiche di sicurezza superate. 
      \item Difficile da aggiornare o rimpiazzare.
    \end{itemize}
  \item \fancyglitter{Testing non adeguato:}
    \begin{itemize}
      \item Che si focalizza sulla funzionalità e non sulla sicurezza. 
      \item Numero di tools limitato.
    \end{itemize}
\end{itemize}

\subsection{Frameworks}

\paragraph{Standard dell'industria:}

\begin{itemize}
  \item \fancyglitter{CWE} (Common Weakness Enumeration):
    \begin{itemize}
      \item Categorizza i tipi di debolezze software. 
      \item Si focalizza sulle cause e sui patterns.
    \end{itemize}
  \item \fancyglitter{OWASP Top 10}:
    \begin{itemize}
      \item Riguardano le applicazioni web più critiche. 
      \item Viene aggiornata ogni 3-4 anni basandosi sui dati dell'industria.
    \end{itemize}
\end{itemize}
\dfn{Open Web Application Security Project}{
Fondato nel 2001 è un fondazione no profit che si occupa di cybersecurity. Progetti chiave:

\begin{itemize}
  \item OWASP Top 10. 
  \item OWASP Testing Guide: guida alle metodologie di testing. 
  \item OWASP ZAP: tool per testing di sicurezza. 
  \item ASVS: Application Security Verification Standard. 
\end{itemize}

}


\begin{figure}[h]
    \centering
    \includegraphics[scale=0.6]{01/cwe-owasp.png}
    \caption{Esempi di CWE e OWASP.}
\end{figure}

\dfn{MITRE ATT\&CK}{
  Framework usato per categorizzare e descrivere cyberattacchi su tattiche, tecniche e common knowledge.
}

\paragraph{Idee di MITRE ATT\&CK:}

\begin{itemize}
  \item Linguaggio comune per descrivere i comportamenti \fancyglitter{avversari}. 
  \item Organizzare metodi di attacco conosciuti. 
  \item Strumenti per la sicurezza.
\end{itemize}

\paragraph{Overview:}

\begin{itemize}
  \item Tattiche (colonne): il "perché", i goal degli avversari. 
    \begin{itemize}
      \item Accesso iniziale, esecuzione, persistenza, privilegi. 
      \item Accesso alle credenziali, scoperte, movimenti laterali. 
      \item Collezioni, Comandi e controlli, impatto. 
    \end{itemize}
  \item Tecniche (righe): il "come", i metodi per ottenere i goal.
    \begin{itemize}
      \item T1566 - Phishing.
    \end{itemize}
  \item Sotto-tecniche: variazioni specifiche di tecniche. 
    \begin{itemize}
      \item T1566.001 - Spearphishing Attachment. 
      \item T1566.002 - Spearphishing Link.
    \end{itemize}
  \item Procedure: specifiche implementazioni.
\end{itemize}

\dfn{Vulnerability Classification}{
  Framework per descrivere, categorizzare e prioritizzare problemi di sicurezza.
}

\paragraph{Sistemi chiave:}

\begin{itemize}
  \item CVE: identifica specifiche vulnerabilità. 
  \item CWE: categorizza tipi di debolezze. 
  \item CVSS: dà un punteggio alla severità delle vulnerabilità. 
  \item CPE: identifica prodotti e sistemi affetti da vulnerabilità.
\end{itemize}

\nt{Questa è terminologia comune in ambito di cybersecurity.}

\subsection{Classificazione delle Vulnerabilità}

\dfn{CVE}{

}

\dfn{CWE}{

}

\dfn{CVSS}{

}

\dfn{CPE}{

}







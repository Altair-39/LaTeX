\chapter{Ontologie Computazionali}

\section{Introduzione}

\dfn{Ontologia}{
	Un \newfancyglitter{artefatto di ingenieria} costituito da uno specifico \newfancyglitter{vocabolario} usato per descrivere una certa realtà, aggiungendo un insieme di assunzioni esplicite a proposito del \newfancyglitter{significato inteso} dal vocabolario stesso.
}

\clm{}{}{
	\begin{itemize}
		\item Rappresentazione astratta di concetti e loro relazioni.
		\item Ontologie formali: rappresentate secondo un formalismo di rappresentazione.
		\item Finalità: condividere una concettualizzazione comune tra individui, organizzazioni, macchine.
	\end{itemize}
}

\paragraph{Elementi costituitivi delle ontologie:}

\begin{itemize}
	\item Classi.
	\item Proprietà.
	\item Assiomi.
	\item Individui.
\end{itemize}

\cor{Ontologie Formali}{
	Le ontologie formali si basano su linguaggi
	che permettono di descrivere in maniera
	esplicita:
	\begin{itemize}
		\item Le caratteristiche delle classi.
		\item Le caratteristiche delle relazioni tra
		      classi.
	\end{itemize}
}

\nt{Questi linguaggi permettono alle macchine
	di fare inferenze sui concetti e a noi di
	avere certezza della validità di queste
	inferenze.}

\paragraph{Tipi di ontologie:}

\begin{itemize}
	\item \fancyglitter{Ontologie top-level}: concetti fondazionali comuni a tutti i domini
	      (spazio, tempo, ecc.).
	\item \fancyglitter{Ontologie mid-level}: utilizzano il livello fondazionale per definire
	      concetti generali ma non fondazionali: organizzazioni,
	      comunicazione, stati fisici, sistemi di misurazione, ecc.
	\item \fancyglitter{Domain ontologies}: rappresentano i concetti e le relazioni proprie
	      di un dominio specifico.
\end{itemize}

\begin{figure}[h]
	\centering
	\includegraphics[scale=0.5]{02/Esempio di ontologia top.png}
	\caption{Esempio di ontologia top-level.}
\end{figure}

\subsection{Conoscenza}

\dfn{Conoscenza di Senso Comune}{
	La conoscenza di senso comune (commonsense knowledge) è molto
	importante per i task che comportano l’interazione con le persone.
}

\nt{Per esempio per chatbot e robot.}

\dfn{CYC}{
	CYC: enCYClopedic Knowledge, conoscenza enciclopedica. Base di conoscenza finalizzata a rappresentare tutto. Comprende oltre 200.000 concetti.
}

\nt{OpenCyc è la versione open source (quindi superiore), ma non è più disponibile direttamente.}

\clm{}{}{
	\begin{itemize}
		\item La base di conoscenza (Knowledge Base, KB) di CYC
		      consiste di:
		      \begin{itemize}
			      \item \fancyglitter{Termini}, il vocabolario di CYC.
			      \item \fancyglitter{Asserzioni} che mettono in relazione questi termini.
		      \end{itemize}
		\item Queste asserzioni includono:
		      \begin{itemize}
			      \item Asserzioni semplici.
			      \item \fancyglitter{Regole}.
		      \end{itemize}
	\end{itemize}
}

\paragraph{Struttura di CYC:}

\begin{itemize}
	\item La KB di Cyc è divisa in molte \fancyglitter{microteorie} (microtheories),
	      ciascuna delle quali è costituita da un insieme di asserzioni che
	      condividono le stesse assunzioni.
	\item Le microteorie si focalizzano su un particolare dominio di conoscenza.
	\item Il ragionamento avviene all’interno della singola microteoria.
	\item Questa suddivisione permette al sistema di fare asserzioni che
	      sarebbero apparentemente contraddittorie.
\end{itemize}

\begin{figure}[h]
	\centering
	\includegraphics[scale=0.5]{02/rule.png}
	\caption{Esempi di regole.}
\end{figure}

\qs{}{Come ragiona CYC?}

\begin{itemize}
	\item Il \fancyglitter{motore inferenziale} di CYC è in grado di effettuare la
	      deduzione logica (modus ponens, modus tollens,
	      quantificazione universale e esistenziale) e i
	      meccanismi di inferenza tipici dell’IA (ereditarietà,
	      classificazione automatica).
	\item La dimensione della base di conoscenza (200.000
	      termini e dozzine di asserzioni riguardanti ogni termine)
	      hanno richiesto la messa a punto di tecniche speciali
	      per affrontare la complessità.
\end{itemize}

\paragraph{Applicazioni di CYC:}

\begin{itemize}
	\item Modellazione della conoscenza.
	\item Apprendimento e pattern recognition.
	\item Assistenti intelligenti.
	\item Sicurezza delle reti.
	\item Basi di dati.
\end{itemize}

\dfn{SUMO}{
	Ontologia non più mantenuta perché incorporata in altri progetti.

	SUMO (Suggested Upper Merged Ontology) è un'ontologia di alto livello che incorpora un insieme di modelli. È scritta in KIF (Knowledge Interchange Format).
}

\paragraph{SUMO contiene:}

\begin{itemize}
	\item \fancyglitter{Termini}: indivisui e classi, con relazioni gerarchiche (instance e subclass).
	\item \fancyglitter{Connettivi AND e OR}.
	\item \fancyglitter{Quantificazione e implicazione}.
\end{itemize}

\clm{}{}{
	\begin{itemize}
		\item Di ogni classe SUMO descrive le caratteristiche attraverso un insieme di assiomi.
		\item Tali assiomi sono espressi utilizzando le relazioni
		      contenute nell’ontologia.
	\end{itemize}
}

\begin{figure}[h]
	\centering
	\includegraphics[scale=0.5]{02/sumo class.png}
	\caption{Esempio di classe.}
\end{figure}

\nt{Si possono descrivere relazioni di parti.}

\begin{figure}[h]
	\centering
	\includegraphics[scale=0.5]{02/partof.png}
	\caption{Esempio di relazione part-of.}
\end{figure}

\nt{Il browser per consultare SUMO si chiama \fancyglitter{Sigma}. No, non metterò un meme brainrot su Sigma.}

\subsection{Altre Ontologie}

\dfn{Ontologie Lightweight}{
	Le ontologie leggere sono, normalmente, semplici tassonomie, senza assioni e con poche relazioni. Facilmente standardizzabili.
}

\cor{WordNet}{
	Rete di concetti rappresentati da insiemi di termini (detti synset).
	Relazione di sussunzione tra concetti (synset).
}

\dfn{Ontologie Large-Scale}{
	Sono ontologie di grandi dimensioni:
	\begin{itemize}
		\item Possono essere ottenute tramite l’estrazione automatica di
		      concetti da testi (Dbpedia e YAGO).
		\item Dall’integrazione di risorse (YAGO e YAGO2) diverse, incluse
		      risorse lessicali.
		\item Dal lavoro di una comunità di utenti (CYC), via crowd sourcing.
	\end{itemize}
}

\nt{Date le dimensioni, gli strumenti di indicizzazione e di
	accesso acquisiscono grande importanza: spesso vengono pubblicate online e come tali integrate in altre
	applicazioni.}

\clm{}{}{
	\begin{itemize}
		\item Per l’accesso ai concetti dell’ontologia, è importante l’integrazione
		      tra di essi e il linguaggio naturale.
		\item L’integrazione avviene:
		      \begin{itemize}
			      \item Attraverso la documentazione, cioè associando a ogni concetto
			            una sua descrizione informale in linguaggio naturale.
			      \item Associando ai concetti una entry lessicale in una risorsa
			            linguistica esterna.
		      \end{itemize}
	\end{itemize}
}

\dfn{DBPedia}{
	Iniziativa di ricerca iniziata nel 2007. DBPedia punta a estrarre contenuti strutturali dal progetto wikipedia. DBPedia consente agli utenti di estrarre relazioni e proprietà associate a risorse di wikipedia.
}

\dfn{Linked Open Data}{
	Dati pubblicamente disponibili (Open), pubblicati secondo il paradigma
	dei Linked Data. I dataset risiedono nella rete, formando il Linked Open Data (LOD) Cloud.
}

\begin{figure}[h]
	\centering
	\includegraphics[scale=0.5]{02/partof.png}
	\caption{Linguaggi per descrivere ontologie.}
\end{figure}

\subsection{Relazioni tra Classi}

\dfn{Sottoclasse}{
	Tutti gli elementi della sottoclasse
	sono elementi della (sovra)classe,
	ma non viceversa.
}

\cor{Classi Disgiunte}{
	Due classi sono disgiunte se non hanno
	elementi in comune.
}

\begin{figure}[h]
	\centering
	\includegraphics[scale=0.6]{02/classdisg.png}
	\caption{Esempio di classi disgiunte.}
\end{figure}

\cor{Scomposizione Esaustiva}{
	Due o più classi sono scomposizione esaustiva
	di una classe se tutti gli elementi della classe
	appartengono a una di esse
}

\nt{\{Statunitensi, Canadesi, Messicani\} sono la
	scomposizione esaustiva di Nordamericani\footnote{Trump non approva questo elemento, è palese che il Canada sia il 51°esimo stato}.}

\cor{Partizione}{
	Scomposizione esaustiva con classi disgiunte.
}

\begin{figure}[h]
	\centering
	\includegraphics[scale=0.6]{02/partizioni.png}
	\caption{Esempio di partizione.}
\end{figure}

\dfn{Logiche Descrittive}{
	Le logiche descrittive sono:
	\begin{itemize}
		\item Orientate alla classificazione.
		\item Basate sulla relazione di sottoclasse (sussunzione).
		\item Completezza e trattabilità computazionale.
		\item Sono la base del Progetto Web Semantico.
	\end{itemize}
}

\clm{}{}{
	\begin{itemize}
		\item Nelle logiche descrittive si distinguono le definizioni di concetti
		      dalle asserzioni sugli individui fatte utilizzando quei concetti:
		      \begin{itemize}
			      \item \fancyglitter{T-Box}: Terminologia, cioè definizione di concetti generali.
			      \item \fancyglitter{A-Box}: Asserzioni su singoli individui.
		      \end{itemize}
	\end{itemize}
}

\dfn{Ruoli}{
	\begin{itemize}
		\item I ruoli costituiscono il mezzo per
		      mettere in relazione i concetti.
		\item In CLASSIC, il predicato che
		      permette di esprimere il concetto
		      di ruolo è FILLS.
		\item Tipicamente, si possono porre
		      restrizioni numeriche sui ruoli
		      (Atleast, Atmost).
		\item Un motore inferenziale (reasoner)
		      usa queste informazioni per
		      effettuare ragionamenti.
	\end{itemize}
}

\paragraph{Terminologia vs. Asserzioni:}

\begin{itemize}
	\item La terminologia è un insieme di assiomi logici su classi
	      e proprietà:
	      \begin{itemize}
		      \item Sussunzione tra concetti (subclass).
		      \item Relazioni generiche in cui gli elementi di determinate classi
		            rivestono un ruolo.
	      \end{itemize}
	\item Data la terminologia, si fanno asserzioni su un insieme
	      di individui.
	      \begin{itemize}
		      \item Le asserzioni devono essere coerenti con la terminologia (non
		            si può dire che uno scapolo è sposato).
	      \end{itemize}
\end{itemize}

\qs{}{
	Cosa si chiede a un sistema basato su DL?
}

\begin{itemize}
	\item \fancyglitter{Instance checking}: verificare se un
	      certo individuo (nella A-Box)
	      appartiene a una classe.
	\item \fancyglitter{Relation checking}: verificare se vale
	      una certa relazione tra classi.
	\item \fancyglitter{Subsumption}: verificare se una classe
	      è un sottoinsieme di un'altra classe.
	\item \fancyglitter{Concept consistency}: verificare che le
	      definizioni e le loro conseguenze non
	      siano contraddittorie.
\end{itemize}

\section{Il Linguaggio OWL}

\dfn{OWL2}{
	OWL 2 (Web Ontology Language) è un linguaggio per creare
	ontologie per il Web Semantico con un significato definito
	formalmente.
}

\clm{}{}{
	\begin{itemize}
		\item Le ontologie scritte in OWL 2 contengono classi, proprietà,
		      individui, e letterali; sono scritte in documenti il cui
		      formato è definito dal Web Semantico.
		\item Le ontologie OWL 2 possono essere utilizzate in
		      associazione con documenti RDF, anzi sono essere stesse
		      normalmente codificate come documenti RDF.
	\end{itemize}
}

\paragraph{Ragionamento automatico:}

\begin{itemize}
	\item OWL si basa su logiche computazionali tali che la
	      conoscenza espressa nell’ontologia può essere oggetto
	      di ragionamento automatico da parte di \fancyglitter{software
		      specifici (reasoner)} che ne verificano la non
	      contradditorietà e sono in grado di rendere esplicita la
	      conoscenza implicita contenuta nell’ontologia.
	\item La sintassi RDF/XML per OWL è normata da una
	      specifica: “OWL 2 RDF Mapping”. Ci sono molte altre
	      sintassi ma questa è l’unica che qualsiasi strumento
	      software per OWL deve essere in grado di gestire.
\end{itemize}

\paragraph{Struttura di un documento OWL:}

\begin{itemize}
	\item OWL 2 non fornisce strumenti che descrivano
	      in maniera prescrittiva la struttura di un
	      documento OWL.
	\item In particolare, non c’è modo per specificare
	      che una determinata informazione (per
	      esempio, il codice fiscale di una persona) deve
	      essere necessariamente presente.
\end{itemize}

\paragraph{Elementi dell'ontologia:}

\begin{itemize}
	\item \fancyglitter{Entità}: gli elementi che si riferiscono al mondo reale.
	\item \fancyglitter{Assiomi}: le asserzioni generali contenute nell’ontologia
	      (per esempio, il concetto di persona è un concetto più
	      specifico di quello di essere vivente).
	\item \fancyglitter{Espressioni}: combinazioni di entità che vanno a
	      formare entità più complesse.
\end{itemize}

\paragraph{Tipi di entità:}

\begin{itemize}
	\item Gli oggetti sono individui.
	\item Le categorie sono le classi.
	\item Le relazioni sono le proprietà.
\end{itemize}

\paragraph{Le proprietà sono suddivise in:}

\begin{itemize}
	\item Object properties: che collegano individui ad individui (come una
	      persona al suo coniuge).
	\item Datatype properties: che assegnano un dato a un individuo (per
	      esempio l’età a una persona).
	\item Annotation properties: che contengono commenti e descrizioni sulle
	      entità.
\end{itemize}

\dfn{Assiomi}{
	Gli assiomi, in OWL2, possono essere dichiarazioni, assiomi sulle classe, sugli oggetti o sulle proprietà dei dati, definizioni di tipi di dati, chiavi, asserzioni (anche chiamate fatti) e assiomi su annotazioni.
}

\begin{figure}[h]
	\centering
	\includegraphics[scale=0.8]{02/axioms.png}
	\caption{Assiomi.}
\end{figure}

\nt{In OWL2 ogni IRI deve essere dichiarato nell'ontologia.}

\paragraph{Ci sono 2 tipi di dichiarazioni:}

\begin{itemize}
	\item Che un certo IRI fa parte dell'ontologia.
	\item A che tipo di entità appartiene l'IRI:
	      \begin{itemize}
		      \item Class.
		      \item Datatype.
		      \item Object property.
		      \item Data property.
		      \item Annotation property.
		      \item An individual.
	      \end{itemize}
\end{itemize}

\paragraph{Tipi di relazioni:}

\begin{itemize}
	\item \fancyglitter{SubClassOf}: ogni istanza di una classe è anche un'istanza di un'altra classe\footnote{Costruisce una gerarchia di classi.}
	\item \fancyglitter{EquivalentClasses}: diverse classi sono equivalenti tra di loro.
	\item \fancyglitter{DisjointClasses}: classi che non hanno nessuna istanza in comune.
	\item \fancyglitter{DisjointUnion}: una scomposizione esaustiva.
\end{itemize}

\ex{}{
	\begin{itemize}
		\item Sottoclasse:
		      \begin{itemize}
			      \item SubClassOf(a:Boy a:Child).
			      \item Boy è sottoclasse di Child.
		      \end{itemize}
		\item Classi equivalenti:
		      \begin{itemize}
			      \item EquivalentClasses(a:CatOwner a:PadroneDiGatti).
			      \item Le due classi sono entrambe sottoclasse l'una dell'altra.
		      \end{itemize}
		\item Disgiunzione:
		      \begin{itemize}
			      \item DisjointClasses(a:Cat a:Dog).
			      \item Cat e Dog sono classi disgiunte.
		      \end{itemize}
		\item Disjoint Union:
		      \begin{itemize}
			      \item DisjointUnion(a: Person a:Child a:Adult).
			      \item Person ha come sottoclassi le due classi disgiunte Child e Adult.
		      \end{itemize}
	\end{itemize}
}

\nt{In questi esempi è stata usata la \fancyglitter{functional style syntax}.}

\begin{figure}[h]
	\centering
	\includegraphics[scale=0.8]{02/cl.png}
	\caption{ClassAxiom}
\end{figure}

\dfn{Espressioni}{
	In OWL2, classi e proprietà sono usate per costruire \newfancyglitter{espressioni di classe} (o descrizioni).
}

\nt{Per esempio "Professoressa" può essere visto come l'unione di "Docente" e "Donna".}

\paragraph{OWL2 fornisce un insieme di operatori per definire le classi:}

\begin{itemize}
	\item \fancyglitter{Connettivi booleani}.
	\item \fancyglitter{Quantificazione universale ed esistenziale}.
	\item \fancyglitter{Restrizioni numeriche}.
	\item \fancyglitter{Enumerazione di individui}.
\end{itemize}

\subsection{Definizioni e Sintassi}

\begin{itemize}
	\item Si definisce una classe dichiarando che essa
	      appartiene al tipo Classe di OWL.
	      \begin{center}
		      \textcolor{blue}{:Person} \textcolor{red}{rdf:type owl:Class};
	      \end{center}
	\item Mary appartiene alla classe Person.
	      \begin{center}
		      \textcolor{blue}{:Mary} \textcolor{red}{rdf:type} \textcolor{blue}{:Person} .
	      \end{center}
	\item John appartiene alla classe Father.
	      \begin{center}
		      \textcolor{blue}{:John} \textcolor{red}{rdf:type} \textcolor{blue}{:Father} .
	      \end{center}
	\item Assiomi di sottoclasse.
	      \begin{center}
		      \textcolor{blue}{:Woman} \textcolor{red}{rdfs:subClassOf} \textcolor{blue}{:Person} .
	      \end{center}
	\item Due classi possono essere dichiarate equivalenti.
	      \begin{center}
		      \textcolor{blue}{:Person} \textcolor{red}{owl:equivalentClass} \textcolor{blue}{:Human} .
	      \end{center}
	\item Un insieme di classi possono essere dichiarate
	      disgiunte.
	\item Si definisce una Object Property asserendo
	      che essa appartiene al tipo ObjectProperty di
	      OWL.
	      \begin{center}
		      \textcolor{blue}{:hasSpouse} \textcolor{red}{rdf:type owl:ObjectProperty} ;
	      \end{center}
	\item Le proprietà di tipo Object Property rappresentano relazioni
	      tra classi (e quindi, si asseriscono degli individui).
	      \begin{center}
		      \textcolor{blue}{:John} \textcolor{red}{:hasWife} \textcolor{blue}{:Mary}
	      \end{center}
	\item Le data property hanno come domain una classe e come range un tipo
	      di dato.
	      \begin{center}
		      \textcolor{blue}{:hasAge} \textcolor{red}{rdf:type owl:DatatypeProperty} ;
	      \end{center}
\end{itemize}

\dfn{Restrizioni}{
	Le restrizioni sono uno dei meccanismi principali
	per definire nuove classi a partire da quelle
	esistenti.
}

\paragraph{Esistono due tipi di restrizioni:}

\begin{itemize}
	\item Restrizioni su \fancyglitter{classi} mediante operatori insiemistici
	      (intersezione-and, unione-or, complemento-not).
	\item Restrizioni poste su \fancyglitter{proprietà} (esistenziale, universale,
	      sulla cardinalità).
\end{itemize}

\dfn{Intersezione}{
	L’intersezione permette di definire una classe come
	intersezione di due classi.
}

\dfn{Complemento}{
	Una classe può essere definita come complemento di un’altra.
}

\paragraph{Necessario e Sufficiente:}

\begin{itemize}
	\item Le classi definite come
	      equivalenti a un certo
	      insieme di restrizioni sono denominate classi definite.
	\item Le restrizioni individuano le condizioni
	      necessarie e sufficienti per
	      l’appartenenza alla classe.
	\item Senza l’utilizzo del costrutto
	      EquivalentTo si possono associare a una
	      classe solo proprietà necessarie ma non
	      sufficienti.
\end{itemize}

\nt{Solo le classi definite permettono determinate forme di
	ragionamento.}

\dfn{Classi Enumerate}{
	È possibile definire una classe come enumerazione
	di individui.
}

\paragraph{Descrivere le proprietà:}

\begin{itemize}
	\item Simmetriche.
	\item Funzionali.
	\item Inverse.
	\item Riflessive.
	\item Transitive.
\end{itemize}

\clm{Modellazione}{}{
	\begin{itemize}
		\item Non confondere la relazione di sottoclasse con la
		      relazione mereologica part-of (un motore non è un'automobile).
		\item L'ontologià deve essere leggibile:
		      \begin{itemize}
			      \item Gerarchica.
			      \item Attribuire alle entità nomi che hanno senso secondo il
			            senso comune o per l’esperto.
			      \item Associare domain e range alle proprietà.
		      \end{itemize}
	\end{itemize}
}






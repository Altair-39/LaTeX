\chapter{Dalle Ontologie ai Grafi}

\section{SPARQL}

\subsection{Introduzione}

\dfn{SPARQL}{
	SPARQL è un linguaggio di interrogazione per RDF. Permette di creare nuovi grafi con le triple esatte.
}

\paragraph{Le query possono essere diverse:}

\begin{itemize}
	\item Selezione di triple secondo un semplice pattern.
	\item Query complesse costruite con filtri, aggregatori, path expressions, ecc.
	\item Query booleane (ASK).
\end{itemize}

\paragraph{Risultati di SPARQL:}

\begin{itemize}
	\item Il risultato di una query può essere un insieme di triple o uno o più grafi RDF.
	\item SPARQL supporta i formati più comuni: XML, CSV, TSV, JSON.
\end{itemize}

\subsection{Query}

\cor{SPARQL Endpoint}{
	Le query vengono indirizzate a un indirizzo HTTP che ospita un endpoint SPARQL. L'endpoint esegue le query su uno o più dataset contenuti in uno store di triple.
}

\nt{
	Molte Linked Data Platforms (LDP) hanno un'interfaccia in cui è possibile inserire manualmente le query.
}

\paragraph{WHERE:}

\begin{itemize}
	\item Nella clausola WHERE si trovano una o
	      più triple che contengono variabili ?v
	\item Le triple formano un pattern che corrisponde a un insieme di grafi.
	\item Il pattern viene unificato con le tripl presenti nel repository rdf.
	\item Il risultato nella query è dato da tutte
	      le occorrenze delle variabili contenute
	      nella clausola WHERE (proiezione) nei
	      grafi trovati.
\end{itemize}

\cor{Filtri}{
	I filtri sono elementi opzionali che si possono includere nella clausola
	WHERE per selezionare un sottoinsieme nell’insieme dei risultati.
}

\paragraph{Operazioni sui risultati:}

\begin{itemize}
	\item OrderBy: ordina i risultati delle query sulla base di una delle variabili.
	\item Distinct.
	\item Group: estrae e conta gli individui che hanno determinati requisiti.
	\item Blank node: seleziona tutte le entità situate da qualche parte.
	\item Count.
\end{itemize}


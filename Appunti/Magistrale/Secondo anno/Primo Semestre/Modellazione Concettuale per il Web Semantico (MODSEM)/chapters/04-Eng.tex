\chapter{Ontology Engineering}

\section{Introduzione}

\dfn{Ontology Engineering}{
	Criteri e linee guida per progettare ontologie interoperabili e ben fondate sul
	piano concettuale. Criteri e linee guida per progettare ontologie interoperabili e ben fondate sul
	piano concettuale. Approcci collaborativi allo sviluppo di ontologie.
}

\subsection{OntoClean}

\dfn{OntoClean}{
	Metodo ontologico basato sull'analisi delle tassonomie delle classi.
}

\nt{
	Non è prescrittiva rispetto alle entità del
	mondo reale, ma fornisce proprietà con cui
	caratterizzare le classi.
}

\paragraph{Meta-Proprietà in OntoClean:}

\begin{itemize}
	\item \fancyglitter{Identity:} proprietà intrinseca che identifica un tipo di oggetti.
	      \begin{itemize}
		      \item Sortal: classe di oggetti che possono essere identificati nello stesso modo.
		      \item Viene ereditata dalle sottoclassi.
	      \end{itemize}
	\item \fancyglitter{Unity:} proprietà di un tipo di oggetto di essere unitario.
	      \begin{itemize}
		      \item Whole: classe di oggetti che sono tali solo in quanto costituiti da parti
		            collegate tra loro da relazioni strutturali.
		      \item  Viene ereditata dalle sottoclassi.
	      \end{itemize}
	\item \fancyglitter{Rigidity:} proprietà di un tipo di oggetto che non è soggetta a cambiamenti.
	      \begin{itemize}
		      \item Solo l'antirigidità viene ereditata.
	      \end{itemize}
	\item \fancyglitter{Dependence:} proprietà di un tipo di oggetti di dipendere da un altro per la
	      propria definizione.
\end{itemize}

\dfn{Phase Sortals}{
	Proprietà che attraversa una serie di fasi. Sono:
	\begin{itemize}
		\item Indipendenti.
		\item Antirigidi.
		\item Hanno un'identità.
	\end{itemize}

}

\subsection{Neon Methodology}

\dfn{Neon Methodology}{
	Metodologia orientata agli aspetti collaborativi nello sviluppo e nel
	mantenimento di network di ontologie.  Si articola in un insieme di 9 scenari, associati a specifiche attività e
	documenti.
}

\paragraph{Scenari:}

\begin{enumerate}
	\item Dalle specifiche all'implementazione.
	\item Riusare e reingegnerizzare risorse non ontologiche.
	\item Riusare risorse ontologiche.
	\item Riusare e reingegnerizzare risorse ontologiche.
	\item Riusare e unire risorse ontologiche.
	\item Riusare, unire e reingegnerizzare risorse ontologiche.
	\item Riusare design pattern ontologici (ODPs).
	\item Ristrutturare risorse ontologiche.
	\item Localizzare risorse ontologiche.
\end{enumerate}

\dfn{Ontology Design Patterns (ODPs)}{
	Sorta di mattoncini per la creazione di ontologie secondo schemi
	(pattern) convidisi.
}

\subsection{Altre Ontologie}

\dfn{DOLCE}{
	Ontologia orientata alla cognizione e al linguaggio.
}

\clm{Concetti Importanti}{}{
	\begin{itemize}
		\item \fancyglitter{Perdurante} $\rightarrow$ Evento, ha natura temporale.
		\item \fancyglitter{Endurant} $\rightarrow$ Partecipanti, non ha natura temporale.
	\end{itemize}
}

\dfn{Prov}{
	Concepita per l’interscambio sul Web di informazioni riguardanti la
	provenienza delle entità. Descrive l’origine delle entità intesa soprattutto come processi che hanno
	determinato la creazione di quelle entità:
	\begin{itemize}
		\item La provenienza degli oggetti è
		      rilevante per determinarne il
		      possesso (inteso come diritti) e
		      l’affidabilità.
	\end{itemize}
}

\paragraph{I metadati propriamente detti delle
	entità e la descrizione degli agenti
	sono affidati a altre ontologie:}

\begin{itemize}
	\item FOAF (agenti).
	\item Dublin Core (entità).
\end{itemize}

\paragraph{In Prov:}

\begin{itemize}
	\item Un \fancyglitter{agente} prende il ruolo in un'attività in modo che gli si possa assegnare un certo grado di responsabilità.
	\item \fancyglitter{Attività:} sono il come le entità esistono e come  i loro attributi cambiano per farle diventare nuove entità.
	\item Un \fancyglitter{ruolo} è una descrizione della funzione di un'entità in un'attività. Specificano la loro relazione.
\end{itemize}

\paragraph{Problematiche:}

\begin{itemize}
	\item Come rappresentare i ruoli?
	      \begin{itemize}
		      \item Classi dell'ontologia (sottoclassi di Ruolo).
		      \item Classe generica con etichette linguistiche.
	      \end{itemize}
	\item Classi dell'ontologia:
	      \begin{itemize}
		      \item Vantaggi: ragionamento.
		      \item Svantaggi: non aperto, non modulare.
	      \end{itemize}
	\item Classe generica:
	      \begin{itemize}
		      \item Vantaggi: modulare.
		      \item Svantaggi: più difficile condurre inferenze.
	      \end{itemize}
\end{itemize}

\dfn{LODE}{
	L’ontologia LODE (Linking Open Descriptions
	of Events) rappresenta gli eventi (per esempio
	eventi storici) con un vocabolario molto semplice.
}

\clm{LODE}{}{
	\begin{itemize}
		\item Non rappresenta i ruoli ma solo i partecipanti
		      e la collocazione nel tempo e nello spazio
		      degli eventi.
		\item Orientata alla costruzione di grandi data set
		      (descrizione di eventi).
		\item Per ogni classe, corrispondenza con DOLCE.
	\end{itemize}
}

\section{Ontologie e Risorse Linguistiche}

\nt{Approfondita nella seconda parte del corso "Tecnologie del Linguaggio Naturale".}

\subsection{FrameNet}

\dfn{FrameNet}{
	Si tratta di un catalogo di frame linguistici su varie parti del discorso che rappresentano un frame con una serie di ruoli (complementi) associati.
}

\nt{
	La struttura a frame di FrameNet si presta a essere utilizzata insieme allo
	schema a ruoli (utilizzato per esempio negli Ontology Design Pattern) per
	rappresentare la partecipazione all’azione. Ogni Frame Element corrisponde a un ruolo nello schema di azione.
}

\paragraph{Descrizione di Eventi e FrameNet:}

\begin{itemize}
	\item Repository di frame che descrivono una situazione in
	      termini di partecipanti con un certo ruolo.
	\item Ogni  frame corrisponde a un insieme di entità lessicali
	      (tipicamente verbi ma anche preposizioni).
	\item Non esiste un insieme predefinito di ruoli (come per
	      esempio in VerbNet.
	\item Conversione diretta in un pattern basato su ruoli.
\end{itemize}

\subsection{WordNet}

\dfn{WordNet}{
	WordNet è una risorsa linguistica che consiste in un lessico
	organizzato secondo relazioni di significato. Gli elementi del lessico (sostantivi, verbi, aggettivi, ecc.) sono
	raggruppati in insiemi denominati synset (synonym set):
	\begin{itemize}
		\item Ogni synset corrisponde a un significato.
		\item L’idea è che i termini che fanno parte di uno stesso synset siano
		      sinonimi.
	\end{itemize}
}

\cor{Synset}{
	Nello stesso synset si trovano più termini. Lo stesso sostantivo può essere collocato in posizioni del
	tutto diverse della rete.
}

\paragraph{Altre relazioni:}

\begin{itemize}
	\item Meronimia: per i sostantivi, oggetto che è parte di un altro.
	\item Implicazione: per i verbi.
	\item Troponimia: per i verbi.
\end{itemize}

\paragraph{WordNet e ontologie:}

\begin{itemize}
	\item  Nell’ontologia SUMO le classi sono associate ai synset di WordNet.
	\item Sigma Browser è uno strumento per cercare concetti nell’ontologia
	      SUMO usando come chiavi di ricerca i termini della lingua inglese.
	\item Il browser funziona perché a ogni concetto di SUMO corrispondono
	      uno o più termini di WordNet.
	\item Cercando in WordNet, si trovano tutti i significati della parola
	      cercata e, per ogni significato, il concetto di SUMO
	      corrispondente.
\end{itemize}

\paragraph{Ci sono due modi in cui un synset di WordNet può
	corrispondere a un concetto SUMO:}

\begin{itemize}
	\item Il concetto SUMO corrisponde esattamente al significato del
	      synset (equivalent mapping).
	\item Il concetto SUMO include al suo interno (‘sussume’) anche il
	      significato del synset (subsumption mapping).
\end{itemize}

\section{SKOS}

\subsection{Contesto}

Con l’avvento dei Linked Data, fanno il loro ingresso nel
Web Semantico molte risorse di tipo tassonomico.

\paragraph{Esempi di tesauri e tassonomie:}

\begin{itemize}
	\item Iconclass.
	\item VIAF.
	\item ULAN.
	\item ACM Computing Classification System.
\end{itemize}

\dfn{VIAF}{
	VIAF (Virtual International Authority File) è un
	Authority file internazionale:
	\begin{itemize}
		\item Ottenuto a partire dagli Authority file nazionali.
		\item Permette di accedere a una determinata entità dal suo
		      nome (in una lingua nazionale) per ottenerne la
		      denominazione in tutti gli authority in cui è presente.
	\end{itemize}
}

\cor{Authority File}{
	Un authority file è una risorsa che contiene un elenco
	strutturato di valori da utilizzare per riferire un insieme
	di entità.
}

\subsection{Simple Knowledge Organization System (SKOS)}

\dfn{SKOS}{
	L’utilizzo di vocabolari di diverse provenienze, tipico dei
	Linked Data, annulla le relazioni tra i concetti proprio delle
	ontologie e si apre alle differenze. Simple Knowledge Organization Systems (SKOS):
	recommendation del W3C per la creazione di vocabolari di
	vario tipo: lessici, thesauri, ecc.
}

\nt{Il suo scopo è quello di facilitare e uniformare la pubblicazione dei vocabolari RDF come linked data.}

\qs{}{Cosa si può fare con SKOS?}

\begin{itemize}
	\item I concetti possono essere identificati mediante URI.
	\item Etichettati con stringhe in una o più lingue naturali.
	\item Assegnare notazioni.
	\item Documentare con vario tipo di note.
	\item Collegare concetti ad altri concetti e organizzarli in gerarchie informali.
	\item Aggregare i concetti in schemi, ordinarli, etichettarli o mapparli.
\end{itemize}

\cor{SKOS Primer}{
	Documento che riporta esempi di utilizzo di SKOS.
}

\paragraph{Concetti ed etichette:}

\begin{itemize}
	\item I concetti sono identificati con gli URI.
	\item Ai concetti sono associate delle etichette lessicali:
	      \begin{itemize}
		      \item prefLabel.
		      \item altLabel.
	      \end{itemize}
	\item Indicizzazione.
\end{itemize}

\paragraph{Relazioni semantiche tra concetti:}

\begin{itemize}
	\item Relazione generica: related.
	\item Relazione specifica/generale: narrowed/broader.
	\item Anche transitive.
\end{itemize}

\paragraph{Documentazione:}

\begin{itemize}
	\item Associare concetti alla loro documentazione.
	\item Definizione: definition.
	\item Esempio: example.
	\item Cambiamenti: historyNote e changeNote.
\end{itemize}

\paragraph{Schemi di concetti:}

\begin{itemize}
	\item Uno schema di concetti e una risorsa strutturata.
	\item Schema di concetti: ConceptScheme.
	\item Appartenenza di un concetto a uno schema: inScheme.
	\item (Ri)utilizzare concetti in uno schema diverso richiamando
	      quello di origine.
\end{itemize}

\chapter{Ontology Engineering}

\section{Introduzione}

\dfn{Ontology Engineering}{
	Criteri e linee guida per progettare ontologie interoperabili e ben fondate sul
	piano concettuale. Criteri e linee guida per progettare ontologie interoperabili e ben fondate sul
	piano concettuale. Approcci collaborativi allo sviluppo di ontologie.
}

\subsection{OntoClean}

\dfn{OntoClean}{
	Metodo ontologico basato sull'analisi delle tassonomie delle classi.
}

\nt{
	Non è prescrittiva rispetto alle entità del
	mondo reale, ma fornisce proprietà con cui
	caratterizzare le classi.
}

\paragraph{Meta-Proprietà in OntoClean:}

\begin{itemize}
	\item \fancyglitter{Identity:} proprietà intrinseca che identifica un tipo di oggetti.
	      \begin{itemize}
		      \item Sortal: classe di oggetti che possono essere identificati nello stesso modo.
		      \item Viene ereditata dalle sottoclassi.
	      \end{itemize}
	\item \fancyglitter{Unity:} proprietà di un tipo di oggetto di essere unitario.
	      \begin{itemize}
		      \item Whole: classe di oggetti che sono tali solo in quanto costituiti da parti
		            collegate tra loro da relazioni strutturali.
		      \item  Viene ereditata dalle sottoclassi.
	      \end{itemize}
	\item \fancyglitter{Rigidity:} proprietà di un tipo di oggetto che non è soggetta a cambiamenti.
	      \begin{itemize}
		      \item Solo l'antirigidità viene ereditata.
	      \end{itemize}
	\item \fancyglitter{Dependence:} proprietà di un tipo di oggetti di dipendere da un altro per la
	      propria definizione.
\end{itemize}

\dfn{Phase Sortals}{
	Proprietà che attraversa una serie di fasi. Sono:
	\begin{itemize}
		\item Indipendenti.
		\item Antirigidi.
		\item Hanno un'identità.
	\end{itemize}

}













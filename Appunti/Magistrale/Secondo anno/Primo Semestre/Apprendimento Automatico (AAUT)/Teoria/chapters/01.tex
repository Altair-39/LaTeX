\chapter{Introduzione}

\section{Che Cos'è l'Apprendimento Automatico?}

\dfn{Machine Learning}{
	Un programma informatico apprende dall'esperienza $E$ rispetto a una classe di task $T$ e una performance $P$, se la sua performance nel task $T$, misurata da $P$, aumenta con l'esperienza $E$.
}

\paragraph{Sostanzialmente:}

\begin{itemize}
	\item L'esperienza viene data sotto forma di esempi "risolti" al computer.
	\item Un task (compito) da risolvere.
	\item Con un modo per valutare la risoluzione (performance).
\end{itemize}

\subsection{Terminologia}

\begin{figure}[h]
	\centering
	\includegraphics[scale=0.45]{01/terminology.png}
	\caption{Terminologia.}
\end{figure}

\begin{itemize}
	\item Attributi (Features): le colonne.
	\item Etichetta (Classe): elemento che indica come risolvere un task.
	\item Istanza (Sample): una riga.
	\item Valori: le celle.
	\item Set di Training: insieme su cui si va a dedurre una regola per classificare.
	\item Test Set: insieme per vedere quanto si sarà accurati su insiemi futuri.
\end{itemize}

\nt{Si usa un set diverso per il training e il test perché se si usasse lo stesso il modello farebbe risultati elevati essendo addestrato su quello.}

\begin{figure}[h]
	\centering
	\includegraphics[scale=0.5]{01/terminology2.png}
	\caption{Terminologia 2.}
\end{figure}

\paragraph{Immaginando gli oggetti in un qualche campo euclideo:}

\begin{itemize}
	\item \fancyglitter{Features vector:} ogni esempio corrisponde a un vettore.
	\item \fancyglitter{Attribute space:} l'insieme di tutti gli esempi.
\end{itemize}

\begin{figure}[h]
	\centering
	\includegraphics[scale=0.5]{01/table.png}
	\caption{Tabella di riferimento.}
\end{figure}

\dfn{Learning (Training)}{
	Il learning è un processo in cui si usano algoritmi di apprendimento automatico per costruire dei modelli.
	\begin{itemize}
		\item I dati utilizzati in questo processo sono detti training data.
		\item Ogni istanza è un training example.
		\item L'insieme di tutti i training example è il training set.
	\end{itemize}
}

\nt{Un modello addestrato corrisponde a una serie di regole sui dati, quindi si chiama anche \fancyglitter{ipotesi} e le regole sono i \fancyglitter{fatti} (grounded-truth).}

\subsection{Tasks}

\dfn{Tasks predittivi}{
	Un task predittivo è focalizzato sul prevedere una variabile sulla base degli esempi. Si parte da problemi vecchi per trovare la soluzione a nuovi problemi.
}

\paragraph{I tasks predittivi possono essere:}

\begin{itemize}
	\item \fancyglitter{Binari e Multi-classe:} di categorizzazione.
	\item \fancyglitter{Regressivi:} con un target numerico.
	\item \fancyglitter{Clustering:} un target sconosciuto.
\end{itemize}

\dfn{Tasks descrittivi}{
	Un task descrittivo si concentra sul fornire regolarità nel dataset.
}

\begin{figure}[h]
	\centering
	\includegraphics[scale=0.5]{01/tasks.png}
	\caption{Vari tasks.}
\end{figure}

\paragraph{Assunzione:}

\begin{itemize}
	\item Si assume che i dati siano \fancyglitter{indipendenti}.
	\item Si assume che i dati siano \fancyglitter{identicamente distribuiti}.
\end{itemize}
\subsection{Spazio di Ipotesi}

Nei sistemi \fancyglitter{assiomatici} il processo di derivare un teorema da assiomi è detto \fancyglitter{deduzione}. Questo è un processo corretto se si assume che gli assiomi siano veri. L'\fancyglitter{induzione} è il processo opposto ed è il principio su cui si basa tutto l'apprendimento automatico.


\nt{ATTENZIONE: l'induzione come intesa in questo corso non è l'induzione matematica.}

\clm{Deduzione vs. Induzione}{}{
	\begin{itemize}
		\item La deduzione è valida: assumere le premesse vere garantisce che le conclusioni siano vere.
		\item L'induzione non è valida come forma di ragionamento: non garantisce che  la conclusione sia vera anche se tutte le osservazioni sono corrette.
	\end{itemize}
}

\dfn{Boolean Concept Learning}{
	L'obiettivo è quello di apprendere una funzione booleana $h: \mcX \mapsto \{0,1\}$
}

\nt{Siamo nel campo del \fancyglitter{symbolic concept learning}, studiato nel campo dellla \fancyglitter{inductive logic programming}.}

\dfn{Spazio di Ipotesi}{
	Lo spazio di Ipotesi è l'insieme di tutte le possibili ipotesi che possono essere imparate da un algoritmo di apprendimento.
}

\nt{Il Machine Learning è la ricerca attraverso lo spazio delle ipotesi per trovare l'insieme di tutte le ipotesi che sono consistenti con i training data e selezionare i migliori secondo un qualche criterio.}








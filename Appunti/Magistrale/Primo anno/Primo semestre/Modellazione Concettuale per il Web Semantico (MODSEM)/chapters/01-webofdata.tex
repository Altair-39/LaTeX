\chapter{Web of Data: I Linguaggi}

\section{Introduzione}

\subsection{Perché Studiare il Semantic Web e Linked Data?}

\begin{itemize}
  \item Gli approcci quantitativi non sono sufficienti per domini complessi: 
    \begin{itemize}
      \item Non è possibile apprendere il comportamento giusto per ogni contesto. 
      \item Reattività e ragionamento.
    \end{itemize}
  \item L'ambito della conoscenza è intrinsecamente basato su modelli: 
    \begin{itemize}
      \item Arte, media, tecnologie, etc. 
    \end{itemize}
  \item Interoperabilità dei dati: 
    \begin{itemize}
      \item Conoscenza esperta per stabilire i \fancyglitter{mapping}. 
      \item Utilizzo di standard.
    \end{itemize}
  \item Ruolo nel ragionamento in molte applicazioni: 
    \begin{itemize}
      \item Elaborazione del linguaggio naturale. 
      \item Question answering. 
      \item Chatbots.
    \end{itemize}
\end{itemize}

\subsection{Obiettivi del Corso}

\begin{itemize}
  \item Imparare a rappresentare un dominio di conoscenza con i linguaggi
del Web Semantico (RDF e OWL), che permettono di implementare
ontologie computazionali. 
\item Utilizzare strumenti di ragionamento automatico per realizzare
inferenze sulla conoscenza formalizzata nelle ontologie
computazionali. 
\item Interrogare basi di conoscenza in cui i dati sono rappresentati in un
formato semantico utilizzando il linguaggio SPARQL. 
\item Rendere interoperabili rappresentazioni diverse (ontologie, basi di
dati, fogli di calcolo) utilizzando strumenti di mapping.
\end{itemize}









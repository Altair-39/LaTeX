\chapter{Quantum Computing}

\section{Introduzione}

La rivoluzione scientifica che diede origine al corpo di conoscenze
che va sotto il nome di Fisica Quantistica ebbe luogo all’incirca nei
primi trenta/quaranta anni del XX secolo. Tra i protagonisti di questa rivoluzione, ricordiamo Max Planck,
Niels Bohr, Albert Einstein, Erwin Schr\"odinger, Louis de Broglie,
Werner Heisenberg, Paul Dirac, John von Neumann. Gli esperimenti condotti in quegli anni mostravano che, a livello
subatomico, la materia si comportava in modo radicalmente diverso
da come invece faceva a livello macroscopico, e gli scienziati
svilupparono opportuni formalismi matematici in grado di rendere
conto delle osservazioni sperimentali e delle nuove scoperte. A livello subatomico la materia manifesta proprietà e comportamenti
in molti casi bizzarri e controintuitivi, se confrontati con le proprietà
e i comportamenti che osserviamo a livello macroscopico (livello del
quale si occupa la fisica classica). 

\dfn{Fisica Quantistica}{
  La Fisica Quantistica è quella branca della fisica
che si occupa dello studio dei fenomeni fisici che hanno luogo
a livello atomico e subatomico.
}

\clm{}{}{
  \begin{itemize}
    \item A livello microscopico le entità atomiche e
subatomiche hanno un comportamento probabilistico, ciò significa
che il loro comportamento (per esempio la loro posizione in un certo
istante, o in che direzione e con che velocità si stanno muovendo) può
essere predetto solo con una legge probabilistica.
\item La formula matematica che riassume il comportamento probabilistico
di una entità atomica prende il nome di funzione d’onda, secondo la
teoria sviluppata da Erwin Schr\"odinger nel 1926.

  \end{itemize}
}

\dfn{Quantum Computing}{
  Col termine \newfancyglitter{Quantum Computing} si indica la disciplina all’intersezione tra Fisica Quantistica
e Informatica che studia come implementare algoritmi sfruttando
alcune caratteristiche manifestate dalla materia a livello subatomico.
}

\nt{Alcune di queste caratteristiche rendono le potenzialità della QC
molto promettenti, anche se dietro gli aspetti teorici si nascondono
difficoltà realizzative non ancora risolte in modo soddisfacente.}

\paragraph{Breve Recap:}

\begin{itemize}
  \item Nell’anno zero dell’era quantica, il 1900, Max Planck riesce
finalmente a modellare matematicamente il modo in cui i metalli,
riscaldati a diverse temperature, emettono una quantità ben precisa
di energia elettromagnetica. La soluzione di Planck
consiste nell’ipotizzare che
la radiazione elettromagnetica
(di cui la luce fa parte) venga
emessa dai metalli non in
modo continuo ma in
pacchetti di dimensione
prestabilita.
\item Nel 1905 Albert Einstein, sfruttando l’idea di Planck, spiega l’effetto
fotoelettrico ipotizzando una natura corpuscolare per la luce: le
particelle di luce (che verranno poi chiamate fotoni, o quanti di
luce), bombardano i metalli e strappano elettroni dalla loro superficie:
un fotone per ciascun elettrone.
\item Nel 1911 Ernest Rutheford, dopo aver scoperto che gli atomi sono
quasi completamente vuoti, ipotizza che assomiglino a microscopici
sistemi solari. 
\item Nel 1913 Niels Bohr, ricorrendo ancora all’idea dei pacchetti di
energia, spiega perché gli elettroni non precipitano sul nucleo fatto di
protoni: gli elettroni si muovono su orbite quantizzate, e possono spostarsi da
un’orbita all’altra solo assorbendo o emettendo un pacchetto di
energia (l’idea di “orbita” è ormai
superata, ma l’idea di stati quantici
rimane valida).
\item Negli stessi anni, lo studio del decadimento radioattivo e del
comportamento degli elettroni negli atomi mostra che atomi in
condizioni apparentemente identiche si comportano in modo diverso. Il loro comportamento può essere predetto solo facendo ricorso a
leggi probabilistiche. 
\item Intorno al 1922-1923, tutti i fisici sono più o meno convinti della
natura corpuscolare della luce: quando un fascio di luce attraversa un
gas, fotoni ed elettroni si scontrano fra di loro con un comportamento
simile a quello delle palle da bigliardo. 
\item Viene utilizzato un esperimento (delle due fenditure) per dimostrare che la luce possa essere contemporaneamente sia un'onda che un insieme di particelle.
\item Nel 1924, nella sua tesi di dottorato Louis de Broglie argomenta che
se le onde elettromagnetiche hanno anche una natura corpuscolare,
allora la materia può avere una natura ondulatoria, e mostra come
calcolare le lunghezze d’onda della materia. E infatti negli anni a venire l’esperimento a due fenditure sarà ripetuto
con elettroni, protoni, interi atomi, e infine con molecole complesse. 
\item La dicotomia classica che aveva regnato per centinaia di anni:
la realtà fisica fatta di materia e campi di forza, o equivalentemente,
particelle e onde, sembrava dissolversi in una unica “sostanza
quantica” che manifestava le proprietà di entrambe.
\begin{figure}[h]
    \centering
    \includegraphics[width=0.3\textwidth]{05-QuantumComputing/meme.jpg}
    \caption{"È solo A o B ahhh rhetoric".}
\end{figure}
\item Tra il 1925 e il 1926 Werner Heisenberg, Erwin Schrödinger e
Paul Dirac formulano in modo indipendente addirittura tre modelli
matematici equivalenti in grado di descrivere e prevedere
il comportamento di questa “sostanza quantica”.
\end{itemize}

\section{Il Problema della Misurazione}

 La funzione d’onda associata ad una entità atomica o subatomica,
nota anche come equazione di Schr\"odinger codifica / sintetizza /
permette di calcolare, con un’unica espressione matematica:
\begin{itemize}
  \item Tutti i valori che possono assumere gli attributi di quella entità. 
  \item La probabilità con cui ciascun valore tra quelli possibili potrà
manifestarsi durante una misurazione.
\end{itemize}

La formulazione appena data, apparentemente semplice, nasconde al
suo interno alcune peculiarità singolari, che i fisici non hanno ancora
saputo spiegare, pur riuscendo a modellarle correttamente con un
appropriato formalismo matematico. Non a caso, uno dei grandi fisici del novecento, Richard Feynman (tra
l’altro il primo a intuire le potenzialità della quantum computing), ha
affermato che “nessuno capisce la fisica quantistica”, e prima di lui, a
Niels Bohr, uno dei padri della fisica quantistica, veniva attribuita
l’opinione secondo cui “lo scopo della scienza non era più di spiegare
la natura, ma solo di descrivere ciò che si può dire sulla natura”. Uno degli aspetti controintuitivi (e importante per la QC) ha
a che fare con il concetto di misura in fisica quantistica, dove
è fondamentale evitare di cercare paralleli con la fisica classica. 

\ex{La Fisica Classica}{
  Se sappiamo che una sfera di ferro si può trovare in uno di n
punti diversi dello spazio, nel momento in cui misuriamo la posizione
della sfera veniamo a conoscenza del punto specifico, tra gli n
possibili, in cui la sfera si trova. Ma diamo per scontato che la sfera fosse in quel particolare punto
anche immediatamente prima che la misurazione avesse luogo. In altre parole, la misurazione compiuta non ha alcuna influenza sul
sistema misurato, e il sistema viene lasciato nello stesso stato in cui
si trovava prima di essere misurato: dopo la misurazione la sfera si
troverà nello stesso punto in cui si trovava prima della misurazione. 
}

\ex{Fisica Quantistica}{
  In fisica quantistica le cose funzionano diversamente. Supponiamo
che un elettrone possa trovarsi in una di n posizioni possibili con
probabilità $c_0, c_1, … c_{n-1}$. Indichiamo con $X = [c_0, c_1, c_2, …c_{n-1}]$ lo
stato dell’elettrone. (dove sarà: $c_0 + c_1 + …+ c_{n-1} = 1$).

Questo non significa (come saremmo tentati di assumere), che
l’elettrone si trova in posizione $k$ con probabilità $c_k$.
Dire che il sistema è nello stato $X$ significa dire che l’elettrone si
trova contemporaneamente in tutte le possibili posizioni ammesse.

Fino a che non decidiamo di misurare la posizione dell’elettrone per
sapere dove si trova, dobbiamo accettare che l’elettrone si trovi
non in una delle posizioni possibili, ma in tutte: più propriamente
diciamo che l’elettrone si trova in una sovrapposizione di stati.

Sarà solo quando decideremo di effettuare una misurazione, che
troveremo l’elettrone in una specifica posizione. È solo nell’atto della
misurazione di un oggetto quantistico che la sovrapposizione di
stati in cui si trova collassa in un singolo stato in senso ordinario.

Ma prima di essere misurato, un sistema quantistico si trova
contemporaneamente in più stati, e in questo risiede il potenziale
della computazione quantistica: corrisponde ad avere un
computer capace di eseguire un algoritmo processando in
parallelo (un parallelismo reale, non simulato) tutti i possibili
input ammessi per quell’algoritmo.
}

\nt{In fisica quantistica dire che un sistema si trova nello stato $X$ significa dire che, solo
dopo averlo misurato, lo troveremo in posizione k con probabilità $c_k$.}

\subsection{Bit e Qubit}

Un bit ordinario descrive un sistema che può trovarsi in uno di due
possibili stati: aperto/chiuso, acceso/spento, vero/falso. I due stati in
cui si può trovare il sistema vengono di solito indicati con 0 e 1. Il formalismo matematico alla base della fisica quantistica fa un uso
esteso di matrici e vettori, dunque proviamo a usare quel formalismo
per rappresentare un bit ordinario a 0 oppure a 1 mediante matrici
2 x 1. Indicheremo il valore 0 con $|0>$ e il valore 1 con $|1>$. 

\nt{La notazione $|x>$, detta \fancyglitter{ket}, è stata introdotta da Paul Dirac. 
Possiamo leggere la notazione matriciale che abbiamo introdotto in
questo modo: $|0>$ significa che il bit si trova nello stato “0” con
probabilità 1, e si trova nello stato “1” con probabilità 0. $|1>$ significa
che il bit si trova nello stato “0” con probabilità 0 e nello stato “1” con
probabilità 1.
}

\clm{}{}{
  \begin{itemize}
    \item Usare probabilità per rappresentare bit “normali” è inutilmente
pesante, ma è perfetto per rappresentare un qubit, ossia un sistema
quantistico che può trovarsi in una di due possibili configurazioni:
0 e 1, con probabilità diverse. 
\item Poiché il sistema quantistico che rappresenta un qubit può assumere,
una volta misurato, uno di due valori possibili, descriviamo il
sistema mediante una matrice a valori complessi con due righe e una
colonna. 
\item Un qubit, fino a che non viene misurato, si trova
sempre in una sovrapposizione di due stati, corrispondenti ai valori 0 e
1. Alla misurazione, lo stato del qubit collasserà in uno dei due stati
possibili, trasformandosi così in un bit ordinario di valore 0 o 1.
\item non è possibile “vedere” un qubit, perché nel
momento in cui lo si osserva, ossia lo si misura, questo si trasforma in
un normale bit, con un valore ben preciso 0 o 1. Non di meno i qubit
esistono in un senso molto reale.
\end{itemize}
}

\subsubsection{}

Nei computer ordinari, un bit viene implementato mediante due livelli
di tensione diversi, associati alle due informazioni 0 e 1. Per costruire un qubit, nella quantum computing si sfruttano alcune
caratteristiche delle particelle subatomiche. Per esempio, un fotone
può trovarsi in uno di due possibili stati di polarizzazione, oppure un
elettrone è dotato di una caratteristica fisica chiamata spin che può
assumere due possibili valori diversi, che possiamo associare a 0 e 1.
Ogni stato di un sistema formato da 8 bit, ossia un qubyte deve
essere scritto con una matrice di
256 numeri complessi che soddisfano la proprietà:
$|c_0|^2 + … +|c_{106}|^2 + |c_{107}|^2 + |c_{108}|^2 + … + |c_{255}|^2
=1$

Dunque, nel mondo ordinario per indicare lo stato di un byte
dobbiamo scrivere in tutto 8 bit. Ma nel mondo dei quanti, lo stato di
otto qubit (appunto, un qubyte) deve essere descritto usando 256
numeri complessi, perché ognuna delle 256 combinazioni possibili
di 8 bit deve essere associata alla probabilità che ha di
manifestarsi, quando viene misurata. Se volessimo simulare su un computer ordinario un
computer quantistico dotato di un solo registro a 64 qubit avremmo
bisogno di poter memorizzare 264 numeri complessi, ossia più di 18
miliardi di miliardi di numeri complessi.
Assumendo di usare 8 byte per scrivere un numero reale, e dato che
ogni numero complesso è formato da una coppia di numeri reali,
questo corrisponde ad uno spazio di più di 288 milioni di terabyte.

\subsection{Porte Logiche}

Un computer è in definitiva un circuito logico estremamente
complesso, in cui dell’informazione in input scritta su $n$ bit viene
trasformata in informazione in output scritta su $m$ bit passando
attraverso un certo numero di porte logiche.
Gli $n$ bit di informazione in input possono essere rappresentati come una matrice di $2^n$ righe x 1 colonna, e gli $m$ bit di output saranno rappresentati con una matrice di
$2^m$ righe x 1 colonna (matrici con più righe e una colonna sono di
solito chiamate vettori colonna, cioè vettori rappresentati in verticale). Dunque rappresenteremo sequenze di bit mediante vettori colonna, e porte logiche mediante matrici.

È noto che qualsiasi circuito logico può essere formulato con una
opportuna combinazione di porte NOT e porte AND. La matrice NOT deve trasformare il vettore colonna $|0>$ nel vettore colonna $|1>$, e viceversa. Abbastanza intuitivamente poi, moltiplicando la matrice NOT per la matrice AND otteniamo la matrice corrispondente alla porta logica
NAND (ossia una porta AND con l’output negato da una porta logica NOT). La NAND è una porta logica universale: qualsiasi circuito logico
può essere costruito con una opportuna combinazione di porte logiche
NAND.

In maniera del tutto analoga, anche nella Quantum Computing la
computazione avviene usando porte logiche quantistiche, che
manipolano qubit in input per trasformarli in qubit di output. Le porte logiche quantistiche devono soddisfare alcune condizioni matematiche, ma a
parte questo, anche nella QC esistono particolari insiemi di porte
logiche quantistiche che sono universali, ossia possono essere
composte fra loro in opportune combinazioni e usate per eseguire
qualsiasi tipo di computazione.
Ma, pensando sempre alla descrizione di porta logica data mediante
una opportuna matrice, la differenza fondamentale tra una porta logica
ordinaria e una porta logica quantistica è che in quest’ultima la
matrice che la descrive può essere composta da numeri diversi da 0 e
1 (e nel caso più generale complessi).

\dfn{Matrice di Hadamard}{
  La \newfancyglitter{matrice di Hadamard}, che ha la proprietà di mettere
un qubit in input in una sovrapposizione di stati per cui il qubit di
output è per metà nello stato $|0>$ e per metà nello stato $|1>$.
}

\nt{In altre parole, se misuriamo ripetutamente il qubit $|0>$ lo troveremo
sempre nello stato 0. Se invece misuriamo ripetutamente il qubit $H|0>$ (a cui è stata applicata la matrice di Hadamard) lo troveremo metà delle volte nello stato 0 e lo troveremo metà delle
volte nello stato 1.}

\subsubsection{}
Proprio come nel caso classico, più porte logiche quantistiche
possono essere combinate assieme per portare avanti computazioni
più complesse. Ciò equivale a moltiplicare fra loro le matrici
corrispondenti. Ma notiamo una differenza fondamentale col caso classico: nel
momento in cui li misuriamo, i qubit si trasformano in bit ordinari,
e qualsiasi vantaggio della computazione quantistica scompare.
Se dovessimo effettuare la misurazione tra le porte A e B, la nostra
computazione quantistica si fermerebbe in quel momento.
Dunque, la misurazione va posticipata fino a quando non sono stati
eseguiti tutti i calcoli desiderati in modalità quantistica.

\dfn{Decoerenza Quantistica}{
  Le particelle subatomiche che implementano i qubit tendono spontaneamente a
interagire con l’ambiente circostante e, così facendo, in tempi
brevissimi si trasformano in bit ordinari e perdono l’informazione che
veicolavano.
}

\nt{Più è complesso il calcolo quantistico da portare avanti, maggiore è
il numero di porte logiche quantistiche da attraversare, e maggiore
sarà la probabilità che la decoerenza avvenga prima del
completamento del calcolo, vanificandolo. Mitigare questo problema
richiede soluzioni tecniche complesse e di difficile implementazione.}

\section{Algoritmi Quantistici}

Gli algoritmi quantistici non sono veri e propri algoritmi, almeno non
nel senso in cui normalmente intendiamo un algoritmo. Non esistono criteri generali per la scrittura degli algoritmi quantistici,
e la soluzione a un certo problema è sempre un po’ ad hoc, studiata
specificamente per quel problema e non immediatamente estendibile
ad altri problemi. 
Nei casi in cui l’algoritmo quantistico si dimostra inerentemente
superiore a una soluzione classica, ciò si deve alla possibilità di
sfruttare il concetto di sovrapposizione degli stati. In questo modo,
i diversi casi che un algoritmo classico dovrebbe considerare uno
per uno, nella versione quantistica possono essere processati tutti
insieme contemporaneamente.

\dfn{Algoritmo Quantistico di Deutsch}{
  
}

























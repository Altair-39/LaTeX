\chapter{Introduzione}

\nt{DISCLAIMER: questi appunti sono stati scritti da una persona che ha dovuto dare questo esame in magistrale (yeah, l'ho skippato in triennale e ora mi tocca darlo): essendo che ho dato molti esami che dipendono da questo insegnamento e possibile che in alcune sezioni dia per scontato delle cose.}

\section{Che Cos'è l'Intelligenza Artificiale?}

Nell'immaginario l'intelligenza artificiale viene solitamente assimilata a quella di un robot antropomorfo che risolve problemi complessi e impara da essi.

\begin{center}
	\begin{minipage}{0.45\textwidth}
		\centering
		\includegraphics[scale=0.45]{01/ia.png}
	\end{minipage}%
	\hfill
	\begin{minipage}{0.45\textwidth}
		\centering
		\includegraphics[scale=0.45]{01/ia2.png}
	\end{minipage}
\end{center}

\paragraph{Però esistono altri tipi di IA:}

\begin{itemize}
	\item Servizi di streaming: portali per l'accesso a molti files. Utilizzano meccanismi di personalizzazione.
	\item Social network.
	\item Assistenti virtuali.
	\item Macchine fotografihche/Smartphone.
\end{itemize}

\subsection{Un Inizio}

\dfn{Intelligenza}{Complesso di facoltà psichiche e mentali che
	consentono all’uomo di pensare, comprendere o
	spiegare i fatti o le azioni, elaborare modelli astratti
	della realtà, intendere e farsi intendere dagli altri,
	giudicare, e lo rendono insieme capace di adattarsi a
	situazioni nuove e di modificare la situazione stessa
	quando questa presenta ostacoli all’adattamento;
	propria dell’uomo, in cui si sviluppa gradualmente a
	partire dall’infanzia e in cui è accompagnata dalla
	consapevolezza e dall’autoconsapevolezza, è
	riconosciuta anche, entro certi limiti (memoria
	associativa, capacità di reagire a stimoli interni ed
	esterni, di comunicare in modo anche complesso, ecc.),
	agli animali.}
\nt{Artificiale indica che non è naturale.}

\paragraph{Prospettiva storica:}

\begin{itemize}
	\item 1936: Alan Turing formalizza la Turing Machine.
	\item 1940: ENIAC: primo computer "moderno".
	\item 1950: Test di Turing, quando un computer può dirsi intelligente?
	\item Il dubbio nasce dal contesto bellico in cui vennerò sviluppati i primi computer: all'epoca solo poche persone istruite riuscivano a fare i calcoli necessari.
	\item 1956: Nasce l'intelligenza artificiale.
\end{itemize}

\paragraph{Breve storia dell'automazione:}

\begin{itemize}
	\item \fancyglitter{Automazione del calcolo:} metà anni '50, pochi dati, molti calcoli.
	\item \fancyglitter{Automazione di procedure amministrative e contabili:} metà anni '60, pochi calcoli, grandi molti di dati alfanumerici.
	\item \fancyglitter{Automazione di fabbrica:} metà anni '70, primi robot industriali, ambiente predeterminato.
	\item \fancyglitter{Automazione di ufficio:} metà anni '80, primi PC, primi strumenti per utenti non esperti.
	\item \fancyglitter{Automazione della ricerca delle informazioni:} fine anni '90, internet, WEB, motori di ricerca.
\end{itemize}

\qs{}{L'automazione è intelligenza?}

\paragraph{Ragionando:} la calcolatrice è automatica, ma non si può dire intellingente. Una lavatrice è automatica, ha diversi programmi e si adatta. Un rover che gira su Marte effettua esperimenti e si adatta, ha una certa autonomia. Infine, gli LLM eseguono un programma e hanno la capacità di comunicare mediante linguaggio naturale.

\subsection{Test di Turing}

\qs{}{Quando un programma può dirsi intelligente?}

\dfn{Turing Test (The Imitation Game)}{
	Un'intervistatore deve capire se un'entità misteriosa è umana o è una macchina. Può fare tutte le domande che vuole su qualsiasi argomento e l'entità deve rispondere (il tutto per scritto). Al termine l'intervistatore enuncia il suo verdetto. Se dice uomo ed era macchina, la macchina ha superato il test.
}

\paragraph{AI:}

\begin{itemize}
	\item Data e luogo di nascita:
	      \begin{itemize}
		      \item Darthmouth Conference (USA), 1956.
		      \item Nome scelto da John McCarthy.
	      \end{itemize}
	\item In precedenza:
	      \begin{itemize}
		      \item Una macchina può pensare ed essere considerata intelligente?
		      \item Vari approcci: cybernetica, teoria degli automi, etc.
		      \item Turing test.
	      \end{itemize}
	\item Successivamente:
	      \begin{itemize}
		      \item Scacchi.
		      \item Giochi.
		      \item Dimostrazioni automatiche.
	      \end{itemize}
\end{itemize}

\qs{}{Basta produrre gli output attesi per dire che vi è comprensione?}

\begin{itemize}
	\item Si può dare una risposta "giusta" avendo certe conoscenze e ragionamento.
	\item Ma si può dare una risposta "giusta" anche tirando a caso.
\end{itemize}

\dfn{Esperimento della Stanza Cinese}{
	Una persona interagisce con un computer, programmato per rispondere con certi
	ideogrammi cinesi ad altri ideogrammi cinesi ricevuti in input. Il computer parla cinese? Lo capisce?
}

\begin{figure}[h]
	\centering
	\includegraphics[scale=0.4]{01/cinese.png}
	\caption{Esperimento di Searle.}
\end{figure}

\nt{Ma supponiamo che una persona chiusa in una
	stanza ha istruzioni per rispondere con certi ideogrammi cinesi in risposta ad altri ideogrammi cinesi. Parla cinese? Lo capisce?}

\begin{figure}[h]
	\centering
	\includegraphics[scale=0.35]{01/cinese2.png}
	\caption{Esperimento di Searle.}
\end{figure}

\dfn{Test di Turing Inverso}{
	Usati per intercettare bot che cercano di accedere a form o a dati (C.A.P.T.C.H.A.).
}

\nt{Una variante di questo test è usata in "Ma gli androidi sognano pecore elettriche?" (Blade Runner).}

\subsection{Strong e Weak AI}

\paragraph{Due tipi di intelligenza:}

\begin{itemize}
	\item \fancyglitter{Strong AI:} è possibile riprodurre l'intelligenza umana?
	\item \fancyglitter{Weak AI:} è possibile trovare dei modi per risolvere problemi che, se risolti dagli esseri umani richiederebbero intelligenza?
\end{itemize}


\paragraph{Obiettivo della weak AI:}

\begin{itemize}
	\item Programmare un agente artificiale in grado di:
	      \begin{itemize}
		      \item Rilevare ostacoli.
		      \item Rilevare oggetti in movimento.
		      \item Costruire un piano d'azione.
		      \item Rilevare segnali significativi.
	      \end{itemize}
	\item In un ambiente che è:
	      \begin{itemize}
		      \item Complesso.
		      \item Parzialmente prevedibile.
		      \item Parzialmente collaborativo.
	      \end{itemize}
\end{itemize}

\nt{Nasce il binomio Agente-Ambiente.}

\dfn{Agente}{
	Un agente è un'astrazione che rappresenta un qualsiasi sistema che percepisce il proprio ambiente tramite i sensori e agisce su di esso tramite degli attuatori.
}

\clm{Caratteristiche dell'ambiente}{}{
	\begin{itemize}
		\item Completamente osservabile: in ogni istante i sensori danno accesso a
		      tutti gli aspetti dell’ambiente rilevanti per
		      la scelta dell’azione.
		\item Parzialmente osservabile: i sensori danno accesso solo a parte
		      dell’informazione rilevante (cause:
		      sensori imprecisi oppure non in grado di
		      rilevare alcuni dati).
		\item Deterministico: lo stato successivo è determinato dallo
		      stato corrente e dall’azione applicata.
		\item Stocastico: applicando più volte una stessa azione in
		      uno stesso stato si possono raggiugnere
		      stati diversi. Si dice strategico quando è
		      stocastico solo per quanto riguarda le
		      azioni degli altri agenti.
		\item Epistodico: l’esperienza degli agenti è divisa in
		      episodi atomici: un episodio è dato da
		      una percezione seguita da una singola
		      azione (esempio: classificazione).
		\item Sequenziale: attività composta da più passi ognuno dei
		      quali in generale influenzerà i successivi.
		\item Statico: l'ambiente non cambia mentre l'agente "pensa".
		\item Dinamico: l'ambiente cambia mentre l'agente "pensa".
		\item Discreto: possono essere discreti stato, tempo,
		      percezioni, azioni (esempio: gli scacchi
		      hanno stati, percezioni, azioni discreti).
		\item Continuo: possono essere continui stato, tempo,
		      percezioni, azioni (esempio: gli scacchi
		      hanno tempo continuo).
		\item Singolo agente: viene modellata come agente una sola
		      entità.
		\item Multi agente: vengono modellate come agenti più
		      entità
	\end{itemize}
}

\paragraph{Paradigmi di programmazione:}

\begin{itemize}
	\item Approccio tradizionale:
	      \begin{itemize}
		      \item Imperativo.
		      \item A oggetti.
		      \item \fancyglitter{Non è AI:} risolve un singolo compito ed è strutturato come una sequenza di passi.
		      \item Descrivono il COME.
	      \end{itemize}
	\item Approccio dichiarativo:
	      \begin{itemize}
		      \item Separa una descrizione del COSA da un programma generale.
		      \item Lo stesso programma è applicato a diverse descrizioni per risolvere problemi diversi.
	      \end{itemize}
\end{itemize}

\subsection{Esempi}

\dfn{Mondo dei Blocchi}{
	Tipico Toy Problem in ambito AI. Si hanno un tavolo con $n$ posizioni e $m$ blocchi. L'obiettivo è passare da uno stato iniziale a uno stato finale. Un agente può spostare un blocco per volta seguendo determinate regole.
}

\begin{figure}[!h]
	\centering
	\includegraphics[scale=0.35]{01/blocchi.png}
	\caption{Mondo dei blocchi.}
\end{figure}

\paragraph{L'agente:}

\begin{itemize}
	\item Percepisce la situazione iniziale.
	\item Costruisce i passi per andare dalla situazione iniziale alla situazione finale.
	\item Deve determinare quali azioni lo avvicinano al \fancyglitter{goal}.
\end{itemize}

\begin{figure}[!h]
	\centering
	\includegraphics[scale=0.45]{01/delibera.png}
	\caption{Meccanismo di deliberazione.}
\end{figure}

\qs{}{Come scegliere le mosse?}

\begin{itemize}
	\item Dipende dal tipo di problema.
	\item A volte basta la prima mossa, a volte si vuole trovare una soluzione ottima.
\end{itemize}

\subsection{Definire AI}

\dfn{Automazione}{
	Si deve programmare la macchina per fare ogni passo: è applicabile in domini fortemente ripetitivi.
}

\dfn{Autonomia}{
	Un agente artificiale riceve compiti ad alto livello, l'utente demanda all'agente la risoluzione.
}

\paragraph{Un agente autonomo:}

\begin{itemize}
	\item Riceve solo compiti ad alto livello.
	\item Ragiona ed esplora alternative (molte mosse possibili a ogni istante).
	\item Riconosce quando non si può andare avanti su una strada \footnote{nota aggiuntiva: non proprio, dipende dal tipo di agente e dal suo control loop.}.
	\item Riconosce che si è già stati in quella situazione.
	\item Prima si ragiona e poi si agisce.
\end{itemize}

\nt{
	In AI gli agenti autonomi sono un modo di concepire
	i programmi, in cui controllo e logica (o modello) sono
	chiaramente separati.

	Un agente fa sempre ciò che è programmato a fare\footnote{No Terminator, sorry :'(}.
}

\qs{}{Cosa vuol dire fare la cosa giusta?}

\dfn{Funzione Deliberativa}{
	La funzione deliberativa di un agente determina le azioni che saranno eseguite. In termini informali un agente è razionale quando “fa la cosa giusta”, cioè opera per conseguire il “successo”.
}

\nt{Occorre quindi definire una \fancyglitter{misura di prestazione}.}

\paragraph{Il comportamento razionale di un agente dipende da 4 fattori:}

\begin{enumerate}
	\item Azioni nelle facoltà dell'agente.
	\item Misura di prestazione.
	\item Conoscenza dell'ambiente.
	\item Percezione.
\end{enumerate}

\cor{Agente Razionale}{
	Un agente razionale dovrebbe scegliere sempre un’azione
	che massimizza la misura di prestazione attesa, data la
	particolare sequenza percettiva in oggetto e le informazioni
	derivabili dalla conoscenza dell’ambiente.
}

\paragraph{Definizioni di AI:}

\begin{itemize}
	\item Sistemi che pensano come esseri umani:
	      \begin{itemize}
		      \item Haugeland, 1985.
		      \item Bellman, 1978.
	      \end{itemize}
	\item Sistemi che agiscono come esseri umani:
	      \begin{itemize}
		      \item Kuzweil, 1990.
		      \item Rich e Knight, 1991.
	      \end{itemize}
	\item Sistemi che pensano razionalmente:
	      \begin{itemize}
		      \item Charniak e McDermott, 1985.
		      \item Winston, 1992.
	      \end{itemize}
	\item Sistemi che agiscono razionalmente:
	      \begin{itemize}
		      \item Poole et al., 1998.
		      \item Nilsson, 1998.
	      \end{itemize}
\end{itemize}

\qs{}{Quali problemi per l'AI:}

\begin{itemize}
	\item Non è adatta per:
	      \begin{itemize}
		      \item Modelli matematici precisi.
		      \item Metodi algoritmici specifici.
	      \end{itemize}
	\item È utile/necessaria per:
	      \begin{itemize}
		      \item Problemi non deterministici.
		      \item Più soluzioni.
		      \item Dati non numerici.
		      \item Grandi Knowledge Base (KB).
		      \item Interazione con ambiente ed esseri umani.
	      \end{itemize}
\end{itemize}

\section{Risoluzione Automatica di Problemi}

In questa parte si affronta la problematica di come definire il concetto di problema e di soluzione,
di distinguere tra soluzione e soluzione ottima. Sono studiati tre approcci alla risoluzione di
problemi: ricerca nello spazio degli stati, ricerca in spazi con avversario (giochi ad informazione
completa), risoluzione di problemi mediante soddisfacimento di vincoli.

\subsection{I Problemi}

\begin{itemize}
	\item La realtà che definisce un problema può essere astratta in un insieme di stati.
	\item La realtà transisce da uno stato ad un altro tramite l’esecuzione di azioni (o operazioni).
\end{itemize}

\paragraph{Caratteristiche:}

\begin{itemize}
	\item \fancyglitter{Stati discreti} (o dentro o fuori, non ci sono stati graduali).
	\item Effetto \fancyglitter{deterministico} delle azioni.
	\item \fancyglitter{Dominio statico} (non cambia durante l'esecuzione delle azioni).
\end{itemize}

\paragraph{Esempio non deterministico:}

\begin{itemize}
	\item Eseguendo più volte la stessa azione si possono avere conseguenze diverse.
	\item Si hanno \fancyglitter{stati continui}.
\end{itemize}

\dfn{Obiettivo}{
	Un obiettivo (goal) è un risultato verso il quale gli sforzi sono diretti. È una condizione data in termini di:
	\begin{itemize}
		\item Situazione.
		\item Prestazione.
	\end{itemize}
}

\nt{L'insieme degli stati obiettivo sono tutti gli stati in cui vale la condizione che li definisce.}

\dfn{Algoritmo di Ricerca}{
	L’algoritmo di ricerca determina una soluzione
	che, a partire da uno stato iniziale, permette di
	raggiungere un dato stato obiettivo. Usa:
	\begin{itemize}
		\item Una descrizione del problema.
		\item Un metodo di ricerca attraverso lo spazio degli stati.
	\end{itemize}
}

\cor{Soluzione}{
	Una soluzione è un percorso nello spazio degli stati.
}

\paragraph{Un problema di ricerca può essere definito come una tupla di 4 elementi:}

\begin{enumerate}
	\item Stato iniziale: cattura la situazione a partire dalla quale viene computata la soluzione.
	\item Funzione successore: dato uno stato e un'azione legale in esso calcola lo stato a cui si transisce eseguendo quell'azione in quello stato.
	\item Test obiettivo: determina se lo stato a cui è applicato è lo stato goal: può verificare
	      una proprietà o verificare l’appartenenza dello stato all’insieme degli
	      stati target.
	\item Funzione di costo del cammino: dato un percorso possibile gli assegna un costo numerico.
\end{enumerate}

\paragraph{Alcune astrazioni:}

\begin{itemize}
	\item \fancyglitter{Stati:} occorre rappresentare solo l'informazione rilevante alla soluzione del problema.
	\item \fancyglitter{Azioni:} occorre rappresentare solo gli aspetti funzionali alla soluzione del problema.
	\item \fancyglitter{Toy problem:} un problema artificiale avente lo scopo di
	      illustrare o mettere alla prova dei metodi di risoluzione.
	      Ha una formulazione precisa e univoca. Utile per
	      confrontare metodi diversi
	\item \fancyglitter{Real-world problem:} problemi concreti, effettivi. Spesso
	      non hanno una formulazione unica.
\end{itemize}

\nt{E.g. di toy problems: problema dell'aspirapolvere, gioco dell'8, problema delle 8 regine.}

\paragraph{Possibili approcci:}

\begin{itemize}
	\item \fancyglitter{Blind:} usano esclusivamente la struttura del problema per cercare una soluzione.
	\item \fancyglitter{Informati:} usano la struttura del problema e ulteriore conoscenza per guidare la ricerca.
\end{itemize}

\subsection{Metodi di Ricerca non Informati (Blind Search)}

\dfn{Albero di Ricerca}{
	Un albero di ricerca è una struttura dati usata per trovare una soluzione a un problema di ricerca:
	\begin{itemize}
		\item Ogni nodo corrisponde a uno stato.
		\item I nodi figli sono costruiti tramite la funzione successore.
		\item Ogni nodo ha un riferimento al nodo padre (per ricostruire le
		      soluzioni).
		\item L’albero è costruito a partire dal nodo corrispondente allo stato
		      iniziale.
		\item L’albero diventa un grafo quando lo stesso nodo (NB: non lo stesso
		      stato) può essere raggiunto tramite più percorsi.
		\item Un percorso che porta dal nodo iniziale a un nodo obiettivo è una
		      soluzione.
	\end{itemize}
}

\nt{Gli alberi sono un caso specifico dei grafi.}

\paragraph{Formalizzando:}

\begin{itemize}
	\item Un \fancyglitter{grafo di ricerca} $G = (\{n_i\}, \{e_{i j}\})$ è costituito da un insieme di nodi $n_i$ e di archi $e_{i j}$.
	\item $e_{pq} \in \{e_{i j}\}$ rappresenta l'esistenza di un arco dal nodo $n_p$ al nodo $n_q$, quindi $n_q$ è successore di $n_p$.
	\item Ciascun arco $e_{i j}$ ha associato un costo $c_{i j}$.
	\item L'esistenza di $e_{i j}$ non implica l'esistenza di $e_{j i}$.
\end{itemize}

\begin{figure}[!h]
	\centering
	\includegraphics[scale=0.45]{01/8.png}
	\caption{Esempio con il gioco dell'8.}
\end{figure}

\qs{}{Se le strategie sono tante ve ne è una migliore?
	Come le confronto?}

\paragraph{Criteri di valutazione:}

\begin{itemize}
	\item \fancyglitter{Completezza:} garanzia di trovare una soluzione, se esiste.
	\item \fancyglitter{Ottimalità:} garanzia di trovare una soluzione ottima (a costo
	      minimo)\footnote{Corso di "Algoritmi e Complessità}.
	\item \fancyglitter{Complessità temporale:} quanto tempo occorre per trovare una soluzione.
	\item \fancyglitter{Complessità spaziale:} quanta memoria occorre per effettuare la ricerca.
\end{itemize}

\qs{}{Come si valuta la complessità?}

\begin{itemize}
	\item \fancyglitter{Complessità computazionale:} dato un problema esistono infiniti algoritmi che lo risolvono.
	\item \fancyglitter{Termine di paragone:}
	      \begin{itemize}
		      \item Tempo.
		      \item Spazio.
	      \end{itemize}
	\item \fancyglitter{Criterio di preferenza:} economicità.
\end{itemize}

\nt{
	Per astrarre dal calcolatore utilizzato lo spazio e il tempo non sono metrici, ma parametrici. E.g. Numero di nodi creati o visitati.

	È interessante vedere l'andamento del costo al variare della dimensione del problema.
}

\subsection{Lista di Strategie}

\dfn{Ricerca in Ampiezza}{
	La ricerca espande il nodo radice, poi tutti i suoi successori, poi
	tutti i discendenti di secondo livello, ecc.

	Si realizza gestendo la frontiera come una coda FIFO.
}
\begin{center}
	\begin{minipage}{0.45\textwidth}
		\centering
		\includegraphics[scale=0.5]{01/amp.png}
	\end{minipage}%
	\hfill
	\begin{minipage}{0.45\textwidth}
		\centering
		\includegraphics[scale=0.5]{01/fifo.png}
	\end{minipage}
\end{center}

\paragraph{Valutazione:}

\begin{itemize}
	\item \fancyglitter{Completezza:} se esiste un nodo obiettivo a una profondità finita $d$, la
	      ricerca in ampiezza lo troverà a patto che il fattore di ramificazione $b$
	      (cioè il numero di figli che un nodo può avere) sia finito.
	\item \fancyglitter{Ottimalità:} la soluzione trovata è ottima solo se il costo del cammino
	      è una funzione monotona crescente della profondità (es. tutte le azioni
	      hanno lo stesso costo).
	\item \fancyglitter{Complessità temporale:} $O(b^{d + 1})$.
	\item \fancyglitter{Complessità spaziale:}  $O(b^{d + 1})$, perché bisogna tenere in memoria sia la frontiera che gli antenati.
\end{itemize}

\dfn{Ricerca a Costo Uniforme}{

}

\dfn{Ricerca in Profondità}{

}

\dfn{Iterative Deepening}{

}

\dfn{Ricerca Bidirezionale}{

}

\section{Ricerca Informata}


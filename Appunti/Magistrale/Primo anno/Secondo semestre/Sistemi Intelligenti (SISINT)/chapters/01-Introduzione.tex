\chapter{Introduzione}

\section{Che Cos'è l'Intelligenza Artificiale?}

Nell'immaginario l'intelligenza artificiale viene solitamente assimilata a quella di un robot antropomorfo che risolve problemi complessi e impara da essi.

\begin{center}
	\begin{minipage}{0.45\textwidth}
		\centering
		\includegraphics[scale=0.45]{01/ia.png}
	\end{minipage}%
	\hfill
	\begin{minipage}{0.45\textwidth}
		\centering
		\includegraphics[scale=0.45]{01/ia2.png}
	\end{minipage}
\end{center}

\paragraph{Però esistono altri tipi di IA:}

\begin{itemize}
	\item Servizi di streaming: portali per l'accesso a molti files. Utilizzano meccanismi di personalizzazione.
	\item Social network.
	\item Assistenti virtuali.
	\item Macchine fotografihche/Smartphone.
\end{itemize}

\subsection{Un inizio}

\dfn{Intelligenza}{Complesso di facoltà psichiche e mentali che
	consentono all’uomo di pensare, comprendere o
	spiegare i fatti o le azioni, elaborare modelli astratti
	della realtà, intendere e farsi intendere dagli altri,
	giudicare, e lo rendono insieme capace di adattarsi a
	situazioni nuove e di modificare la situazione stessa
	quando questa presenta ostacoli all’adattamento;
	propria dell’uomo, in cui si sviluppa gradualmente a
	partire dall’infanzia e in cui è accompagnata dalla
	consapevolezza e dall’autoconsapevolezza, è
	riconosciuta anche, entro certi limiti (memoria
	associativa, capacità di reagire a stimoli interni ed
	esterni, di comunicare in modo anche complesso, ecc.),
	agli animali.}
\nt{Artificiale indica che non è naturale.}

\paragraph{Prospettiva storica:}

\begin{itemize}
	\item 1936: Alan Turing formalizza la Turing Machine.
	\item 1940: ENIAC: primo computer "moderno".
	\item 1950: Test di Turing, quando un computer può dirsi intelligente?
	\item Il dubbio nasce dal contesto bellico in cui vennerò sviluppati i primi computer: all'epoca solo poche persone istruite riuscivano a fare i calcoli necessari.
	\item 1956: Nasce l'intelligenza artificiale.
\end{itemize}

\paragraph{Breve storia dell'automazione:}

\begin{itemize}
	\item \fancyglitter{Automazione del calcolo:} metà anni '50, pochi dati, molti calcoli.
	\item \fancyglitter{Automazione di procedure amministrative e contabili:} metà anni '60, pochi calcoli, grandi molti di dati alfanumerici.
	\item \fancyglitter{Automazione di fabbrica:} metà anni '70, primi robot industriali, ambiente predeterminato.
	\item \fancyglitter{Automazione di ufficio:} metà anni '80, primi PC, primi strumenti per utenti non esperti.
	\item \fancyglitter{Automazione della ricerca delle informazioni:} fine anni '90, internet, WEB, motori di ricerca.
\end{itemize}

\qs{}{L'automazione è intelligenza?}

\paragraph{Ragionando:} la calcolatrice è automatica, ma non si può dire intellingente. Una lavatrice è automatica, ha diversi programmi e si adatta. Un rover che gira su Marte effettua esperimenti e si adatta, ha una certa autonomia. Infine, gli LLM eseguono un programma e hanno la capacità di comunicare mediante linguaggio naturale.

\subsection{Test di Turing}

\qs{}{Quando un programma può dirsi intelligente?}

\dfn{Turing Test (The Imitation Game)}{
	Un'intervistatore deve capire se un'entità misteriosa è umana o è una macchina. Può fare tutte le domande che vuole su qualsiasi argomento e l'entità deve rispondere (il tutto per scritto). Al termine l'intervistatore enuncia il suo verdetto. Se dice uomo ed era macchina, la macchina ha superato il test.
}

\paragraph{AI:}

\begin{itemize}
	\item Data e luogo di nascita:
	      \begin{itemize}
		      \item Darthmouth Conference (USA), 1956.
		      \item Nome scelto da John McCarthy.
	      \end{itemize}
	\item In precedenza:
	      \begin{itemize}
		      \item Una macchina può pensare ed essere considerata intelligente?
		      \item Vari approcci: cybernetica, teoria degli automi, etc.
		      \item Turing test.
	      \end{itemize}
	\item Successivamente:
	      \begin{itemize}
		      \item Scacchi.
		      \item Giochi.
		      \item Dimostrazioni automatiche.
	      \end{itemize}
\end{itemize}

\qs{}{Basta produrre gli output attesi per dire che vi è comprensione?}

\begin{itemize}
	\item Si può dare una risposta "giusta" avendo certe conoscenze e ragionamento.
	\item Ma si può dare una risposta "giusta" anche tirando a caso.
\end{itemize}

\dfn{Esperimento della Stanza Cinese}{
	Una persona interagisce con un computer, programmato per rispondere con certi
	ideogrammi cinesi ad altri ideogrammi cinesi ricevuti in input. Il computer parla cinese? Lo capisce?
}

\begin{figure}[h]
	\centering
	\includegraphics[scale=0.4]{01/cinese.png}
	\caption{Esperimento di Searle.}
\end{figure}

\nt{Ma supponiamo che una persona chiusa in una
	stanza ha istruzioni per rispondere con certi ideogrammi cinesi in risposta ad altri ideogrammi cinesi. Parla cinese? Lo capisce?}

\begin{figure}[h]
	\centering
	\includegraphics[scale=0.4]{01/cinese2.png}
	\caption{Esperimento di Searle.}
\end{figure}

\dfn{Test di Turing Inverso}{
	Usati per intercettare bot che cercano di accedere a form o a dati (C.A.P.T.C.H.A.).
}

\nt{Una variante di questo test è usata in "Ma gli androidi sognano pecore elettriche?" (Blade Runner).}

\subsection{Strong e Weak AI}

\paragraph{Due tipi di intelligenza:}

\begin{itemize}
	\item \fancyglitter{Strong AI:} è possibile riprodurre l'intelligenza umana?
	\item \fancyglitter{Weak AI:} è possibile trovare dei modi per risolvere problemi che, se risolti dagli esseri umani richiederebbero intelligenza?
\end{itemize}


\paragraph{Obiettivo della weak AI:}

\begin{itemize}
	\item Programmare un agente artificiale in grado di:
	      \begin{itemize}
		      \item Rilevare ostacoli.
		      \item Rilevare oggetti in movimento.
		      \item Costruire un piano d'azione.
		      \item Rilevare segnali significativi.
	      \end{itemize}
	\item In un ambiente che è:
	      \begin{itemize}
		      \item Complesso.
		      \item Parzialmente prevedibile.
		      \item Parzialmente collaborativo.
	      \end{itemize}
\end{itemize}

\nt{Nasce il binomio Agente-Ambiente.}

\dfn{Agente}{
	Un agente è un'astrazione che rappresenta un qualsiasi sistema che percepisce il proprio ambiente tramite i sensori e agisce su di esso tramite degli attuatori.
}

\clm{Caratteristiche dell'ambiente}{}{
	\begin{itemize}
		\item Completamente osservabile: in ogni istante i sensori danno accesso a
		      tutti gli aspetti dell’ambiente rilevanti per
		      la scelta dell’azione.
		\item Parzialmente osservabile: i sensori danno accesso solo a parte
		      dell’informazione rilevante (cause:
		      sensori imprecisi oppure non in grado di
		      rilevare alcuni dati).
		\item Deterministico: lo stato successivo è determinato dallo
		      stato corrente e dall’azione applicata.
		\item Stocastico: applicando più volte una stessa azione in
		      uno stesso stato si possono raggiugnere
		      stati diversi. Si dice strategico quando è
		      stocastico solo per quanto riguarda le
		      azioni degli altri agenti.
		\item Epistodico: l’esperienza degli agenti è divisa in
		      episodi atomici: un episodio è dato da
		      una percezione seguita da una singola
		      azione (esempio: classificazione).
		\item Sequenziale: attività composta da più passi ognuno dei
		      quali in generale influenzerà i successivi.
		\item Statico: l'ambiente non cambia mentre l'agente "pensa".
		\item Dinamico: l'ambiente cambia mentre l'agente "pensa".
		\item Discreto: possono essere discreti stato, tempo,
		      percezioni, azioni (esempio: gli scacchi
		      hanno stati, percezioni, azioni discreti).
		\item Continuo: possono essere continui stato, tempo,
		      percezioni, azioni (esempio: gli scacchi
		      hanno tempo continuo).
		\item Singolo agente: viene modellata come agente una sola
		      entità.
		\item Multi agente: vengono modellate come agenti più
		      entità
	\end{itemize}
}






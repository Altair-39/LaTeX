\chapter{CSP e Rappresentazione della conoscenza}

\section{Constraint Satisfaction Problem}

\subsection{Introduzione}

\dfn{Constraint Satisfaction Problem}{
	Un constraint satisfaction problem (CSP) è definito
	da:
	\begin{itemize}
		\item Un insieme di variabili $X_1, \dots, X_n$.
		\item Un insieme di vincoli $C_1,\dots,C_m$.
		\item In alcuni casi è richiesta la massimizzazione di una funzione obiettivo.
	\end{itemize}
}

\cor{Stati}{
	Gli stati di un CSP sono dati da tutti gli assegnamenti
	possibili per le variabili del CSP.

	Un assegnamento $X_{i1} = v_{i1}, X_{i2} = v_{i2}$ è un'attribuzione di valori a un sottoinsieme delle variabili del CSP.
}

\paragraph{Un assegnamento è detto:}

\begin{itemize}
	\item \fancyglitter{Completo:} se assegna valori a tutte le variabili del CSP.
	\item \fancyglitter{Consistente:} se non viola alcun vincolo del CSP.
	\item \fancyglitter{Soluzione:} se è completo e consistente.
\end{itemize}

\nt{Quando esiste una soluzione per un vincolo si dice anche che
	esiste un mondo possibile che soddisfa il vincolo.

	I vincoli binari possono essere rappresentati come archi di un grafo i cui nodi sono le variabili del CSP.
}








\section{Rappresentazione della Conoscenza}

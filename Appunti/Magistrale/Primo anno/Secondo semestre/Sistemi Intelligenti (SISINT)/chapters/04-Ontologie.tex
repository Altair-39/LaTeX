\chapter{Ontologie e Agenti}

\qs{}{
	Quali sono gli aspetti fondamentali della costruzione e del
	mantenimento delle KB?
}

\begin{itemize}
	\item Il mondo reale non è fatto di formule, è fatto di oggetti.
	\item Le persone concettualizzano tali oggetti e le relazioni che questi intrattengono gli uni con gli altri.
\end{itemize}

\section{Tassonomie e Categorie}

\subsection{introduzione}

\dfn{Categorie}{
	Gli esseri umani interpretano la realtà per categorie. Una parte consistente dell’apprendimento consiste nel definire
	e ridefinire categorie.
}

\nt{È necessario standardizzare la
	rappresentazione di categorie, introdurre relazioni fra
	categorie e implementare meccanismi di eredità di
	proprietà fra categorie.}

\dfn{Tassonomia}{
	Organizzazione gerarchica di categorie o concetti.
}


\paragraph{Predicati:}

\begin{itemize}
	\item Member(P, C) è un predicato che restituisce
	      vero se P è un elemento della categoria C (in
	      questo caso P è detto istanza di C).
	\item is-a(C1, C2) è una relazione tra due categorie con C1 sottocategoria di C2.
\end{itemize}

\paragraph{Due categorie:}

\begin{itemize}
	\item \fancyglitter{Sono disgiunte:} quando non hanno istanze in comune.
	\item \fancyglitter{Costituiscono una decomposizione esaustiva:} quando tutte e
	      istanze della sovracategoria appartengono necessariamente ad
	      almeno una delle categorie considerate (che potranno avere anche
	      istanze comuni).
	\item \fancyglitter{Costituiscono una partizione:} quando sono disgiunte e
	      costituiscono una decomposizione esaustiva.
\end{itemize}

\paragraph{Vengono definite le seguenti proprietà:}

\begin{itemize}
	\item S è un insieme disgiunto di categorie:
	      $$\text{Disjoint}(S) \Rightarrow \forall X_i, X_j \in S, X_i \not = X_j \Rightarrow \text{Intersection}(X_i, X_j) = \{\}$$
	\item S è una decomposizione esaustiva di C:
	      $$\text{ExhaustiveDec}(S, C) \Rightarrow \forall I (\text{Member}(I, C) \Leftrightarrow \exists X_i \text{is-a}(X_i, C) \land \text{Member}(I, X_i)) $$
	\item S è una partizione di C:
	      $$\text{Partition}(S, C) \Leftrightarrow \text{Disjoint}(S) \land \text{ExhaustiveDec}(S, C)$$
\end{itemize}

\subsection{Proprietà}

\cor{Part-Of}{
	Indica che alcuni oggetti sono parti di altri. Gode della proprietà transitiva:

	\[
		\text{Part-of}(X, Y) \land \text{Part-of}(Y, Z) \Rightarrow \text{Part-of}(X, Z)
	\]

}

\cor{Bunch-Of}{
	A volte è comodo indicare che un oggetto è composto da parti
	senza specificare le relazioni fra queste ultime. Per far ciò si utilizza la nozione di bunch:

	\[
		\forall x \text{In}(x, s) \Rightarrow \text{Part-of}(x, \text{Bunch-of}(s))
	\]

}

\section{Ontologie}

\subsection{Introduzione}

\dfn{Ontologia}{
	Una KB descrittiva di un dominio può assumere forma più generale
	di quella tassonomica. L’insieme dei concetti e delle loro relazioni prende in questo caso il
	nome di ontologia (rete semantica).
}

\begin{figure}[h]
	\centering
	\includegraphics[scale=0.4]{04/ont.png}
	\caption{Ontologia.}
\end{figure}

\paragraph{Modi di interrogare un'ontologia:}

\begin{itemize}
	\item Un'istanza appartiene a una categoria?
	\item Un'istanza gode di una proprietà?
	\item Differenza tra categorie?
	\item Identificare varie istanze.
\end{itemize}

\dfn{Semantic Web}{
	Da World-Wide Web a Semantic
	Web: estensione del WWW in cui
	il materiale pubblicato è
	arricchito da metadati che
	abilitano l’interpretazione,
	l’inferenza, l’interrogazione,
	l’elaborazione automatica.
}

\cor{Resource Description Framework (RDF)}{
	Si tratta di un linguaggio di rappresentazione. È la base dei linguaggi OWL e SKOS che permettono di scrivere ontologie e FOAF (Friend of a Friend) per applicazioni sociali.
}

\clm{}{}{
	\begin{itemize}
		\item In RDF la conoscenza è espressa da statement, cioè triple soggetto –
		      predicato – oggetto: il predicato mette in relazione soggetto e
		      oggetto.
		\item Soggetto, predicato e oggetto sono IRI (internationalized resource
		      identifier, per esempio degli URL).
		\item RDFS (RDF Schemas) permette di realizzare tassonomie
		      appoggiandosi a RDF.
		\item Un insieme di triple costituisce un grafo RDF.
	\end{itemize}
}

\begin{figure}[h]
	\centering
	\includegraphics[scale=0.4]{04/pr.png}
	\caption{Relazione tra soggetto e oggetto.}
\end{figure}

\cor{Ontology Web Language(OWL)}{
	Linguaggio dichiarativo del semantic web ideato per definire ontologie tramite specifica di classi (categorie), proprietà, individui e valori.
}

\nt{Le ontologie OWL possono essere pubblicate sul web e
	riferite da altre ontologie, per costruire KB più complesse
	e raffinate.}

\paragraph{OWL prevede tre elementi:}

\begin{itemize}
	\item \fancyglitter{Entità:} elementi usati per riferirsi a oggetti del mondo
	      reale. Sono elementi atomici che possono essere usati
	      negli assiomi.
	\item \fancyglitter{Assiomi:} affermazioni (statement) di base espressi da
	      un’ontologia OWL.
	\item \fancyglitter{Espressioni:} combinazioni di entità che costituiscono
	      descrizioni complesse sulla base di altre.
\end{itemize}


\paragraph{Come costruire un'ontologia}

\begin{itemize}
	\item \fancyglitter{Identificazione dei concetti:}
	      \begin{itemize}
		      \item Elencare tutti i concetti riferiti nel DB di partenza.
		      \item I concetti sono solitamente catturati da sostantivi.
		      \item Definire per ciascuno un’etichetta e una breve descrizione.
		      \item Successivamente si identificano le sottoclassi.
	      \end{itemize}
	\item \fancyglitter{Identificazione delle proprietà:}
	      \begin{itemize}
		      \item Elencare tutte le relazioni catturate nel DB di partenza.
		      \item Le relazioni tipicamente sono esprimibili come verbi.
		      \item Definire per ciascuna un’etichetta e una breve descrizione.
	      \end{itemize}
\end{itemize}

\subsection{Allineamento Ontologico}

Un problema frequente è combinare concettualizzazioni sviluppate
separatamente e indipendentemente.

\dfn{Matching di Ontologie}{
	Date due ontologie $O_1$ e $O_2$ costruire un
	allineamento individuando le relazioni fra concetti corrispondenti.
}

\nt{La corrispondenza, in generale, sarà imperfetta.}

\dfn{FIPA Ontology}{
	Postula la presenza di un agente dedicato a gestire ontologie e che fornisce fra i suoi servizi:
	\begin{itemize}
		\item Discovery di ontologie pubbliche.
		\item Traduzione di espressioni in ontologie differenti.
		\item Rispondere a query relative alle differenze fra termini o
		      ontologie.
	\end{itemize}
}

\begin{figure}[h]
	\centering
	\includegraphics[scale=0.6]{04/allineamento.png}
	\caption{Allineamento ontologico.}
\end{figure}

\paragraph{Relazioni tra ontologie:}

\begin{itemize}
	\item \fancyglitter{Identiche:} .
	\item \fancyglitter{Equivalenti:} .
	\item \fancyglitter{Estensioni:} .
	\item \fancyglitter{Weakly-Translatable:} .
	\item \fancyglitter{Strongly-Translatable:} .
	\item \fancyglitter{Approx-Translatable:} .
\end{itemize}



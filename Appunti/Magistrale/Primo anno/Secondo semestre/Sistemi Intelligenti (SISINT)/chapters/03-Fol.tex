\chapter{Logica del Prim'Ordine}

\section{Introduzione alla Logica del Prim'Ordine}

\paragraph{La logica proposizionale è \fancyglitter{dichiarativa}:}

\begin{itemize}
	\item Separa nettamente conoscenza da inferenza.
	\item Consente di derivare fatti da fatti.
	\item La sua semantica è data da una relazione di verità che
	      collega formule e mondi possibili.
\end{itemize}

\nt{Però la logica proposizionale non permette rappresentazioni compatte e manca di espressività.}

\paragraph{Altre logiche:}

\begin{itemize}
	\item Logica temporale: permette di rappresentare e ragionare sul tempo, esempio “A
	      non è vero finché B non diventa vero”, “Quando A è vero subito
	      dopo B sarà vero”.
	\item Logica epistemica (della conoscenza): permette di esprimere relazioni come “l’agente i sa A”o “tutti
	      sanno A”e di ragionare sulle implicazioni.
	\item Logica deontica (normativa): permette di esprimere obblighi, permessi, proibizioni,
	      commitment e di ragionare su di essi.
	\item Logica fuzzy (a valori sfumati): introduce e ragiona su gradi di verità. I valori di verità
	      appartengono all’intervallo \[0,1\].
\end{itemize}

\dfn{Logica del Prim'Ordine}{
	Il mondo è fatto di oggetti in
	relazione fra di loro, una relazione può essere verificata
	oppure no.
}

\paragraph{Modelli:}

\begin{itemize}
	\item Proposizionale: attribuzione di valori di verità ai fatti (simboli proposizionali).
	\item Prim'ordine: contiene un dominio, cioè l’insieme degli oggetti del mondo
	      considerati, e delle relazioni fra tali oggetti.
\end{itemize}

\cor{Dominio di Riferimento}{
	Un dominio di riferimento è astratto in:
	\begin{itemize}
		\item Un insieme di oggetti (ognuno caratterizzato dalla propria
		      identità).
		\item Un insieme di relazioni ognuna espressa come insieme di tuple.
	\end{itemize}
}

\cor{Relazione}{
	Una relazione è un insieme di tuple costituite da oggetti del
	dominio.
}

\paragraph{Predicati e funzioni:}

\begin{itemize}
	\item Funzioni: dato un insieme di oggetti restituiscono un
	      oggetto.
	\item Predicati: dato un insieme di oggetti ne catturano una
	      proprietà, restituiscono vero o falso.
	\item Simboli:
	      \begin{itemize}
		      \item Costante.
		      \item Predicato.
		      \item Variabile.
	      \end{itemize}
\end{itemize}

\nt{Tutti i simboli hanno un'\fancyglitter{interpretazione}.}

\dfn{interpretazione}{
	Un modello è una coppia M = (D, I), dove D è il dominio del
	discorso e I è un’interpretazione. L’interpretazione è il
	fondamento per determinare il valore di verità delle formule. È
	un’associazione fra i simboli e gli oggetti del dominio del discorso.
}

\nt{Se si cambiano coerentemente i simboli le formule non cambieranno valori di verità, ma se si cambia l'interpretazione dei simboli le formule potranno cambiare valore di verità.}

\paragraph{Una formula è:}

\begin{itemize}
	\item È soddisfacibile quando esiste almeno
	      un modello che la rende vera.
	\item È valida quando è vera in tutti i modelli.
	\item È insoddisfacibile quando non è mai vera.
\end{itemize}

\paragraph{}










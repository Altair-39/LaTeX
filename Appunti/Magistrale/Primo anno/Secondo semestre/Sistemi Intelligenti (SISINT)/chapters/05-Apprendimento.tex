\chapter{Apprendimento e Reti Neurali}

\section{Apprendimento}

\subsection{Classificazione}

\paragraph{Il problema:}

\begin{itemize}
	\item Dati:
	      \begin{itemize}
		      \item Esempi.
		      \item Categorie/classi.
	      \end{itemize}
	\item Costruire:
	      \begin{itemize}
		      \item Una rappresentazione astratta (modello) che permetta di
		            associare in modo corretto nuove istanze alla classe (o alle
		            classi) di appartenenza.
	      \end{itemize}
\end{itemize}

\dfn{Apprendimento Supervisionato}{
	Gli esempi dal quale astrarre le definizioni delle classi hanno associata la classe a cui appartengono.
}

\begin{figure}[h]
	\centering
	\includegraphics[scale=0.4]{05/apprendimento.png}
	\caption{Schema generale.}
\end{figure}

\cor{Learning Set}{
	Per learning (o training) set si intende la collezione di dati usati per svolgere il
	compito di apprendimento. I dati sono divisi in istanze (o record o esempi). Ogni
	esempio è rappresentato da una tupla (x, y) dove x è a sua volta una tupla di valori
	di attributi descrittivi e y è la classe di appartenenza dell'istanza.
}

\paragraph{Uso dei modelli appresi:}

\begin{itemize}
	\item \fancyglitter{Predittivo:} viene usato per predire la classe
	      di appartenenza di istanze ignote
	      in fase di apprendimento.
	\item \fancyglitter{Descrittivo:} Viene usato come strumento esplicativo
	      che permette di evidenziare quali carat-
	      teristiche distinguono le diverse categorie.
\end{itemize}







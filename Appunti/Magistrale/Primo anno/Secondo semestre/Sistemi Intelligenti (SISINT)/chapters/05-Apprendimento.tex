\chapter{Apprendimento e Reti Neurali}

\section{Apprendimento}

\subsection{Introduzione alla Classificazione}

\paragraph{Il problema:}

\begin{itemize}
	\item Dati:
	      \begin{itemize}
		      \item Esempi.
		      \item Categorie/classi.
	      \end{itemize}
	\item Costruire:
	      \begin{itemize}
		      \item Una rappresentazione astratta (modello) che permetta di
		            associare in modo corretto nuove istanze alla classe (o alle
		            classi) di appartenenza.
	      \end{itemize}
\end{itemize}

\dfn{Apprendimento Supervisionato}{
	Gli esempi dal quale astrarre le definizioni delle classi hanno associata la classe a cui appartengono.
}

\begin{figure}[h]
	\centering
	\includegraphics[scale=0.4]{05/apprendimento.png}
	\caption{Schema generale.}
\end{figure}

\cor{Learning Set}{
	Per learning (o training) set si intende la collezione di dati usati per svolgere il
	compito di apprendimento. I dati sono divisi in istanze (o record o esempi). Ogni
	esempio è rappresentato da una tupla (x, y) dove x è a sua volta una tupla di valori
	di attributi descrittivi e y è la classe di appartenenza dell'istanza.
}

\paragraph{Uso dei modelli appresi:}

\begin{itemize}
	\item \fancyglitter{Predittivo:} viene usato per predire la classe
	      di appartenenza di istanze ignote
	      in fase di apprendimento.
	\item \fancyglitter{Descrittivo:} Viene usato come strumento esplicativo
	      che permette di evidenziare quali carat-
	      teristiche distinguono le diverse categorie.
\end{itemize}

\qs{}{Ma qual è la bontà dei modelli appresi?}

\dfn{Valutazione Sperimentale}{
	Il modello viene usato per classificare le istanze di un
	test set. La valutazione della bontà è fatta sulla base del comportamento di classificazione corretto/sbagliato su questi dati.
}

\paragraph{Proprietà:}

\begin{itemize}
	\item \fancyglitter{Accuratezza} = predizioni corrette / predizioni totali.
	\item \fancyglitter{Error Rate} = predizioni sbagliate / predizioni totali.
\end{itemize}

\cor{Matrice di Confusione}{
	Matrice quadrata $N x N$ (con $N$ numero di classi). Le righe indicano le classi reali di appartenenza, le colonne indicano le classi predette.
}

\nt{L'ideale è che tutte le predizioni stiano sulla diagonale principale.}

\begin{figure}[h]
	\centering
	\includegraphics[scale=0.4]{05/confusion.png}
	\caption{Matrice di confusione.}
\end{figure}

\cor{Matrice dei Costi}{
	Matrice $N x N$ (con $N$ numero di classi). Associa un costo allo indovinare/sbagliare una predizione.
}

\begin{figure}[h]
	\centering
	\includegraphics[scale=0.4]{05/cost.png}
	\caption{Matrice dei costi.}
\end{figure}

\subsection{Costruire un Modello}

\dfn{Rote Learning (Apprendimento Meccanico)}{
	Si tratta di memorizzare le varie istanze. Tramite confronti cerca un’istanza
	identica:
	\begin{itemize}
		\item Se la trova restituisce la classe corrispondente.
		\item Se non la trova prova a indovinare (cerca istanze simili utilizzando una misura di distanza).
	\end{itemize}
}

\paragraph{Strategie per decidere:}

\begin{itemize}
	\item Votazione a Maggioranza: la classe più votata vince.
	\item Votazione pesata: ogni voto ha un peso maggiore/minore a seconda
	      della “distanza” fra le istanze considerate.
\end{itemize}

\nt{
	In caso di votazione pesata, i pesi vengono
	calcolati, usati e poi dimenticati.
}

\paragraph{Algoritmi di apprendimento diversi producono modelli di tipo diverso:}

\begin{itemize}
	\item Alberi di decisione: albero.
	\item Sistemi a regole: if-then.
	\item Reti neurali: matrici di numeri.
	\item Apprendimento per rinforzo: distribuzioni di
	      probabilità e matrici di numeri.
\end{itemize}

\nt{Nell'apprendimento automatico non sono importanti i numeri, ma cosa essi rappresentato e come sono ottenuti.}

\dfn{Alberi di Decisione}{
	Sono strumenti di supporto alle decisioni che
	usano modelli strutturati ad albero, comunemente utilizzati per esempio per la definizione
	di strategie mirate al conseguimento di un goal.
}

\nt{Per esempio i sottomenu a tendina sono alberi di decisione.}

\begin{figure}[h]
	\centering
	\includegraphics[scale=0.4]{05/dec.png}
	\caption{Struttura di un albero di decisione.}
\end{figure}

\clm{}{}{
	\begin{itemize}
		\item Ogni test è su un attributo.
		\item Le foglie sono classi.
		\item A ogni branch dell'albero si prende una decisione sulla base di un test e si scende al nodo successivo.
		\item Un dataset noto è il dataset degli iris.
	\end{itemize}
}

\paragraph{Tipi di attributi:}

\begin{itemize}
	\item Binari: booleani.
	\item Nominali: che hanno un nome.
	\item Ordinali: per cui vale un ordine.
	\item Continui.
\end{itemize}





\chapter{Domande per Prepararsi per l'Esame}

\section{Parte 1 (PROLOG e CLINGO)}

\subsection{PROLOG}

\qs{}{Scrivere un semplice programma PROLOG che va in loop.}

\qs{}{Fare l'esempio del pinguino.}

\qs{}{Perché in PROLOG non è presente la negazione forte (negazione classica)?}

\qs{}{Come funziona la negazione per fallimento?}

\qs{}{Fare un esempio di negazione per fallimento.}

\qs{}{Tipi di fallimenti in PROLOG.}

\qs{}{Che cos'è la logica monotòna?}

\qs{}{Come funziona la ricerca SLD?}

\qs{}{Perché si usa la risoluzione SLD se è completa solo con le clausole di Horn?}

\qs{}{Spiegare il cut(!) e qual è il suo vantaggio.}

\qs{}{Scrivere un programmino con il cut(!).}

\qs{}{Cut(!) danneggia la correttezza o la completezza? Perché?}

\qs{}{Scrivere lo stack dell'interprete PROLOG di un codice in cui è presente il cut(!).}

\subsection{ASP}

\qs{}{Dire se un dato programma è PROLOG o ASP.}

\qs{}{Differenze tra PROLOG e CLINGO.}

\qs{}{PROLOG e ASP sono monotòni?}

\qs{}{Perché in ASP non c'è il cut(!)?}

\qs{}{Come funziona la negazione per fallimento in ASP?}

\qs{}{Che cos'è l'Integrity Constrain e a cosa serve?}

\qs{}{In PROLOG si può avere Integrity Constrain?}

\qs{}{Fare un esempio di codice ASP che risulta insoddisfacibile.}

\qs{}{Come modificare un semplice programma ASP per renderlo soddisfacibile.}

\qs{}{Che cos'è e a che cosa serve il ridotto di un programma?}

\qs{}{Fare il ridotto di un programma ASP rispetto a un insieme dato.}

\qs{}{Dire se un programma ASP presenta un answer set.}

\qs{}{Scrivere un programma che presenti due answer set diversi.}

\section{Parte 2}

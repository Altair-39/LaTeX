\chapter{Domande per Prepararsi per l'Esame}

\section{Parte 1 (PROLOG e CLINGO)}

\subsection{PROLOG}

\qs{}{Scrivere un semplice programma PROLOG che va in loop.}

\qs{}{Fare l'esempio del pinguino.}

\qs{}{Perché in PROLOG non è presente la negazione forte (negazione classica)?}

\qs{}{Come funziona la negazione per fallimento?}

\qs{}{Fare un esempio di negazione per fallimento.}

\qs{}{Tipi di fallimenti in PROLOG.}

\qs{}{Che cos'è la logica monotòna?}

\qs{}{Come funziona la ricerca SLD?}

\qs{}{Perché si usa la risoluzione SLD se è completa solo con le clausole di Horn?}

\qs{}{Spiegare il cut(!) e qual è il suo vantaggio.}

\qs{}{Scrivere un programmino con il cut(!).}

\qs{}{Cut(!) danneggia la correttezza o la completezza? Perché?}

\qs{}{Scrivere lo stack dell'interprete PROLOG di un codice in cui è presente il cut(!).}

\subsection{ASP}

\qs{}{Dire se un dato programma è PROLOG o ASP.}

\nt{Suggerimento: guardare se il programma ha cut (PROLOG) o no (ASP)}

\qs{}{Differenze tra PROLOG e CLINGO.}

\paragraph{Risposta:}

\begin{itemize}
  \item ASP (CLINGO) ha integrity constrain, PROLOG no.
  \item ASP è proposizionale, PROLOG è logica del primordine. 
\item In ASP l’ordine dei letterali non ha alcuna importanza. 
  \item Prolog è goal-directed, ASP no.
  \item In ASP non c'è il concetto di dimostrazione.
  \item La SLD-risoluzione del Prolog può portare a loop,
mentre gli ASP solver non lo consentono (aka. ASP non va in loop). 
\item PROLOG ha il cut(!), ASP no.
\item ASP ha sia la negazione per fallimento che la negazione classica, PROLOG solo la negazione per fallimento.
\end{itemize}

\qs{}{Esempio di Nixon pacifista.}

\begin{figure}[h]
    \centering
    \begin{minipage}{0.45\textwidth}
        \centering
        \includegraphics[scale=0.5]{Domande/nixon1.png}
        \caption{Codice di Nixon pacifista.}
    \end{minipage}
    \hfill
    \begin{minipage}{0.45\textwidth}
        \centering
        \includegraphics[scale=0.25]{Domande/nixon2.png}
        \caption{Modelli di Nixon pacifista.}
    \end{minipage}
\end{figure}


\qs{}{PROLOG e ASP sono monotòni?}

\qs{}{Perché in ASP non c'è il cut(!)?}

\paragraph{Risposta:} non esistono né una dimostrazione né backtracking, ASP si cerca i suoi modelli indipendentemente.

\qs{}{Come funziona la negazione per fallimento in ASP?}

\qs{}{Che cos'è l'Integrity Constrain e a cosa serve?}

\qs{}{In PROLOG si può avere Integrity Constrain?}

\qs{}{Fare un esempio di codice ASP che risulta insoddisfacibile.}

\begin{figure}[h]
    \centering
    \begin{minipage}{0.45\textwidth}
        \centering
        \includegraphics[scale=0.5]{Domande/tux1.png}
        \caption{In questo esempio si ha che tux vola ma non vola.}
    \end{minipage}
    \hfill
    \begin{minipage}{0.45\textwidth}
        \centering
        \includegraphics[scale=0.25]{Domande/tux2.png}
        \caption{Non esistono modelli che siano vero.}
    \end{minipage}
\end{figure}


\qs{}{Come modificare un semplice programma ASP per renderlo soddisfacibile.}

\nt{Suggerimento: rimuovere le contraddizioni.}

\qs{}{Che cos'è e a che cosa serve il ridotto di un programma?}

\dfn{Ridotto}{
  Il \newfancyglitter{ridotto} $P^S$ rispetto a un insieme di atomi S: 
  \begin{itemize}
    \item Rimuove ogni regola il cui corpo contiene $not L$, per $L \in S$. 
    \item Rimuove tutti i $not L$ dai corpi delle restanti regole.
  \end{itemize}

  $P^S$ non contiene atomi con negazione per fallimento: 
  \begin{itemize}
    \item Ha un unico answer set. 
    \item Se tale answer ser coincide con S, allora S è un answer set per P.
  \end{itemize}
}

\qs{}{Fare il ridotto di un programma ASP rispetto a un insieme dato.}

\begin{center}
  p :- a. 
a :- not b.
b :- not a.
\end{center}

\nt{In questo esempio il ridotto c'è per S = \{b\} e S = \{a, p\}}

\qs{}{Dire se un programma ASP presenta un answer set.}

\qs{}{Scrivere un programma che presenti due answer set diversi.}

\paragraph{Risposta:} banalmente si può scrivere il programma di Nixon pacifista. Ha un answer set in cui Nixon è pacifista e quacchero e un answer set in cui Nixon è repubblicano e guerrafondaio (non pacifista).

\section{Parte 2}

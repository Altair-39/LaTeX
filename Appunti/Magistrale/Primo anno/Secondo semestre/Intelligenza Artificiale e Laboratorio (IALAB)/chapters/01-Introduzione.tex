\chapter{Introduzione}

\section{Il Corso in Breve...}

\subsection{Motivazioni}

\dfn{Intelligenza Artificiale}{
  L'intelligenza artficiale (o IA, dalle iniziali delle due parole, in italiano) è una disciplina appartenente all'informatica che studia i fondamenti teorici, le metodologie e le tecniche che consentono la progettazione di sistemi hardware e sistemi di programmi software capaci di fornire all’elaboratore elettronico prestazioni che, a un osservatore comune, sembrerebbero essere di pertinenza esclusiva dell’intelligenza umana.
}

\nt{Meh, in realtà l'IA è una disciplina di confine. Però le tematiche sono prettamente informatiche.}

\paragraph{IA In breve:}

\begin{itemize}
  \item Area di ricerca dell'informatica. 
  \item Si occupa di tutto ciò che serve
per rendere un computer
intelligente come un essere
umano. 
\item Interessata a problemi \fancyglitter{intelligenti}: problemi per cui non esiste/non è noto un algoritmo di risoluzione\footnote{Tris, il labirinto, etc.}.
\end{itemize}

\nt{Il cubo di Rubik non è un gioco intelligente >:(}

\paragraph{Ci sono tante sotto-aree di ricerca:}

\begin{itemize}
  \item Rappresentazione della conoscenza e ragionamento. 
  \item Interpretazione/sintesi del linguaggio naturale. 
  \item Apprendimento automatico. 
  \item Pianificazione. 
  \item Robotica.
\end{itemize}

\paragraph{Si collega a tante discipline, oltre all'informatica:}

\begin{itemize}
  \item Filosofia. 
  \item Fisica. 
  \item Psicologia.
\end{itemize}

\subsubsection{}

Questo insegnamento ha l’obiettivo di approfondire le
conoscenze di Intelligenza Artificiale con particolare riguardo
alle capacità di un agente intelligente di fare \fancyglitter{inferenze} sulla
base di una \fancyglitter{rappresentazione esplicita della conoscenza} sul dominio. In questo corso si faranno anche sperimentazione di metodi di ragionamento basati sul
paradigma della \fancyglitter{programmazione logica}, sull’uso di
\fancyglitter{formalismi a regole} (CLIPS) e su \fancyglitter{reti bayesiane} (ragionamento probabilistico\footnote{Odio la probabilità con tutto il mio cuore <3}).

\paragraph{Programma:}

\begin{itemize}
  \item Dal punto di vista metodologico saranno a rontate problematiche relative a: 
    \begin{itemize}
      \item Meccanismi di ragionamento per calcolo dei predicati del primo
ordine. 
\item Programmazione logica.
\item Ragionamento non monotono. 
\item Answer set programming.
    \end{itemize}
  \item Queste metodologie verranno a rontate dal punto di vista sperimentale con
    l’introduzione dei principali costrutti del \fancyglitter{Prolog}, lo sviluppo di strategie di
ricerca in Prolog e l’utilizzo dell’ambiente \fancyglitter{CLINGO} nella risoluzione di
problemi in cui sia necessaria l’applicazione di meccanismi di ragionamento
non monotono e del paradigma dell’Answer Set Programming.
\end{itemize}


\qs{}{E le novità dell’AI che vanno di moda?}

\paragraph{Risposta:} vengono trattate in altri corsi (TLN, RNDL, AAUT, ELIVA, AGINT).



\chapter{Privacy - Introduzione}

\section{Il Corso in Breve...}

\paragraph{Obiettivi:}

\begin{itemize}
  \item Riconoscere problemi di privacy nella modellazione e nell'analisi dei dati. 
  \item Conoscenza di base su metodi per preservare la privacy.
\end{itemize}

\paragraph{Syllabus:}

\begin{enumerate}
  \item Il concetto di privacy e le leggi sulla privacy in differenti paesi. 
  \item Le sfide della privacy nell'era dei Big Data. 
  \item Sistemi di informazione. 
  \item Modelli statistici. 
  \item Attacchi alla privacy e modelli di anonimizzazione in database statistici. 
  \item Privacy differenziale. 
  \item Separazione dei dati.
\end{enumerate}

\section{Privacy e Leggi sulla Privacy}

\subsection{Che Cos'è la Privacy?}

\qs{}{Che cos'è la privacy?}

\dfn{Privacy 1}{
  La \newfancyglitter{privacy} può essere definita come il diritto a stare da soli.
}

\paragraph{Warren and Brandeis (1890), \textit{The Right to Privacy}:}

\begin{itemize}
  \item Una delle tesi più influenti nella storia americana. 
  \item GLi autori tentarono di trovare un modo di descrivere legalmente la privacy.
\end{itemize}

\nt{Esempio: il diritto di una persona a scegliere la seclusione dalle attenzioni altrui, il diritto di non essere osservati nella sfera privata.}

\dfn{Privacy 2}{
  La \newfancyglitter{privacy} può essere definita come accesso limitato alle informazioni.
}

\begin{itemize}
  \item Una persona deve essere libera di scegliere in che misura partecipare alla società senza che gli altri debbano sapere. 
  \item Godkin (1880): “nothing is better worthy of legal protection than
private life, or, in other words, the right of every man to keep his
affairs to himself, and to decide for himself to what extent they
shall be the subject of public observation and discussion.”. 
\item Bok (1989): la privacy è "the condition of being protected from
unwanted access by others—either physical access, personal
information, or attention.".
\end{itemize}

\dfn{Privacy 3}{La \newfancyglitter{privacy} può essere vista come controllo sull'informazione.}

\begin{itemize}
  \item Westin and Blom-Cooper (1970): "privacy is the claim of
individuals, groups, or institutions to determine for
themselves when, how, and to what extent information
about them is communicated to others.“. 
\item Fried (1968): "Privacy is not simply an absence of
information about us in the minds of others; rather it is the
control we have over information about ourselves.”.
\end{itemize}

\dfn{Privacy FIPS PUB 41}{
  Il diritto di un'entità ... a determinare il grado con il quale interagire con il proprio ambiente, compreso il grado con cui un'entità voglia condividere informazioni personali con gli altri.
}

\dfn{Privacy ISO}{
  Il diritto di un individuo a controllare o influenzare quali informazioni collegate a loro possono essere collezionate e salvate e da chi e a chi queste informazioni possono essere accedute.
}

\clm{}{}{
  Definizioni derivabili:
  \begin{itemize}
    \item La privacy è l'abilità di una persona di controllare la disponibilità di \fancyglitter{informaziioni} e la sua \fancyglitter{esposizione}.
    \item È collegata a essere abili a funzionare in una società \fancyglitter{anonimamente}.
  \end{itemize}
}

\paragraph{Edward Snowden files:} viene pubblicato il file contente le informazioni riguardo i dati raccolti dal NSA riguardo le chiamate di milioni di privati cittadini. 

\subsection{Perché la Privacy è Così Importante?}

\qs{}{
  Perché la privacy è così importante?
}

\paragraph{La privacy è importante perché:}


\begin{itemize}
  \item \fancyglitter{Sentimenti individuali:}
    \begin{itemize}
      \item Non confortabile: possesso dell'informazione. 
      \item Non sicuro: l'informazione può essere usata in modo improprio (furto d'identità). 
    \end{itemize}
  \item \fancyglitter{Le aziende hanno bisogno di:}
    \begin{itemize}
      \item Far si che i loro clienti si sentano al sicuro. 
      \item Mantenere una buona reputazione\footnote{Beh, visto che merda sono Twitter e META non credo gliene freghi qualcosa.}. 
      \item Proteggere sé stesse da ogni disputa legale. 
      \item Obbedire alle leggi.
    \end{itemize}
\end{itemize}

\paragraph{Tipi di privacy:}

\begin{itemize}
  \item Political privacy. 
  \item Consumer Privacy. 
  \item Medical privacy. 
  \item Private property. 
  \item Information/Data privacy.
\end{itemize}

\dfn{Data Privacy}{
  Il problema della \newfancyglitter{data privacy} quando dei dati unicamente associabili a un individuo sono collezionati e salvati.
}

\paragraph{Le fonti più comuni di dati affetti da data privacy:}

\begin{itemize}
  \item Record sanitari. 
  \item Investigazioni e processi criminali. 
  \item Transazioni finanziarie. 
  \item Tratti biologici (e. g. materiale genetico). 
  \item Residenza e posizione geografica. 
  \item Geolocalizzazione. 
  \item Utilizzo del web.
\end{itemize}

\clm{}{}{
  \begin{itemize}
    \item La sfida della data privacy è trovare un modo per \fancyglitter{condividere dati} proteggendo le informazioni che permettono di identificare una determinata persona.
      \begin{itemize}
        \item Per esempio negli ospedali i dati vengono trasferiti in maniera aggregata. 
        \item L'idea di condividere i dati in forma aggregata garantisce che solo dati non identificabili sono condivisi.
      \end{itemize}
    \item La protezione legale deò diritto alla privacy \fancyglitter{cambia drasticamente a seconda della nazione}.
  \end{itemize}
}

\subsection{Privacy negli Stati Uniti}

La data privacy non è molto regolata negli USA:

\begin{itemize}
  \item 
\end{itemize}





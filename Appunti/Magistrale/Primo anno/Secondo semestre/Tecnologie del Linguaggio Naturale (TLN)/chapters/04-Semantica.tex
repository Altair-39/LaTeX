\chapter{Semantica}


\qs{}{Si possono costruire le grammatiche a mano?}

\begin{itemize}
  \item Per l'italiano: L. Lesmo (1985 $\rightarrow$ 2014), Common Lisp.
  \item Per l'inglese: 
    \begin{itemize}
      \item SHRFLU (Winograd 1972). 
      \item CHAT-80 (1979 $\rightarrow$ 1982), sviluppato in Prolog. È un linguaggio naturale per un database sulla geografia.
    \end{itemize}
\end{itemize}

\section{Fondamenti di Semantica Computazionale}


\paragraph{Rappresentare il Significato:}

\begin{itemize}
  \item Quale forma per il significato?
    \begin{itemize}
      \item Tavole di un database, logica descrittiva, logica modale, AMR. 
    \end{itemize}
  \item Blackburn e Bos $\rightarrow$ Logica del primordine. 
  \item Il problema nasce per le frasi incomplete. 
  \item Lambda e logica del primordine per parole e sintagmi.
\end{itemize}

\subsection{Algoritmo Fondamentale della COmputer Science}

\begin{enumerate}
  \item Parsificare la frase per ottenere l'albero sintattico. 
  \item Cercare la semantica di ogni persona nel lessico. 
  \item Costruire la semantica per ogni sintagma:
    \begin{itemize}
      \item Bottom-up. 
      \item Syntax-driven: traduzione rule-to-rule.
    \end{itemize}
\end{enumerate}

\dfn{Principio di Composizionalità di Frege}{
  Il significato del tutto è determinato dal significato delle parti e dalla
maniera in cui sono combinate.
}

\paragraph{Il significato della frase è costruito:}
\begin{itemize}
  \item Dal significato delle parole $\rightarrow$ Lessico. 
  \item Risalendo le costruzioni semantiche $\rightarrow$ Regole semantiche.
\end{itemize}

\paragraph{Sistematicità:}

\begin{itemize}
  \item Come costruiamo il significato del VP? 
  \item Come specificare in che maniera combinare i pezzi? 
  \item Come rappresentare pezzi di formule?
\end{itemize}

\section{Lambda Astrazione}










\chapter{JADE e JASON}

\section{JADE}

\subsection{FIPA}

\dfn{FIPA}{
  Consorzio per la standardizzazione di sistemi ad agenti. Il suo obiettivo era quello di promuovere tecnologie interoperabili: sistemi di agenti intelligenti che lavorano insieme.
}

\qs{}{Cosa rientrà nelle competenze di FIPA?}

\begin{itemize}
  \item Gestione del ciclo di vita di un agente. 
  \item Come trasportare un messaggio. 
  \item La struttura di un messaggio. 
  \item Protocolli di interazione tra agenti. 
  \item Ontologie e sicurezza.
\end{itemize}

\nt{Gli agenti sono al di fuori degli obiettivi di FIPA.}

\paragraph{Caratteristiche che vengono assunte per gli agenti:}

\begin{itemize}
  \item Autonomi. 
  \item Reattivi. 
  \item Proattivi. 
  \item Goal-driven. 
  \item Sociali. 
  \item Adattivi. 
  \item Cognitivi.
\end{itemize}

\begin{figure}[!h]
    \centering
    \includegraphics[scale=0.35]{05/FIPA.png}
  \caption{Piattaforma FIPA per agenti.}
\end{figure}

\cor{Agent Management}{
  Gli agenti hanno una descrizione di sé stessi e dei servizi che vengono offerti. L'Agent Management sistem è a sua volta un agente con cui si interagisce con scambi di messaggi. In jade si può accedere così oppure accedere direttamente agli oggetti Java.
}

\begin{figure}[!h]
    \centering
    \includegraphics[scale=0.35]{05/messaggi.png}
  \caption{Specifica per i messaggi.}
\end{figure}

\subsection{Introduzione a JADE}

\qs{}{Che cos'è JADE?}

\paragraph{Risposta:} JADE è la piattaforma, all'interno del consorzio FIPA, che implementa lo standard sopra citato. Si presenta come codice Java puro. Offre una serie di librerie per agenti e un runtime envinronment per gli agenti creati. L'obiettivo è quello di "nascondere" FIPA al programmatore. 

\begin{figure}[!h]
    \centering
    \includegraphics[scale=0.45]{05/arch.png}
  \caption{Modello architetturale di JADE.}
\end{figure}

\paragraph{Caratteristiche importanti:}

\begin{itemize}
  \item \fancyglitter{HalloWorld Agent:}
    \begin{itemize}
      \item Un tipo di agente è creato estendendo \texttt{jade.core.Agent} e ridefinendo il metodo \texttt{setup()}. 
      \item Ogni agente è identificato da un AID (nome univoco e qualche indirizzo).
      \item A ogni agente è assegnato uno e un solo thread (per evitare deadlock).
    \end{itemize}
  \item \fancyglitter{Local names, GUID and addresses:}
    \begin{itemize}
      \item Il nome completo di un agente deve essere globalmente unico. 
      \item In una singola piattaforma JADE ci si riferisce agli agenti solamente mediante il loro nome. 
      \item È possibile creare AID. 
      \item Si possono passare argomenti agli agenti.
    \end{itemize}
  \item \fancyglitter{Terminazione:}
    \begin{itemize}
      \item Un agente termina quando è chiamato il metodo \texttt{doDelete()}. Questo metodo non cancella all'istante, ma segnala che l'agente deve essere eliminato (solo dopo aver completato la \texttt{setup()}. 
      \item Nella terminazione è chiamato il metodo \texttt{takeDown()} per fare operazioni di clean-up.
    \end{itemize}
\end{itemize}

\dfn{Classe Behaviour}{
Il lavoro si un agente è eseguito dalle sue "behaviours". Per far sì che un agente esegua un task è sufficiente definire una behaviour e aggiungerla all'agente. 
}

\paragraph{Ogni behaviour deve implementare:}

\begin{itemize}
  \item \texttt{void action():} cosa la behaviour fa. 
  \item \texttt{boolean done():} quando la behaviour finisce.
\end{itemize}

\nt{Un agente può eseguire più behaviours in parallelo, il loro scheduling è cooperativo e tutti avvengono nello stesso thread java.}

\begin{figure}[!h]
    \centering
    \includegraphics[scale=0.5]{05/exec.png}
  \caption{Modello di esecuzione di un agente.}
\end{figure}

\paragraph{Tipi di behaviour:}

\begin{itemize}
  \item \fancyglitter{One shot:}
    \begin{itemize}
      \item La loro \texttt{action()} viene eseguita solo una volta. 
      \item \texttt{done()} restituisce sempre true (non si deve reimplementare).
    \end{itemize}
  \item \fancyglitter{Cyclic:}
    \begin{itemize}
      \item Non si completano mai, \texttt{action()} effettua sempre le stesse operazioni. 
      \item \texttt{done()} restituisce sempre false (non si deve reimplementare).
    \end{itemize}
  \item \fancyglitter{Complex:}
    \begin{itemize}
      \item Hanno uno stato (e.g. DFA) e a seconda di esso svolgono operazioni differenti. 
      \item Completano quando una data condizione diventa vera.
    \end{itemize}
\end{itemize}

\paragraph{Altri tipi:}

\begin{itemize}
  \item \fancyglitter{WakerBehaviour:}
    \begin{itemize}
      \item Al posto di \texttt{action()} e \texttt{done()} si deve ridefinire \texttt{onWake()} che verrà eseguito dopo un certo timeout. 
      \item Subito dopo l'esecuzione il behaviour è completato.
    \end{itemize}
  \item \fancyglitter{TickerBehaviour:}
    \begin{itemize}
      \item Al posto di \texttt{action()} e \texttt{done()} si deve ridefinire \texttt{onTick()} che verrà eseguito periodicamente.
      \item L'esecuzione va avanti all'infinito o finché non viene invocato \texttt{stop()}. 
      \item Solitamente si usa per aggiungere periodicamente delle behaviours.
    \end{itemize}
\end{itemize}

\paragraph{Other shits on behaviours:}

\begin{itemize}
  \item \texttt{onStart():} invocato prima della prima esecuzione di \texttt{action()}. Sono operazioni che devono essere eseguite prima di ogni behaviour.
  \item \texttt{onEnd():} invocato dopo che \texttt{done()} restituisce true. Sono operazioni che devono essere eseguite alla fine di ogni behaviour.
  \item \texttt{removeBehaviour():} well, prova a indovinare cosa fa... RIMUOVE LA FUCKING BEHAVIOUR DAL POOL DI UN AGENTE\footnote{Non chiama \texttt{onEnd}.}. 
  \item Quando il pool di behaviours attive di un agente è vuoto esso entra in IDLE a il suo thread va in sleep.
\end{itemize}

\subsection{Comunicazione tra Agenti}

\dfn{Modello di Comunicazione}{
  Il modello di comunicazione è basato sul passaggio asincrono di messaggi. I messaggi sono definiti dal linguaggio ACL (FIPA).
}

\begin{figure}[!h]
    \centering
    \includegraphics[scale=0.65]{05/mess.png}
  \caption{Modello di comunicazione.}
\end{figure}

\paragraph{I messaggi:}

\begin{itemize}
  \item Invio: inviare i messaggi è la creazione di un messaggio ACL e l'invio del messaggio tramite \text{send(ACLMessage)}.
  \item Ricezione: per leggere i messaggi si usa \texttt{receive()}.
\end{itemize}

\paragraph{Bloccare una behaviour mentre si aspetta per un messaggio:}

\begin{itemize}
  \item Il metodo \texttt{block()} non blocca il thread, ma sposta il behaviour dal pool di comportamenti attivi in un pool di comportamenti bloccati. 
  \item Non è una chiamata bloccante. 
  \item Tutte le behaviours vengono riattivate quando arriva un nuovo messaggio.
\end{itemize}

\paragraph{Leggere selettivamente una coda di messaggi:}

\begin{itemize}
  \item \texttt{receive()} restituisce il primo messaggio nella coda di messaggi e lo rimuove. 
  \item Se ci sono più behaviours che aspettano un messaggio una può rubare il messaggio a un'altra behaviour interessata. 
      \item Per evitare questo è possibile leggere solo certi tipi di messaggi con certe caratteristiche.
\end{itemize}

\paragraph{Ricevere un messaggio in "modalità bloccante":}

\begin{itemize}
  \item Gli agenti hanno anche un metodo \texttt{blockingReceive()} che restituisce quando c'è un messaggio nella coda di messaggi.
  \item Questa è una chiamata bloccante, per cui nessuna altra behaviour può eseguire finché la behaviour che ha invocato il metodo non ha ricevuto un messaggio.
\end{itemize}

\dfn{Pagine Gialle}{
Un servizio che consente agli agenti di pubblicare uno o più servizi che forniscono in modo che gli altri agenti possano trovarli.
}

\nt{In JADE ci si può accedere come oggetto remoto o interagendo come fosse un agente con messaggi ACL (modalità imposta da FIPA).}

\begin{figure}[!h]
    \centering
    \includegraphics[scale=0.5]{05/yellow.png}
  \caption{Servizio di pagine gialle.}
\end{figure}

\pagebreak

\subsection{Funzionalità Avanzate}

\dfn{Protocolli di Interazione}{
  Hanno un insieme di tipi di messaggio (INFORM, REQUEST, PROPOSE, \dots) che consentono lo scambio di specifiche sequenze predefinite di messaggi durante una conversazione.
}

\begin{figure}[!h]
    \centering
    \includegraphics[scale=0.3]{05/FIPA_SUB.jpg}
  \caption{FIPA Protocol.}
\end{figure}

\section{JASON}

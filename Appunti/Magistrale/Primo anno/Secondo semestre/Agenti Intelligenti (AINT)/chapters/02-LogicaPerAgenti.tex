\chapter{Logica per Agenti}

La logica è stata sviluppata nel corso di molti secoli da matematici e filosofi per modellare il "ragionamento corretto" degli esseri umani e ha avuto un ruolo importante nello sviluppo dell'intelligenza artificiale per realizzare agenti e sistemi multiagente.

\paragraph{La logica fornisce strumenti formali per:}

\begin{itemize}
  \item \fancyglitter{La rappresentazione della conoscenza}: 
    \begin{itemize}
      \item Un linguaggio formale con una semantica precisa. 
      \item Rappresentazione dichiarativa della conoscenza.
    \end{itemize}
  \item \fancyglitter{Ragionamento automatico}:
    \begin{itemize}
      \item Refole di inferenza.
    \end{itemize}
\end{itemize}

\nt{Queste proprietà della logica hanno diverse qualità: per esempio possono spiegare in modo preciso i passi di ragionamento di un agente.}

\paragraph{Ruoli della logica per gli agenti:}

\begin{itemize}
  \item La logica può essere utilizzata da un agente intelligente per \fancyglitter{rappresentare la conoscenza} e per \fancyglitter{ragionare}. Dato che la conoscenza è espressa in un linguaggio formale, gli agenti possono usare metodi formali per derivare altra conoscenza. 
  \item La logica può servire per \fancyglitter{specificare il comportamento} di un agente intelligente. In questo caso la logica può essere usata per \fancyglitter{verificare} che l'agente si comporti come specificato, anche se l'agente non fa uso della logica nel suo funzionamento.
\end{itemize}

\paragraph{La logica classica:}

\begin{itemize}
  \item Normalmente, quando si parla di logica in ambito AI si intende la \fancyglitter{logica classica}, ossia la \fancyglitter{logica proposizionale} o la \fancyglitter{logica del primordine}. 
  \item Però la necessità di modellare concetti diversi e le esigenze di efficienza hanno portato l'AI all'uso di logiche diverse da quelle classiche e anche alla definizione di nuove logica, per esempio le logiche non monotone. 
  \item Nell'ambito degli agenti e dei sistemi multiagente si preferisce utilizzare la \fancyglitter{logica modale}.
\end{itemize}

\section{Richiami di Logica Classica}

\dfn{Logica Classica}{
  Le formule sono costituite da proposizioni atomiche appartenenti a un insieme $P$ e da connettivi logici secondo la seguente formulazione, con $p \in P, \phi, \psi$ sono formule:
  \begin{itemize}
    \item $p$. 
    \item $\phi \lor \psi$. 
    \item $\neg \phi$.
  \end{itemize}
}

\nt{Quelli presentati nella definizione sono solo alcuni dei connettivi (ci sono anche and, nand, nor, xor, implica, etc.).}
\subsection{Semantica}
\dfn{Semantica della Logica Proposizionale}{
  La semantica definisce la verità delle formule rispetto a ogni modello. Nella logica proposizionale, un modello  assegna un valore di verità (true o false) a ogni simbolo proposizionale.
}

\clm{}{}{
  \begin{itemize}
    \item Se $|P| = n$ ci sono $n$ modelli (tavole di verità). 
    \item Un modello può essere rappresentato come un insieme $M \subseteq P$ che contiene tutte le proposizioni atomiche che sono vere nel modello. Quelle che non appartengono a $M$ sono false.
  \end{itemize}
}

\dfn{Soddisfacibilità}{
  La semantica definisce una relazione di soddisfacibilità $M \vDash \phi$ di una formula $\phi$ in un modello $M$ (interpetazione):
  \begin{itemize}
    \item $M \vDash p$ se e solo se $p \in M$. 
    \item $M \vDash \phi \lor \psi$ se e solo se $M \vDash \phi$ o $M \vDash \psi$. 
    \item $M \vDash \neg \phi$ se e solo se $M \not \vDash \phi$. 
  \end{itemize}
}

\clm{}{}{
  \begin{itemize}
    \item Una formula è \fancyglitter{soddisfacibile} se è solo se c'è \fancyglitter{qualche} modlelo che la soddisfa. 
    \item Una formula è \fancyglitter{valida} se e solo se è soddisfatta da \fancyglitter{ogni modello} (tautologia).
  \end{itemize}
}

\subsection{Sistemi Deduttivi e Logica del Primordine}

\dfn{Modus Ponens}{
  $$\frac{\alpha \Rightarrow \beta \:\:\:\:\:\: \alpha}{\beta}$$
}

\nt{L'applicazione di una sequenza di regole di inferenza porta a una derivazione $KB \vdash \alpha$. Ovviamente ciò che si deriva da un insieme $KB$ è vero in tutti i modelli di $KB$.}

\dfn{Deduzione}{
  Data una base di conoscenza (insieme di formule) da $KB$, una formula $\alpha$ segue logicamente da $KB$: 
$$KB \vDash \alpha$$
se e solo se, in ogni modello in cui $KB$ è vera, anche $\alpha$ lo è.
}

\cor{Teorema di Deduzione}{
  Date due formule $\alpha$ e $\beta$, 
$$\alpha \vDash \beta$$ 
se e solo se la formula 
$$\alpha \Rightarrow \beta$$ 
è valida.
}

\nt{La logica classica del primordine estende la logica proposizionale con quantificatori universali ed esistenziali.}

\section{La Logica Modale}

\subsection{Inadeguatezza della Logica Classica}

Gli agenti sono descritti come \fancyglitter{sistemi intenzionali} attribuendo loro \fancyglitter{stati mentali}. Supponiamo di voler formulare, con la logica, la seguente frase: \fancyglitter{John crede che Superman voli}. In logica classica questo potrebbe essere espresso come: \textit{Bel(John, vola(Superman))}, dove \textit{Bel} è un predicato. 

\paragraph{Questa formulazione non funziona per almeno due ragioni:}

\begin{itemize}
  \item \fancyglitter{Ragione sintattica}: le formule della logica classica hanno la seguente struttura \textit{Predicato(Term, \dots, Term)}, però il secondo argomento di \textit{Bel} è una formula e non un termine come richiesto. 
\item \fancyglitter{Ragione semantica}: gli operatori intenzionali come \textit{Bel}sono \fancyglitter{referentially opaque}, ossia creano contesti chiusi in cui non è possibile sostituire una formula con una equivalente come in logica classica.
\end{itemize}







\subsection{Introduzione alla Logica Modale}











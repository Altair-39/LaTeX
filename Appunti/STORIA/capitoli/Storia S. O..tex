\chapter{Storia dei sistemi operativi}

\section{Prima dei sistemi operativi}

Il concetto di programma negli anni '40 era appenna accennato e molto
diverso da quello attuale. Per esempio, l'ENIAC, il primo computer
elettronico, era programmato tramite cablaggi elettrici. Il concetto di
programma memorizzato in memoria centrale è stato introdotto da John
von Neumann nel 1945 (con l?EDVAC e le istruzioni macchina).
Ma questa era una procedura manuale, a quei tempi non si pensava minimamente
al farsi aiutare da un programma per inserire un programma nel computer.

\dfn{Sistema operativo}{
    Un sistema operativo è un programma che aiuta a "far girare" gli altri
    programmi. 
}

Le cose iniziaro a cambiare dal 1955. Con il tempo i sistemi operativi sono diventati sempre
più complessi e sofisticati. Uno dei compiti storici degli S. O. era quello di rendere 
semplice agli utenti l'utilizzo di un computer.

Storicamente molti dei concetti ancora usati nei moderni S. O. (paginazione della memoria,
memoria virtuale, etc.) nascono dalla scarsità di risorse dei primi computer.

\section{Fase 1 : Job by Job / Open Shop}
\begin{itemize}
    \item I computer erano usati solo dai loro progettisti e dai loro collaboratori;
    \item Scrivere un programma veniva svolto in più passi:
    \begin{enumerate}
        \item Scrivere il programma su carta;
        \item Trasferire il programma su schede perforate;
        \item Nello slot prenotato il programmatore si recava nella sala del computer;
        \item Inseriva le schede perforate nel computer (se il computer non era guasto);
        \item Faceva partire il programma;
        \item Potevano essere lette le celle di memoria per seguire l'esecuzione del programma;
        \item L'output poteva essere stampato o convertito in schede perforate
    \end{enumerate}
    \item Se un programma andava storto era difficile il debugging;
    \item Si introduce la figura professionale dell'operatore addetto alla sala macchine
    che stampava la fotografia della memoria in caso di comportamenti anomali;
    \item Si caricava un solo programma alla volta;
    \item A metà degli anni '50 nascono i primi assembler.
\end{itemize}

\section{Fase 2 : Sistemi batch}

\begin{itemize}
    \item 
\end{itemize}
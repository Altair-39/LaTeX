\chapter{Thread}

\section{Introduzione}

\section{Thread in Java}

\dfn{Semaphore}{La classe Semaphore è stata introdotta per aiutare chi aveva programmato in C.
Semaphore(int n) è un semaforo con n permessi. Possiede i metodi:
\begin{itemize}
    \item acquire() - prende un permesso;
    \item release() - rilascia un permesso.
\end{itemize}
}

\nt{Ogni oggetto ha un proprio lock: un semaforo binario con una lista che, se utilizzato, blocca gli accessi
concorrenti garantendo la mutua esclusione. Lo scheduler del lock ha una propria politica di gestione.}

\dfn{Sincronizzazione}{
Per ogni classe Java è possibile definire dei metodi synchronized. Quando un thread invoca un metodo
synchronized, il thread acquisisce il lock dell'oggetto su cui è invocato il metodo. La sintassi è:
public synchronized void metodo() \{...sezione critica...\}

}

\section{Sincronizzazione lato server e lato client}

\dfn{Sincronizzazione lato server}{
\begin{itemize}
    \item [$\Rightarrow$] L'oggetto protegge le variabili condivise e offre metodi
    synchronized per operare su di esse per cui si auto-
    protegge dagli accessi esterni;
    \item [$\Rightarrow$] I client, invocando metodi synchronized sull'oggetto
    condiviso, automaticamente si sincronizzano
    nell'accesso all'oggetto stesso;
    \item [$\Rightarrow$] imitazioni: definire synchronized i metodi dell'oggetto
    potrebbe non essere sufficiente per sincronizzare le
    attività dei thread.
\end{itemize}
}

\dfn{Sincronizzazione lato client}{
\begin{itemize}
    \item [$\Rightarrow$] L'oggetto non viene protetto da accessi paralleli;
    \item [$\Rightarrow$] Tutti i client di un oggetto condiviso accedono all'oggetto
    attraverso blocchi sincronizzati sul lock dell'oggetto
    stesso;
    \item [$\Rightarrow$] Limitazioni: se un client non implementa correttamente
    gli accessi all'oggetto condiviso si ottiene un
    malfunzionamento;
    \item [$\Rightarrow$] Ma talvolta è necessario implementare la mutua
    esclusione in questo modo.
\end{itemize}
}
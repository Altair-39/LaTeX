\chapter{L'informatica a scuola}

\section{L'organizzazione del sistema scolastico italiano}

I gradi di istruzione in Italia sono:
\begin{enumerate}
    \item Scuola dell'infanzia;
    \item Scuola di primo grado (\evidence{ex elementari});
    \item Scuola secondaria di primo grado (\evidence{ex medie});
    \item Scuola secondaria di secondo grado (\evidence{ex superiori});
    \item Formazione superiore (università, master, dottorato).
\end{enumerate}

\subsection{Come si accede all'insegnamento?}

Per diventare docenti è necessario avere:
\begin{itemize}
    \item un titolo di accesso all'insegnamento (Laurea, Diploma, etc.);
    \item  l'\glitter{abilitazione all'insegnamento}.
\end{itemize}
\paragraph{}
Chi possiede solo il primo requisito può essere inserito nelle graduatorie di istituto di III fascia per incarichi di supplenza a \evidence{tempo determinato}\footnote{In realtà i criteri vengono aggiustati in base all'istituto, abbassando i requisiti}.
\paragraph{}
Quando si consegue l'abilitazione si può accedere alle graduatorie di istituto di II fascia e si può partecipare al \newfancyglitter{concorso}.

\nt{Dalle graduatorie dei concorsi, annualmente, si attinge per l'immissione in ruolo a tempo indeterminato.}

\dfn{Titoli di accesso all'insegnamento per la scuola dell'infanzia e primaria}{
Per accedere all'insegnamento per la scuola dell'infanzia e per la scuola primaria i titoli di accesso sono: Laurea in scienze della formazione primaria, Diploma di Istituto Magistrale o Diploma di Liceo Socio-Pedagogico conseguiti entro l'anno scolastico (2001-2002)\footnote{Tutti questi titoli sono anche abilitanti}.
}

\dfn{Titoli di accesso all'insegnamento per la scuola secondaria di I e II grado}{
Per accedere all'insegnamento per la scuola secondaria di I e II grado è necessario avere un determinato tipo di Laurea appartenente a una \evidence{classe di concorso}. In più bisogna acquisire 60 CFU nei settori: antro/psico/pedagogici/metodologico\footnote{Di cui 16 CFU di disciplina}.
}

\section{Cosa si può insegnare con la laurea magistrale in informatica?}

La \evidence{Laurea magistrale in informatica} dà accesso "diretto\footnote{Se si ha l'abilitazione}" alle classi di concorso:

\begin{itemize}
    \item A-41, Scienze e tecnologie informatiche;
    \item A-47, Scienze matematiche applicate.
\end{itemize}
\subsubsection{}

Mentre, prendendo dei crediti extra in altri settori, si può accedere a:

\begin{itemize}
    \item A-26, Matematica;
    \item A-28, Matematica e scienze;
    \item A-40, Scienze e tecnologie elettriche ed elettroniche.
\end{itemize}


\nt{Nelle scuole, per via di carenza di professori, molte cattedre di informatica sono affidate a docenti di matematica. Questo fa si che ci sia una preparazione molto eterogenea a seconda della scuola.}

\dfn{Le indicazioni nazionali}{
Per ogni scuola e per ogni indirizzo sono presenti le \evidence{indicazioni nazionali} in cui si elencano le competenze presupposte in ingresso, gli obiettivi di apprendimento e le competenze attese in uscita. Gli insegnanti devono progettare le loro attività didattiche a partire da queste indicazioni.
}
\chapter{Domande e risposte}

\section{La natura dell'informatica}

\qs{}{Quali sono i problemi dell'informatica?}

\sol{il termine "informatica" è spesso utilizzato in modo improprio nel linguaggio comune. 
per esempio, viene usato per riferirsi a computers, cellulari, televisori, etc. Ma l'informatica non rigurda
i computers che sono solo uno strumento, non il fine. Un altro problema è il termine "digitale" che, solitamente, 
viene usato come sinonimo di "informatica". In realtà, digitale si riferisce alla rappresentazione di un dato 
mediante un simbolo numerico e a tutte le tecnologie basate sui computers.}

\qs{}{Quali sono i problemi relativi all'insegnamento dell'informatica nel nostro sistema scolastico?}

\sol{}

\qs{}{Perchè è importante insegnare informatica fin dalla scuola primaria?}

\sol{}

\qs{}{Perchè è importante riflettere sulla natura dell'informatica prima di insegnarla?}

\sol{ perchè ciò andrà a impattare sul modo di insegnare la materia. A seconda della propria visione del
mondo si privilegeranno alcuni aspetti rispetto ad altri. Inoltre bisogna sempre chiedersi il perchè si studia informatica:
essa dà una chiave di lettura del mondo digitale che è sempre più presente nella nostra quotidianità.}

\qs{}{Quali sono le 3 anime/paradigmi che abbiamo discusso per inquadrare la natura dell'informatica?}

\sol{ i tre paradigmi sono: 

\begin{itemize}
    \item matematico: l'informatica vista come una scienza matematica che formalizza i problemi e li risolve mediante algoritmi;
    \item ingegneristico: l'informatica vista come un'ingegneria che si occupa di progettare e realizzare sistemi software; 
    \item scientifico: l'informatica vista come una scienza che studia i sistemi software e i processi di sviluppo empiricamente.
\end{itemize}
}
\section{Teorie dell'apprendimento}

\qs{}{Che cos'è il comportamentismo?}

\sol{il comportamentismo è una teoria dell'apprendimento che si basa sull'osservazione del comportamento. Essa prevede
di modellare un comportamento desiderabile. La valutazione si basa sui cambiamenti nei comportamenti degli alunni visti come "tabule rase".
Spesso erano previsti "rinforzi" ossia punizioni corporali. Questo approccio è istruttivista.}

\qs{}{Che cos'è il cognitivismo?}

\sol{il cognitivismo rappresenta un superamento del comportamentismo. Il cognitivismo è una teoria dell'apprendimento che si basa
su alcune idee principali:
\begin{itemize}
    \item il carico cognitivo: il carico di lavoro mentale che un individuo deve sostenere per svolgere un compito;
    \item gli schemi e i modelli mentali.
\end{itemize}

L'approccio cognitivista punta a far ricordare e applicare la conoscenza.
}
\qs{}{Che cos'è il costruttivismo? (socio-costruttivismo e costruttivismo cognitivo)}

\sol{ il costruttivismo (ideato da J. Piaget) è una teoria dell'apprendimento che si basa sullo scetticismo:
\begin{itemize}
    \item la conoscenza è frutto dell'esperienza;
    \item non c'è modo di sapere la "vera" verità.
\end{itemize}

Si ricorre alla "viabilità" per valutare la conoscenza: un'idea è valida se ha funzionato fino a quel momento. Per le azioni fisiche è viabile tutto ciò che
porta a un risultato. Sul piano concettuale ci si basa sulla non contradditorietà.

\begin{itemize}
    \item [$\Rightarrow$] Socio-costruttivismo: l'apprendimento è un processo sociale che avviene in un contesto sociale. Si crea insieme nuova conoscenza;
    \item [$\Rightarrow$] Costruttivismo cognitivo: l'apprendimento è il processo di costruzione del significato. L'insegnante deve solo facilitare la scoperta
    offrendo le risorse necessarie.
\end{itemize}
}
\qs{}{Che cos'è il costruzionismo?}

\sol{il costruzionismo (ideato da S. Papert) si basa sull'idea costruttivista di conoscenza. Ma a ciò viene aggiunta l'idea che la conoscenza deve essere
finalizzata alla costruzione di artefatti.}

\qs{}{Quali sono i punti fondamentali del brano di Papert "Gli ingranaggi della mia infanzia"?}

\sol{}

\qs{}{Cos'è l'assimilazione?}

\sol{l'assimilazione è l'incorporazione di un determinato concetto in uno schema che è stato già acquisito.}

\qs{}{Cos'è l'accomodamento?}

\sol{l'accomodamento è la modifica di una struttura cognitiva in relazione al contatto con nuove informazioni.}

\qs{}{Cos'è la zona di sviluppo prossimo (ZSP)?}

\sol{la zona di sviluppo prossimo è la seconda delle tre aree di apprendimento di un bambino. Nella ZSP il bambino è in grado di apprendere solo con il supporto del docente ed è in quest'area che l'insegnante può intervenire.}

\qs{}{Quali sono le caratteristiche principali dell'apprendimento attivo?}

\sol{}
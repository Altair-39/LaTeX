\chapter{Domande}

\section{Lullo, Leibniz e la storia del calcolo}

\qs{}{Che cos'è l'Ars Magna di Lullo?}

\paragraph{Risposta:} L'Ars Magna di Lullo è un'opera scritta da Raimondo Lullo, un filosofo, teologo e mistico catalano,
che trattava di un metodo per risolvere i problemi logici attraverso l'uso di diagrammi e figure.
In questo modo si potevano raggiungere verità in ogni campo del sapere.

\subsubsection{}

\qs{}{Quali sono le funzioni della combinatoria di Lullo?}

\paragraph{Risposta:} L'Ars Combinatoria di Lullo è un metodo inventivo
che permette di elaborare dimostrazioni allo scopo di convertire gli "infedeli" al cristianesimo.   

\subsubsection{}

\qs{}{Elencare almeno tre dei principi alla base della caratteristica universale di Leibniz.}

\paragraph{Risposta:}

\begin{itemize}
    \item [$\Rightarrow$] Le idee sono analizzabili;
    \item [$\Rightarrow$] L'analisi termina con le idee primitive;
    \item [$\Rightarrow$] Le idee possono essere rappresentate da simboli;
    \item [$\Rightarrow$] Le relazioni tra le idee possono essere rappresentate da simboli;
    \item [$\Rightarrow$] Le idee possono essere combinate per ottenere nuove idee tramite opportune regole.
\end{itemize}

\subsubsection{}

\qs{}{Elencare le principali caratteristiche della lingua universale.}

\paragraph{Risposta:}

\begin{itemize}
    \item [$\Rightarrow$] Segni che rappresentano direttamente le nozioni e le cose, non le parole;
    \item [$\Rightarrow$] Segni composti di figure geometiche e di pitture;
    \item [$\Rightarrow$] Direttamente collegata con l'enciclopedia;
    \item [$\Rightarrow$] Connessioni tra i caratteri corrispondono alle connessioni tra le cose;
    \item [$\Rightarrow$] I caratteri della lingua universale esprimono relazioni tra pensieri.
\end{itemize}

\subsubsection{}

\qs{}{Fare un esempio di proposizione dal frammento XX di Leibniz.}

\paragraph{Risposta:} Se $A \geq B$ e $B \geq C$, allora $A \geq C$.

\subsubsection{}

\qs{}{Qual è l'idea alla base di un approccio "pointless" alle nozioni geometriche e topologiche?}

\paragraph{Risposta:} L'idea alla base di un approccio "pointless" è quella che la classe
degli spazi aperti di uno spazio topologico costituisce un reticolo rispetto all'inclusione
per cui vale la proprietà di distributività.

\subsubsection{}

\qs{}{Che cosa è lo Entscheidungsproblem formulato da Hilbert e Ackermann nel 1928?}

\paragraph{Risposta:} L'Entscheidungsproblem è un problema matematico che consiste nel determinare
un modo combinatorio finito per il quale combinazioni di simboli primitivi conducono a dimostrazioni.
In altre parole, trovare una procedura che consente di decidere la validità di una data
espressione logica con un numero finito di operazioni.

\section{Turing e la fisica del calcolo}

\qs{}{Perché Turing richiede che i simboli che possono essere scritti sul nastro 
di una macchina di Turing siano elementi di un insieme finito?}

\paragraph{Risposta:} 

\subsubsection{}

\section{Funzioni calcolabili e combinatori}

\section{"As we may think"}

\section{Engelbart e Nelson}

\section{Otlet, Lickider e Kay}
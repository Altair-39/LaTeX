\chapter{Introduzione}

Il corso si divide in una serie di sezioni:
\begin{itemize}
    \item Macchine mai esistite (Knowledge
    Navigator, Dynabook, Memex): poichè la storia dell'informatica
    è una storia di idee, di pionieri e di visionari;
    \item Il modo di organizzare i testi in rapporto all'inondazione 
    informativa: workstation di Otlet (non realizzata), the mother of all
    demos (Doug Engelbart, 1968), gli ipertesti di Ted Nelson, libraries
    of the future;
    \item La cybernetica: originata da Norbert Wiener, ci si concentra sugli
    aspetti matematici della neurofisiologia che portarono al modello formale di
    neurone (McCulloch e Pitts, 1943) utilizzato da Von Neumann per la
    definizione della sua architettura di calcolatore;
    \item Gli hippies e la controcultura: si parla di come la controcultura 
    abbia influenzato la nascita dell'informatica;
    \item La semiotica: si parte in ordine cronologico dalla preistoria, passando per Lullo fino
    a Leibniz, per arrivare ai sistemi formali.
\end{itemize}

\section{La semiotica}

\subsection{La preistoria}
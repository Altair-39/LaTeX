\chapter{Nelson e Kay}

\section{Verso gli ipertesti: Ted Nelson}

Ted Nelson nacque nel 1937, i suoi genitori (un'attrice e un regista) divorziano nel 1939 e lui viene cresciuto dai nonni
materni. Nelson studiò filosofia e si laureò in sociologia (1963) prendendo un dottorato nel 
2002 (in realtà la sua tesi è una riproposizione di idee avute nel 1959).
Le sue pubblicazioni più importanti sono:

\begin{itemize}
    \item [$\Rightarrow$] \fancyglitter{Complex Information Processing: A File Structure for the Complex, the Changing and the Indeterminate} (1965): 
    articolo tecnico on cui si introduce il concetto di \fancyglitter{ipertesto} con le strutture dati collegate;
    \item [$\Rightarrow$] \fancyglitter{Computer Lib/Dream Machines} (1974): libro autopubblicato.
    Il titolo è doppio perché il libro è doppio, ossia si legge in due direzioni.
    Inoltre la grafica richiama una pubblicazione della controcultura hippie (curata da Stewart Brand),
    ossia il \fancyglitter{Whole Earth Catalog}\footnote{Catalogo di oggetti (in particolar modo libri)
    messi in vendita. Molti testi appartenevano alla libreria dello Xerox PARC.};
    \item [$\Rightarrow$] \fancyglitter{Literary Machines} (1981): descrizione del sistema Xanadu.
\end{itemize}

\clm{}{}{

\paragraph{Nelson ha anche girato un video "estremo":}

\begin{itemize}
    \item [$\Rightarrow$] \fancyglitter{Silicon Valley Story} (1992): non si capisce niente, surreale. Ma compaiono personaggi come 
    Timoty Leary (profeta delle droghe psichedeliche\footnote{AKA psicologo di Harvard.}), Douglas Engelbart e Stuart Brand.
\end{itemize}
}


\subsection{Nelson e il Memex}

\section{Dal Dynabook al Knowledge Navigator: Alan Kay}

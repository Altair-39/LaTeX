\chapter{Seconda consegna}

\nt{\paragraph{\textcolor{red}{Obiettivi sulle conoscenze (K):}} lo studente è in grado di \textbf{\textit{capire se un determinato pseudo-algoritmo rappresenti un algoritmo reale}}.
Lo studente è in grado di \textbf{\textit{definire cosa è un algoritmo}}.
\paragraph{\textcolor{red}{Obiettivi sulle abilità (A):}} lo studente è in grado di \textbf{\textit{discutere con i compagni le proprie idee}}.
\paragraph{\textcolor{red}{Obiettivi sulle competenze (C):}} lo studente è in grado di \textbf{riconoscere un algoritmo reale}.
}

\section{Proposta di suddivisione in fasi}

\begin{enumerate}[label=\roman*.]
    \item \textbf{Fase 1:} lettura e analisi, a coppie, del testo degli "algoritmi".
    \begin{enumerate}
        \item \textbf{\textcolor{cyan}{Consegna:}} leggere e analizzare gli pseudo-algoritmi proposti nel documento cercando di capire quali rappresentino algoritmi.
        \item \textbf{\textcolor{magenta}{Svolgimento:}} gli alunni vengono divisi in coppie. Ogni coppia legge il testo proposto e cerchi di capire se gli algoritmi siano reali o meno. 
        \item \textbf{\textcolor{teal}{Discussione:}} ciascun elemento della coppia discuta con l'altro quali algoritmi siano reali e quali no.
        \item \textbf{\textcolor{orange}{Conclusione:}} l'insegnante unisce le coppie, a due a due, e introduce la fase successiva.
    \end{enumerate}
    \item \textbf{Fase 2:} analisi, a gruppi, degli "algoritmi".
    \begin{enumerate}
        \item \textbf{\textcolor{cyan}{Consegna:}} a gruppi, confrontare le proprie risposte ottenute nella fase precedente.
        \item \textbf{\textcolor{magenta}{Svolgimento:}} ciascuna coppia, all'interno del gruppo, discute con l'altra coppia le proprie risposte. 
        \item \textbf{\textcolor{teal}{Discussione:}} ogni gruppo decide quali siano le risposte corrette.
        \item \textbf{\textcolor{orange}{Conclusione:}} l'insegnate introduce la fase successiva.
    \end{enumerate}
    \item \textbf{Fase 3:} definizione di algoritmo.
    \begin{enumerate}
        \item \textbf{\textcolor{cyan}{Consegna:}} ogni gruppo definisca che cos'è un algoritmo, partendo dalle proprie risposte.
        \item \textbf{\textcolor{magenta}{Svolgimento:}} ciascun gruppo discute e cerca di definire che cos'è un algoritmo. 
        \item \textbf{\textcolor{teal}{Discussione:}} ogni gruppo propone alla classe la propria definizione e la discute. L'insegnate deve mettere in risalto le differenze e le somiglianze nelle definizioni.
        \item \textbf{\textcolor{orange}{Conclusione:}} il docente presenta una definizione "consolidata" di algoritmo e la mette a confronto con le definizioni trovate dagli studenti.
    \end{enumerate}
\end{enumerate}

\section{Snodi}

\begin{enumerate}[label=\roman*.]
    \item \textbf{Fase 1:}
    \begin{enumerate}
        \item Comprendere il testo di uno pseudo-algoritmo.
        \item Analizzare uno pseudo-algoritmo.
        \item Trovare un modo per stabilire se un algoritmo sia reale o meno.
    \end{enumerate}
    \item \textbf{Fase 2:}
    \begin{enumerate}
        \item Lavorare in gruppo per trovare una soluzione condivisa.
    \end{enumerate}
    \item \textbf{Fase 3:}
    \begin{enumerate}
        \item Capire che cosa si intende con la parola "algoritmo".
        \item Trovare le caratteristiche che definiscono un algoritmo.
    \end{enumerate}
\end{enumerate}

\section{Indicatori}

\begin{enumerate}[label=\roman*.]
    \item \textbf{Fase 1:}
    \begin{enumerate}
        \item Quali sono le caratteristiche di un algoritmo?
        \item Un algoritmo termina sempre?
        \item Un algoritmo deve essere preciso?
    \end{enumerate}
    \item \textbf{Fase 2:}
    \begin{enumerate}
        \item Come mai si hanno opinioni diverse?
        \item Si hanno dei punti in comune?
    \end{enumerate}
    \item \textbf{Fase 3:}
    \begin{enumerate}
        \item Commentare gli pseudo-algoritmi in modo da evidenziare le caratteristiche che un algoritmo reale deve o non deve avere.
        \item Cosa si intende per "passo" di un algoritmo?
        \item Ci sono algoritmi più efficienti di altri?
        \item Quali sono altre proprietà che un algoritmo deve avere?
    \end{enumerate}
\end{enumerate}
\chapter{Prima consegna}

\section{Punti su cui sono d'accordo}
\subsection{Frasi}
\begin{enumerate}
    \item "\textit{Certe attività cognitive non sono più dominio esclusivo dell’umanità: lo vediamo in tutta una serie di giochi da scacchiera (dama, scacchi, go, …) un tempo unità di misura per l’intelligenza e nei quali ormai il computer batte regolarmente i campioni del mondo.}", \textbf{\fancyglitter{Informatica: la terza rivoluzione "dei rapporti di potere"}};  
    \item "\textit{Per preparare i cittadini alla società industriale, nei due secoli passati, non sono state date agli studenti competenze operative sui macchinari industriali, ma sono state inserite nelle scuole le discipline scientifiche che ne spiegavano i princìpi scientifici alla base.}", \textbf{\newfancyglitter{Informatica e competenze digitali: cosa insegnare?}};
    \item "\textit{Le tecnologie digitali dovrebbero essere progettate per promuovere la democrazia
e l‘inclusione. }", \textbf{\evidence{Manifesto di Vienna per l'umanesimo digitale}.}
\end{enumerate}

\subsection{Motivazioni}
\begin{enumerate}
    \item Attualmente l'essere umano con l'ELO\footnote{Sistema di valutazione della forza relativa di uno scacchista} più elevato è \textbf{\fancyglitter{Magnus Carlsen}}\footnote{Campione del mondo dal 2013 al 2023} che ha raggiunto \href{https://www.chess.com/it/article/view/magnus-carlsen-campione-mondo-scacchi-retrospettiva#elo}{un picco di 2882 nella variante classica}. \textbf{\fancyglitter{Stockfish}}, il più forte motore scacchistico attualmente esistente, ha un ELO stimato di circa 3600. Una differenza abisssale. Inoltre è molto interessante osservare le partite tra vari computers in cui la probabilità che Stockfish perda con il bianco è quasi e, inoltre riesce anche a vincere con il nero\footnote{Statisticamente, ad alti livelli, è molto più difficile vincere con il nero dato che non si ha il bonus dell'iniziativa}. Alcune delle mosse che vengono fatte in questi "scontri" sono basati su calcoli molto complessi e per molti esseri umani, inclusi dei granmaestri (\textbf{\fancyglitter{GM}}), risultano innaturali. Oggi vengono usati \textit{engine}, dagli stessi GM, per allenarsi poiché un essere umano non giocherà sempre la mossa migliore, un computer sì;
    \item Come ricordato nell'articolo sono importanti sia le competenze digitali che quelle informatiche, tuttavia bisogna mettere in evidenza una realtà ineluttabile: le tecnologie si \textbf{\newfancyglitter{evolvono continuamente}}. Ciò significa che anche gli applicativi cambino nel tempo e, il compito della scuola non è solo quello di preparare i ragazzi al momento corrente, ma di fornire le capacità per \textbf{\newfancyglitter{adattarsi}} ai nuovi modelli. Ciò può essere fatto insegnando l'informatica perché, anche se passa il tempo, le strutture fondamentali non ricevono quasi nessun cambiamento\footnote{Ci possono essere delle "rivoluzioni" estreme nel modo di concepire la base di una tecnologia, ma sono casi estremi e limitati}. Questo fa si che "insegnare la programmazione" sia molto più efficace che "insegnare un determinato linguaggio" o, per mantenerci più sul generico "insegnare come si usa un editor di testo" sia più efficacie che "insegnare come si usa Word/Office";
    \item Tutt*  dovrebbero essere liber* di esprimersi liberamente nella società della tecnologia. Purtroppo l'aumento della platea dei beneficiari di queste tecnologie può aver contribuito ad aumentare l'odio verso il diverso dando voce a persone la cui opinione, un tempo, sarebbe stata relegata alle quattro mura di un bar il sabato sera. La frase in sé e per sé è corretta, ma per trasformare l'inchiostro in realtà ci vorrà ancora molto tempo e forse non si raggiungerà mai una tecnologia tale da permettere una profonda e sincera "\evidence{\textbf{inclusione}}".
\end{enumerate}

\section{Punti su cui non sono d'accordo}
\subsection{Frasi}
\begin{enumerate}
    \item "I professionisti di tutto il mondo dovrebbero riconoscere la loro corresponsabilità
    nell‘impatto sociale delle tecnologie digitali. Devono capire che nessuna tecnologia è
    neutrale ed essere sensibilizzati a considerare sia i potenziali benefici sia i possibili aspetti
    negativi.", \textbf{\evidence{Manifesto di Vienna per l'umanesimo digitale}};
    \item "Per preparare i cittadini alla società industriale, nei due secoli passati, non sono state date agli studenti competenze operative sui macchinari industriali, ma sono state inserite nelle scuole le discipline scientifiche che ne spiegavano i princìpi scientifici alla base.", \newfancyglitter{\textbf{Informatica e competenze digitali: cosa insegnare?}}. 
\end{enumerate}

\subsection{Motivazioni}
\begin{enumerate}
    \item Nonostante io concordi con l'esaminare sia i beneficiche gli aspetti negativi non posso
    essere d'accordo con la parte della frase in cui si afferma che "nessuna tecnologia è neutrale".
    Questo perché è l'uso che se ne fa a essere negativo o positivo. Spesso, negli ultimi anni, si sentono
    studiosi parlare del fatto che la tecnologia non è neutra, ma io lo vedo come un tentativo di deresponsabilizzarsi.
    Basta immaginarsi un mondo senza esseri umani: in tal mondo la tecnologia non potrebbe essere usata dato che non ci
    sarebbe nessuno capace di usarla. E se è la presenza dell'uomo a rendere una tecnologia negativa o positiva allora 
    non è forse vero che la tecnologia è neutra?
    \item Questa frase è molto interessante perché, in un certo senso, è vero che non si è insegnato l'utilizzo dei macchinari 
    industriali nella società industriale, ma bisogna tener presente che, all'epoca, la scuola non era ancora così accessibile.
    Solitamente a scuola andavano persone con un certo livello economico che raramente sarebbero diventate operai, quindi non 
    serviva loro una formazione specifica sull'utilizzo dei macchinari. Questo tipo di formazione veniva invece fatta fare agli
    operai che, spesso, non avevano nemmeno la licenza elementare.
\end{enumerate}

\section{Considerazioni sui 3 paradigmi}
\subsection{Analisi dei 3 paradigmi}

\dfn{Paradigma matematico}{Per il \newfancyglitter{paradigma matematico} si  formalizza un linguaggio che garantisce certe proprietà (il tutto prima dell'esecuzione).}

\dfn{Paradigma ingegneristico}{Per il \newfancyglitter{paradigma ingegneristico} si devono fare molti test di affidabilità con diversi input (unit test, a run time).}

\dfn{Paradigma scientifico}{Per il \newfancyglitter{paradigma scientifico} si deve validare empiricamente la correttezza di un programma.}
\qs{}{Dove si trovano, nell'informatica, questi paradigmi?}

\begin{itemize}
    \item \textbf{Paradigma matematico:} questo paradigma trova la sua massima espressione nei linguaggi funzionali. Per esempio, in Haskell\footnote{Linguaggio funzionale puro} ogni cosa è una funzione matematica. Il punto di forza dei linguaggi funzionali (lazy) è che la loro correttezza è vera a priori per cui, se un programma viene eseguito allora è corretto. Inoltre esistono linguaggi come Agda, Coq, etc. che servono per dimostrare matematicamente la correttezza formale di alcuni programmi (purtroppo non sono Turing completi);
    \item \textbf{Paradigma ingegneristico:} si basa su un intenso uso di Unit test e testing generici per assicurare una "correttezza" su un ampio insieme di casi (spesso nei casi limite). Questo implica una correttezza a posteriori per cui, il programma deve prima essere eseguito;
    \item \textbf{Paradigma scientifico:} si effettuano delle ipotesi e delle deduzioni. Un esempio di ciò è la logica di Floyd-Hoare in cui vengono poste delle pre-condizioni (ipotesi sui dati) e delle post-condizioni (dati attesi se il programma è corretto). 
\end{itemize}

\subsection{Conclusioni generali}

Tutti e tre i paradigmi mostrano un differente modo di osservare la realtà. É difficile metterli in una classifica o dire quale sia il più corretto perché ciò dipende in gran parte dal background personale del singolo: un logico probabilmente sarà orientato verso il paradigma matematico, un tecnico verso quello ingegneristico e un chimico o un biologo verso quello scientifico. L'informatica è qualcosa di troppo complesso per essere ridotto a un solo di questi paradigmi. Come mostrato nella sezione precedente ogni paradigma include un pezzo dell'informatica quindi è inutile affermare che appartiene al paradigma X o al paradigma Y\footnote{Per usare un termine informatico possiamo dire che l'informatica, così come C++ o Java, è \textbf{multiparadigma}}

\section{L'informatica può essere considerata una scienza?}

\subsection{I requisiti}

\begin{itemize}
    \item Organizzati per comprendere, sfruttare e far fronte a un fenomeno pervasivo: esiste un intero campo di applicazione dell'informatica a questo livello ossia la \fancyglitter{\textbf{data science}};
    \item Comprende i processi naturali e artificiali del fenomeno: l'informatica vuole comprendere e replicare in modo automatico dei fenomeni sia naturali che artificiali. Basti vedere alcune strutture chiaramente ispirate alla natura come gli \newfancyglitter{\textbf{alberi}}; 
    \item Corpo di conoscenza strutturato e codificato: l'informatica ha un compendio di conoscenze ben identificabile e strutturato in vari argomenti. Per esempio: le \evidence{\textbf{basi di dati}}, gli \evidence{\textbf{algoritmi}}, etc.;
    \item Impegno verso metodi sperimentali per la scoperta e la validazione: ci sono parti dell'informatica con carattere prevalentemente sperimentale. Si è parlato nel capitolo precedente del paradigma ingegneristico che ha come oggetto il testing per validare i programmi;
    \item Riproducibilità dei risultati: i risultati dei programmi sono riproducibili se si possiede un interprete per un dato linguaggio e una macchina su cui eseguirli;
    \item Falsificabilità di ipotesi e modelli: usando il \fancyglitter{\textbf{moduls tollens}}\footnote{Se A $\rightarrow$ B, $\neg$B $\rightarrow$ $\neg$A} si può falsificare un programma. Per esempio, se un programma non termina allora non è corretto;
    \item Abilità di fare predizioni: l'informatica è in grado di predire il comportamento di un programma. Per esempio, se un programma è corretto allora termina;
\end{itemize}

\subsection{Conclusioni}

In conclusione si può ritenere l'informatica una scienza dato che soddisfa tutti i requisiti. Inoltre, come mostrato nella sezione precedente, l'informatica è un campo molto vasto che include molti altri campi. Per esempio, l'informatica è strettamente 
legata alla matematica e alla logica. Inoltre, l'informatica è strettamente legata alla fisica e alla chimica poiché, per esempio, un computer è un dispositivo fisico che sfrutta la chimica per funzionare. Infine, l'informatica è strettamente legata alla 
biologia poiché, per esempio, l'informatica è usata per studiare il DNA e per simulare la vita (si veda il Game of Life di Conway, gli automi cellulari e la vita artificiale).